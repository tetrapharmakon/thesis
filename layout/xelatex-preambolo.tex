\usepackage[usenames,dvipsnames]{xcolor}
  \definecolor{semilightgray}{rgb}{0.65, 0.65, 0.65}
  \def\gray#1{\textcolor{semilightgray}{#1}}

\emergencystretch=1.5em

\usepackage[pages=some,placement=top]{background}

\usepackage{%
   manfnt
  ,marginnote
  ,url
  ,cancel
  ,verse
  ,tipa
  ,chessfss
  ,stmaryrd
  ,todonotes
  ,mathtools
  % ,eucal
  ,listings
  ,datetime}


\lstset{language=[LaTeX]TeX,
basicstyle=\footnotesize\ttfamily,
classoffset=0,
  keywordstyle=\color{blue},
  morekeywords={obj, mor, lattice},
classoffset=1,
  keywordstyle=\color{red},
  morekeywords={;, :, \\},
classoffset=0,
commentstyle=\color{darkgray},
stringstyle=\color{red},
showstringspaces=true}



\def\fare{\todo[inline]{TODO!}}

\usepackage{hyphenat}

\usepackage{makeidx}
  \makeindex

\usepackage{float}

\usepackage{tikz}
  \usetikzlibrary{kD}
  \usetikzlibrary{calc}
  \usetikzlibrary{fit}
  \usetikzlibrary{positioning}
  \usetikzlibrary{arrows}

\usepackage[all,2cell,cmtip]{xy}\UseAllTwocells
\usepackage[type={CC},modifier={by-nc-sa},version={3.0}]{doclicense}

%%%%%%%%%%%%%%%%%%%%%%%%%%%%%%%%%%%%%%%% comandi personali

\newcommand{\cate}[1]{\mathbf{#1}}
	\providecommand{\CC}{\cate{C}}
	\providecommand{\D}{\cate{D}}
  \providecommand{\Set}{\cate{Set}}
  \providecommand{\Cat}{\cate{Cat}}
  \providecommand{\sSet}{\cate{sSet}}
  \providecommand{\QCat}{\cate{QCat}}
  \newcommand{\A}{\cate{A}}
  \providecommand{\ch}{\cate{Ch}}
  \newcommand{\hrt}{\cate{H}}
  \renewcommand{\sp}{\cate{Sp}}

\providecommand{\bDelta}{\Delta}
\def\ter{\boldsymbol{\tau}}

\newcommand{\mcal}[1]{\mathsf{#1}}
	\providecommand{\K}{\mcal{K}}
	\newcommand{\mS}{\mcal{S}}
	\newcommand{\TT}{\mcal{T}}
  \providecommand{\F}{\mcal{F}}
  \providecommand{\LL}{\mcal{L}}
  \providecommand{\R}{\mcal{R}}
	\providecommand{\EE}{\mcal{E}}
	\providecommand{\MM}{\mcal{M}}

  \providecommand{\fF}{\mathbb{F}}
  \providecommand{\Z}{\mathbb{Z}}

\newcommand{\smallcap}[1]{\text{\scshape #1}}
	\providecommand{\iso}{\smallcap{Eqv}}
	\providecommand{\fs}{\smallcap{fs}}
  \providecommand{\cart}{\smallcap{Cart}}
  \providecommand{\cocart}{\smallcap{Cocart}}
	\providecommand{\ts}{\smallcap{ts}}
  \providecommand{\pf}{\smallcap{pf}}

\def\dgrm#1{\mathsf{#1}}

\providecommand{\abbrv}[1]{#1.\@\xspace}
	\providecommand{\ie}{\abbrv{i.e}}
	\providecommand{\etc}{\abbrv{etc}}
	\providecommand{\prof}{\abbrv{prof}}
	\providecommand{\viz}{\abbrv{viz}}
	\providecommand{\eg}{\abbrv{e.g}}
  \providecommand{\achap}{\abbrv{Ch}}

\providecommand{\omissis}{[\dots\unkern] }
\providecommand{\ror}{Ror\-schach\@\xspace}

\providecommand{\rec}		{\text{\mxedb r}}
\providecommand{\marked}	{\varsigma}
\providecommand{\heart}		{\heartsuit}
\providecommand{\iddots}	{\protect{\rotatebox[origin=c]{90}{$\ddots$}}}
\providecommand{\sneet}		{\bigstar}
\providecommand{\defequal}	{\overset{\triangle}{=}}
\renewcommand{\setminus}	{\smallsetminus}

%###################################################

\providecommand{\lrlarrows}{%
	\mathrel{\substack{\leftarrow\\[-.9em] \rightarrow \\[-.9em] \leftarrow}}
		}

\newcommand{\docupdot}[2]{%
  \ooalign{$#1\cup$\cr\hfil$#1\cdot$\hfil}
  		}

\providecommand{\cupdot}{\mathbin{\mathpalette\docupdot\relax}}

\def\rwr{\raisebox{\depth}{\rotatebox[origin=c]{90}{$\wr$}}}
\def\tilt{\mathrel{\scalebox{.75}{\rotatebox[origin=c]{-90}{\sf 6}}}}

\providecommand{\tenuis}{\scalebox{.7}[.6]{\ooalign{\hss\textbar\hss\cr\hss$\equiv$\hss}}}
\providecommand{\glue}{\mathbin{\scalebox{.8}[.6]{\ooalign{\hss{\raisebox{4.45pt}{$\cup$}}\hss\cr\hss\tenuis\hss}}}}
%%%
\providecommand{\smalltenuis}{\scalebox{.7}[.6]{\ooalign{\hss\textbar\hss\cr\hss$\equiv$\hss}}}
\providecommand{\smallglue}{\mathbin{\scalebox{.8}[.6]{\ooalign{\hss{\raisebox{4pt}{$\cup$}}\hss\cr\hss\smalltenuis\hss}}}}

\providecommand{\var}[2]{ \left[ \begin{smallmatrix} %
      #1 \\ \downarrow \\ #2 %
      \end{smallmatrix} \right]}

\newcommand{\lowcons}{\ooalign{\hss\raisebox{-4pt}{$\cdot$}\hss\cr\hss$\curlywedge$\hss}}
\newcommand{\upcons}{\ooalign{\hss\raisebox{4pt}{$\cdot$}\hss\cr\hss$\curlyvee$\hss}}
\newcommand{\cons}{\ooalign{\hss$\curlyvee$\hss\cr\hss$\curlywedge$\hss}}%{\ooalign{\hss\raisebox{-4pt}{$\cdot$}\hss\cr\hss$\curlyvee$\hss}}

\providecommand{\tee}{\mathfrak{t}}
\renewcommand{\iff}{\Longleftrightarrow}
%###################################################

\newcommand{\id}	 {\text{id}}
\newcommand{\fib}	 {\text{fib}}
\newcommand{\cofib}{\text{cofib}}
\newcommand{\coker}{\text{coker}}

\newcommand{\mrk}	{\text{Mrk}}
\newcommand{\ho}	{\text{Ho}}
\newcommand{\ob}	{\text{Ob}}
\newcommand{\dg}  {\smallcap{dg}}
\newcommand{\Rex} {\text{Rex}}
%
\renewcommand{\hom} {\text{hom}}
\renewcommand{\max} {\text{max}}
\renewcommand{\min} {\text{min}}
\renewcommand{\sup} {\text{sup}}
\renewcommand{\inf} {\text{inf}}

\providecommand{\op}{\text{op}}
%
\newcommand{\im}{\text{im}}
\newcommand{\coim}{\text{coim}}
\def\length{\text{length}}

%%%%%%%%%%%%%%%%%%%%%%%%%%%%%%%%%%%%%%%% hyphenation

\tracingparagraphs=1
\tracingonline=1

\hyphenation{
  re-duce
  fol-low-ing
  con-cep-tu-al
  pref-er-en-tial
  re-in-ter-pret
  cat-e-go-ry
  ap-pli-ca-tion
  or-thog-o-nal
  pop-u-lar-ize
  ar-range-ment
  sec-tion
  writ-ing
  de-vel-op-ing
  cor-re-spon-dence
  the-o-ry
  con-di-tion
  nat-u-ral
  ho-mo-log-i-cal
  mod-el
  cor-re-spond
  dif-fi-cult
  dif-fer-ent
  def-i-ni-tion
  tor-sion
  rec-og-nize
  cer-tain-ly
  al-low
  al-ways
  ref-er-ence
  aes-thet-ic
  ap-pen-dix
  pro-duce
  en-vi-ron-ment
  where-as
  fac-tor-i-za-tion
  be-tween
  gen-er-al-i-ty
  dis-cov-er
  some-what
  pi-o-neer
  re-mark
  mod-ern
  to-geth-er
  de-vel-oped
  cat-e-go-ry
  min-i-mal
  con-fuse
  equiv-a-lent
  dis-cus-sion
  con-nec-tion
  ex-plic-it
  re-spec-tive-ly
  func-tor
  Ga-lois
  ex-change
  an-oth-er
  ter-mi-nal
  in-tro-duce
  con-tract
  cat-e-gor-i-cal
  ad-junc-tion
  ca-non-i-cal
  as-so-ci-ate
  char-ac-ter-ize
  de-fine
  par-tic-u-lar
  char-ac-ter-i-za-tion
  com-po-si-tion
  pre-sen-ta-tion
  con-se-quence
  de-ter-mine
  prob-lem
  equiv-a-lence
  ob-struc-tion
  or-thog-o-nal
  un-der
  re-tract
  com-pos-ite
  prop-er-ty
  sat-u-ra-tion
  com-plete
  sim-ple
  set-ting
  fac-tor-i-za-tion
  iso-mor-phic
  iso-mor-phism
  de-fine
  sim-pli-cial
  no-ta-tion
  state-ment
  es-pe-cial-ly
  mark-ing
  con-sist
  fac-tor
  as-sump-tion
  no-tion
  when-ev-er
  ap-pli-ca-tion
  sys-tem
  def-i-ni-tion
  prop-er-ty
  re-flec-tive
  lo-cal-ize
  sta-bil-i-ty
  con-sti-tute
  ad-di-tion-al
  con-fess
  co-her-ence
  asym-met-ri-cal
  chap-ter
  in-ter-est-ing
  bi-jec-tion
  sys-tem
  pull-out
  di-a-gram
  re-mark
  re-flec-tion
  ob-ject
  equiv-a-lent
  struc-ture
  con-trol
  in-for-ma-tion
  sug-ges-tion
  cal-cu-lus
  mo-ti-vate
  nor-mal
  sta-ble
  de-sire
  con-struc-tion
  cor-re-spon-dence
  out-line
  struc-ture
  de-note
  fac-tor
  tri-an-gu-late
  in-vari-ant
  ob-vi-ous
  sim-plic-i-ty
  al-ready
  show-ing
  su-per-flu-ous
  the-o-rem
  the-o-ry
  re-place
  sub-se-quent
  com-plex
  acy-clic
  con-clude
  trun-cate
  van-ish
  spec-trum
  arith-me-tic
  char-ac-ter-ize
  se-quence
  de-scrip-tion
  ex-cep-tion
  spec-u-la-tion
  en-rich
  ex-ten-sion
  sep-a-rate
  fam-i-ly
  con-cen-trate
  clas-si-cal
  en-dow
  re-fer
  sat-is-fy
  func-tion
  cus-tom-ary
  ap-prox-i-ma-tion
  max-i-mal
  bound-ed-ness
  in-deed
  ini-tial
  hor-i-zon-tal
  bound-ed
  nec-es-sary
  com-mu-ta-tive
  in-deed
  pull-out
  re-duce
  prop-o-si-tion
  ob-ject
  pre-vi-ous
  or-dered
  cap-ture
  de-com-pose
  as-sump-tion
  di-rec-tion
  char-ac-ter-ize
  gen-er-al
  de-com-pose
  Bern-stein
  con-struc-tion
  com-bi-na-tor-ics
  as-so-cia-tive
  or-ga-nize
  to-po-log-i-cal
  cat-e-gor-i-cal
  ig-nore
  var-i-a-tion
  sub-space
  func-tor
  mo-ti-va-tion
  ex-cep-tion-al
  ad-joint
  equiv-a-lence
  nev-er-the-less
  al-be-it
  de-ter-mined
  pro-cess
  con-sec-u-tive
  al-ge-bra
  mean-ing
  ob-tain
  ex-act
  for-mu-la-tion
  sev-er-al
  con-struct
  iso-mor-phism
  de-pen-dence
  sym-met-ri-cal
  trans-la-tion
  trans-port
  in-di-rect
  im-pli-ca-tion
  eas-i-ly
  op-er-a-tion
  com-pat-i-ble
  pres-ence
  strat-i-fi-ca-tion
  in-clu-sion
  com-par-i-son
  ar-bi-trary
  em-bed
  sat-is-fy
  in-ter-act
  ar-chi-tect
  in-ter-val
  dis-play
  spe-cial
  pri-mor-di-al
  as-so-cia-tive
  fam-i-ly
  re-main
  da-tum
  gen-er-al-ize
  com-pass
  quo-tient
  pres-ent
  how-ev-er
  short-hand
  op-er-a-tion
  ar-gu-ment
  con-sis-ten-cy
  de-fine
  pre-sent-able
  bi-lin-ear
  ex-is-tence
  in-duce
  met-ric
  un-for-tu-nate-ly
  ini-ti-ate
  in-crease
  in-dex
  di-rec-tion
  be-ing
  im-me-di-ate
  noth-ing
  fi-nite
  en-do-car-di-um
  Num-bers
  neigh-bor-hood
  pre-serve
  fac-tor
  quo-tient
  con-sid-er
  it-er-ate
  re-sult
  ar-gue
  en-tail
  in-duc-tion
  ar-gu-ment
  con-sid-er-ation
  de-sire
  white-head
  rec-og-nize
  cum-ber-some
  ho-mol-o-gy
  con-vey
  dis-tin-guished
  ad-di-tive
  tri-an-gle
  ar-row
  tri-an-gle
  de-ter-mine
  lo-cal-ize
  im-por-tant
  con-crete
  pri-or-i-ty
  be-have
  fun-da-men-tal
  uni-ver-sal
  pro-ce-dure
  treat-ment
  al-ge-bra
  al-ge-bra-ic
  suc-cess-ful
  un-der-stand-ing
  di-a-gram
  de-duce
  pleas-ant
  mu-tu-al
  im-age
  ax-i-om
  al-ge-bra
  ax-i-omat-i-za-tion
  re-la-tion
  po-ten-tial
  in-ter-val
  al-ter-na-tive
  com-mute
  ex-ist
  pull-back
  dif-fer-en-tial
  through-out
  uni-ver-sal-i-ty
  es-tab-lish
  de-gree
  mono-graph
  anal-y-sis
  Cam-bridge
  di-men-sion
  pub-lish-er
  to-pol-o-gy
  ori-en-tal
  lec-ture
  com-mu-ni-ca-tion
  com-po
  grad-u-ate
  with-out
  uni-ver-si-ty
  tri-an-gu-late
  the-o-ry
  %
  %
  o-mo-lo-gi-ca
  fon-da-men-ti
  ma-te-ma-ti-ca
  di-scu-ti-bi-le
  pro-prio
  af-fol-la-ta
  es-sen-zial-men-te
  pro-fes-sio-na-le
  re-la-to-re
  Spal-lan-za-ni
  per-mette
  ir-re-mo-vi-bi-le
  Watch-men
  per-so-ne
  men-to-re
  sbir-cia-re
  ter-mi-ne
  e-stre-ma-men-te
  gior-na-ta
  pub-bli-ci
  al-me-no
  bla-bla-bla
  qual-co-sa
  Al-ge-bri-sti-ca
  con-te-sti
  ric-chez-za
  met-ter-la
  qual-co-sa
  ra-gio-ne
  car-di-nal
  den-tro
  per-so-ne
  ab-bia
  car-di-nal
  es-sen-do
  con-trap-pun-to
  pro-ba-bi-li-tà
  mi-ni-ma
  in-ca-pa-ci-tà
  An-che
  con-sa-pe-vo-lez-za
  peg-gio-re
  con-di-zio-ne
  ac-cen-der-ti
  fan-ta-sie
  Pia-no
  pren-de-re
  pa-ro-le
  %
  %
  ho-mo-to-py
  neighbourhood
  morphisms
  right-or-tho-go-nal
  left-ob-ject-or-tho-go-na-li-ty
  pre-fac-to-ri-za-tions
  pre-fac-to-ri-za-tion
  sub-pos-et
  con-nec-ted
  co-li-mits
  Po-stni-kov
  sur-jec-tion-mo-no
  sub-ca-te-go-ries
  se-mi-left/right-ex-act-ness
  non-nor-ma-li-ty
  bi-re-flec-ti-ve
  se-mi-right-ex-act
  co-re-flec-ti-ve/-re-flec-ti-ve
  mor-phism
  re-a-so-ning
  se-mi-ex-act-ness
  qua-si-ca-te-go-ry
  de-fi-nes
  po-si-ti-ve/ne-ga-ti-ve
  re-flec-tion/co-re-flec-tion
  cor-res-pon-ding
  ar-row-or-tho-go-na-li-ty
  ob-ject-or-tho-go-nal
  co/re-flec-tion
  se-mi-or-tho-go-nal
  po-gro-up
  right-boun-ded
  fold
  sub-ca-te-go-ry
  high-er-ca-te-go-ri-cal
  wea-ved
  e-qui-va-riant
  co-lo-ca-li-za-tions
  Be-li-gian-nis
  re-col-le-ment
  thin-king
  ho-mo-to-pi-cal
  non-e-qui-va-riant
  re-col-le-ments
  so-cal-led
  tor-sion-free
  self
  con-tai-ned
  i-ni-ti-al/ter-mi-nal
  pa-ren-the-si-za-tion
  de-ri-ved
  Beck-Che-val-ley
  left-winged
  U-ri-zen
  sli-cing
  well
  i-ma-ge-fac-to-ri-za-tion
  co-re-flec-tion/re-flec-tion
  map-ping
  se-mi-sta-ble
  pre-sta-bi-li-ty
  e-qui-va-rian-cy
  Spa-nier
  Fro-be-nius
  a-xio-ma-tics
  Nee-man
  pull-back/push-out
  re-for-mu-la-tion
  co-re-flec-tion
  qua-si-sa-tu-ra-ted
  pre-trian-gu-la-ted
  Uni-versi-text
  an-wen-dun-gen
  Ox-ford-Lon-don-New
  Ei-len-berg-Moo-re
  adic
  Grenz-ge-bie-te
  Un-por-ted
cat-e-go-ry
ap-pli-ca-tion
or-thog-o-nal
pop-u-lar-ize
ar-range-ment
sec-tion
writ-ing
de-vel-op-ing
cor-re-spon-dence
the-o-ry
con-di-tion
nat-u-ral
ho-mo-log-i-cal
mod-el
cor-re-spond
dif-fi-cult
dif-fer-ent
def-i-ni-tion
tor-sion
rec-og-nize
cer-tain-ly
al-low
al-ways
ref-er-ence
aes-thet-ic
ap-pen-dix
pro-duce
en-vi-ron-ment
where-as
fac-tor-i-za-tion
be-tween
gen-er-al-i-ty
dis-cov-er
some-what
pi-o-neer
re-mark
mod-ern
to-geth-er
de-vel-oped
cat-e-go-ry
min-i-mal
con-fuse
equiv-a-lent
dis-cus-sion
con-nec-tion
ex-plic-it
re-spec-tive-ly
func-tor
Ga-lois
ex-change
an-oth-er
ter-mi-nal
in-tro-duce
con-tract
cat-e-gor-i-cal
ad-junc-tion
ca-non-i-cal
as-so-ci-ate
char-ac-ter-ize
de-fine
par-tic-u-lar
char-ac-ter-i-za-tion
com-po-si-tion
pre-sen-ta-tion
con-se-quence
de-ter-mine
prob-lem
equiv-a-lence
ob-struc-tion
or-thog-o-nal
un-der
re-tract
com-pos-ite
prop-er-ty
sat-u-ra-tion
com-plete
sim-ple
set-ting
fac-tor-i-za-tion
iso-mor-phic
iso-mor-phism
de-fine
sim-pli-cial
no-ta-tion
state-ment
es-pe-cial-ly
mark-ing
con-sist
fac-tor
as-sump-tion
no-tion
when-ev-er
ap-pli-ca-tion
sys-tem
def-i-ni-tion
prop-er-ty
re-flec-tive
lo-cal-ize
sta-bil-i-ty
con-sti-tute
ad-di-tion-al
con-fess
co-her-ence
asym-met-ri-cal
chap-ter
in-ter-est-ing
bi-jec-tion
sys-tem
pull-out
di-a-gram
re-mark
re-flec-tion
ob-ject
equiv-a-lent
struc-ture
con-trol
in-for-ma-tion
sug-ges-tion
cal-cu-lus
mo-ti-vate
nor-mal
sta-ble
de-sire
con-struc-tion
cor-re-spon-dence
out-line
struc-ture
de-note
fac-tor
tri-an-gu-late
in-vari-ant
ob-vi-ous
sim-plic-i-ty
al-ready
show-ing
su-per-flu-ous
the-o-rem
the-o-ry
re-place
sub-se-quent
com-plex
acy-clic
con-clude
trun-cate
van-ish
spec-trum
arith-me-tic
char-ac-ter-ize
se-quence
de-scrip-tion
ex-cep-tion
spec-u-la-tion
en-rich
ex-ten-sion
sep-a-rate
fam-i-ly
con-cen-trate
clas-si-cal
en-dow
re-fer
sat-is-fy
func-tion
cus-tom-ary
ap-prox-i-ma-tion
max-i-mal
bound-ed-ness
in-deed
ini-tial
hor-i-zon-tal
bound-ed
nec-es-sary
com-mu-ta-tive
in-deed
pull-out
re-duce
prop-o-si-tion
ob-ject
pre-vi-ous
or-dered
cap-ture
de-com-pose
as-sump-tion
di-rec-tion
char-ac-ter-ize
gen-er-al
de-com-pose
Bern-stein
con-struc-tion
com-bi-na-tor-ics
as-so-cia-tive
or-ga-nize
to-po-log-i-cal
cat-e-gor-i-cal
ig-nore
var-i-a-tion
sub-space
func-tor
mo-ti-va-tion
ex-cep-tion-al
ad-joint
equiv-a-lence
nev-er-the-less
al-be-it
de-ter-mined
pro-cess
con-sec-u-tive
al-ge-bra
mean-ing
ob-tain
ex-act
for-mu-la-tion
sev-er-al
con-struct
iso-mor-phism
de-pen-dence
sym-met-ri-cal
trans-la-tion
trans-port
in-di-rect
im-pli-ca-tion
eas-i-ly
op-er-a-tion
com-pat-i-ble
pres-ence
strat-i-fi-ca-tion
in-clu-sion
com-par-i-son
ar-bi-trary
em-bed
sat-is-fy
in-ter-act
ar-chi-tect
in-ter-val
dis-play
spe-cial
pri-mor-di-al
as-so-cia-tive
fam-i-ly
re-main
da-tum
gen-er-al-ize
com-pass
quo-tient
pres-ent
how-ev-er
short-hand
op-er-a-tion
ar-gu-ment
con-sis-ten-cy
de-fine
pre-sent-able
bi-lin-ear
ex-is-tence
in-duce
met-ric
un-for-tu-nate-ly
ini-ti-ate
in-crease
in-dex
di-rec-tion
be-ing
im-me-di-ate
noth-ing
fi-nite
en-do-car-di-um
Num-bers
neigh-bor-hood
pre-serve
fac-tor
quo-tient
con-sid-er
it-er-ate
re-sult
ar-gue
en-tail
in-duc-tion
ar-gu-ment
con-sid-er-ation
de-sire
white-head
rec-og-nize
cum-ber-some
ho-mol-o-gy
con-vey
dis-tin-guished
ad-di-tive
tri-an-gle
ar-row
tri-an-gle
de-ter-mine
lo-cal-ize
im-por-tant
con-crete
pri-or-i-ty
be-have
fun-da-men-tal
uni-ver-sal
pro-ce-dure
treat-ment
al-ge-bra
al-ge-bra-ic
suc-cess-ful
un-der-stand-ing
di-a-gram
de-duce
pleas-ant
mu-tu-al
im-age
ax-i-om
al-ge-bra
ax-i-omat-i-za-tion
re-la-tion
po-ten-tial
in-ter-val
al-ter-na-tive
com-mute
ex-ist
pull-back
dif-fer-en-tial
through-out
uni-ver-sal-i-ty
es-tab-lish
de-gree
mono-graph
anal-y-sis
Cam-bridge
di-men-sion
pub-lish-er
to-pol-o-gy
ori-en-tal
lec-ture
com-mu-ni-ca-tion
com-po
grad-u-ate
with-out
uni-ver-si-ty
tri-an-gu-late
the-o-ry
%
%
o-mo-lo-gi-ca
fon-da-men-ti
ma-te-ma-ti-ca
di-scu-ti-bi-le
pro-prio
af-fol-la-ta
es-sen-zial-men-te
pro-fes-sio-na-le
re-la-to-re
Spal-lan-za-ni
per-mette
ir-re-mo-vi-bi-le
Watch-men
per-so-ne
men-to-re
sbir-cia-re
ter-mi-ne
e-stre-ma-men-te
gior-na-ta
pub-bli-ci
al-me-no
bla-bla-bla
qual-co-sa
Al-ge-bri-sti-ca
con-te-sti
ric-chez-za
met-ter-la
qual-co-sa
ra-gio-ne
car-di-nal
den-tro
per-so-ne
ab-bia
car-di-nal
es-sen-do
con-trap-pun-to
pro-ba-bi-li-tà
mi-ni-ma
in-ca-pa-ci-tà
An-che
con-sa-pe-vo-lez-za
peg-gio-re
con-di-zio-ne
ac-cen-der-ti
fan-ta-sie
Pia-no
pren-de-re
pa-ro-le
%
%
ho-mo-to-py
neighbourhood
morphisms
right-or-tho-go-nal
left-ob-ject-or-tho-go-na-li-ty
pre-fac-to-ri-za-tions
pre-fac-to-ri-za-tion
sub-pos-et
con-nec-ted
co-li-mits
Po-stni-kov
sur-jec-tion-mo-no
sub-ca-te-go-ries
se-mi-left/right-ex-act-ness
non-nor-ma-li-ty
bi-re-flec-ti-ve
se-mi-right-ex-act
co-re-flec-ti-ve/-re-flec-ti-ve
mor-phism
re-a-so-ning
se-mi-ex-act-ness
qua-si-ca-te-go-ry
de-fi-nes
po-si-ti-ve/ne-ga-ti-ve
re-flec-tion/co-re-flec-tion
cor-res-pon-ding
ar-row-or-tho-go-na-li-ty
ob-ject-or-tho-go-nal
co/re-flec-tion
se-mi-or-tho-go-nal
po-gro-up
right-boun-ded
fold
sub-ca-te-go-ry
high-er-ca-te-go-ri-cal
wea-ved
e-qui-va-riant
co-lo-ca-li-za-tions
Be-li-gian-nis
re-col-le-ment
thin-king
ho-mo-to-pi-cal
non-e-qui-va-riant
re-col-le-ments
so-cal-led
tor-sion-free
self
con-tai-ned
i-ni-ti-al/ter-mi-nal
pa-ren-the-si-za-tion
de-ri-ved
Beck-Che-val-ley
left-winged
U-ri-zen
sli-cing
well
i-ma-ge-fac-to-ri-za-tion
co-re-flec-tion/re-flec-tion
map-ping
se-mi-sta-ble
pre-sta-bi-li-ty
e-qui-va-rian-cy
Spa-nier
Fro-be-nius
a-xio-ma-tics
Nee-man
pull-back/push-out
re-for-mu-la-tion
co-re-flec-tion
qua-si-sa-tu-ra-ted
pre-trian-gu-la-ted
Uni-versi-text
an-wen-dun-gen
Ox-ford-Lon-don-New
Ei-len-berg-Moo-re
adic
Grenz-ge-bie-te
Un-por-ted
}


%%%%%%%%%%%%%%%%%%%%%%%%%%%%%%%%%%%%%%%% shenanigans

\def\sator{\smallcap{s} & \smallcap{a} & \smallcap{t} & \smallcap{o} & \smallcap{r}}
\def\arepo{\smallcap{a} & \smallcap{r} & \smallcap{e} & \smallcap{p} & \smallcap{o}}
\def\tenet{\smallcap{t} & \smallcap{e} & \smallcap{n} & \smallcap{e} & \smallcap{t}}
\def\opera{\smallcap{o} & \smallcap{p} & \smallcap{e} & \smallcap{r} & \smallcap{a}}
\def\rotas{\smallcap{r} & \smallcap{o} & \smallcap{t} & \smallcap{a} & \smallcap{s}}

\renewcommand*{\thefootnote}{(\textbf{\arabic{footnote}})}

\usepackage{xspace}
	\providecommand{\abbrv}[1]{#1.\@\xspace}
	\providecommand{\ie}       {\abbrv{i.e}}
	\providecommand{\etc}      {\abbrv{etc}}
	\providecommand{\prof}     {\abbrv{prof}}
	\providecommand{\viz}      {\abbrv{viz}}
	\providecommand{\eg}       {\abbrv{e.g}}
  \providecommand{\adef}     {\abbrv{Def}}
  \providecommand{\acor}     {\abbrv{Cor}}
  \providecommand{\aprop}    {\abbrv{Prop}}
  \providecommand{\athm}     {\abbrv{Thm}}

\providecommand{\refbf}[1]{\textbf{\ref{#1}}}

\usepackage{graphicx}
\usepackage{setspace}
  % \linespread{1.1}
% \usepackage{fouriernc}

% SIMBOLI NON PERMESSI DAL FONT
%=========
  % \let\circledS=\undefined
  % \let\amalg=\undefined
  % \let\coprod=\undefined
  % \DeclareSymbolFont{cmsymbols}{OMS}{cmsy}{m}{n}
  % \DeclareSymbolFont{cmlargesymbols}{OMX}{cmex}{m}{n}
  % \DeclareMathSymbol{\amalg}{\mathbin}{cmsymbols}{"71}
  % \DeclareMathSymbol{\coprod}{\mathop}{cmlargesymbols}{"60}
%=========

\usepackage{fontspec}
% \usepackage{mathspec}

\setmainfont[SmallCapsFont={Latin Modern Roman Caps},]{Latin Modern Roman}
%\setmonofont[Scale = MatchLowercase]{Ubuntu Mono}

\newfontfamily{\japanese}{TakaoMincho}
\newfontfamily{\bigjapanese}{TakaoExMincho}
\newfontfamily{\hebrew}{Shlomo LB}
\newfontfamily{\russian}{CMU Serif}
\newfontfamily{\greek}{Latin Modern Roman}
\newfontfamily{\georgian}{AcadNusx}
  \def\rec{\text{\georgian{r}}}
\newfontfamily{\fraktur}{Plakat-Fraktur}
% \newfontfamily{\mycal}{}[Scale=MatchUppercase]{Lucida Calligraphy}

\usepackage{
   amsmath
  ,amsthm
  ,amssymb
  ,stmaryrd}

\def\semicol{\fatsemi}
\newtheoremstyle{reference}%
   {\topsep}
   {\topsep}
   {}
   {}
   {\scshape}
   {}
   {0em}
   {\thmname{#1}
    \thmnumber{#2}.
    \thmnote{({\scshape #3}): }}

\theoremstyle{reference}
  \newtheorem{theorem}{Theorem}[section]
  \newtheorem{lemma}[theorem]{Lemma}
  \newtheorem{proposition}[theorem]{Proposition}
  \newtheorem{example}[theorem]{Example}
  \newtheorem{exercise}[theorem]{Exercise}
  \newtheorem{remark}[theorem]{Remark}
  \newtheorem{definition}[theorem]{Definition}
  \newtheorem{corollary}[theorem]{Corollary}
  \newtheorem{notat}[theorem]{Notation}
  \newtheorem*{acknowledgements}{Acknowledgements}
  \newtheorem{scholium}[theorem]{Scholium}
  \newtheorem{df-prop}[theorem]{Definition/Proposition}
  % starred versions
  \newtheorem*{theorem*}{Theorem}
  \newtheorem*{lemma*}{Lemma}
  \newtheorem*{proposition*}{Proposition}
  \newtheorem*{example*}{Example}
  \newtheorem*{exercise*}{Exercise}
  \newtheorem*{remark*}{Remark}
  \newtheorem*{definition*}{Definition}
  \newtheorem*{corollary*}{Corollary}
  \newtheorem*{notat*}{Notation}
  \newtheorem*{scholium*}{Scholium}

  \newenvironment{thededication}
  {\clearpage
   \thispagestyle{empty}
   \vspace*{\stretch{1}}
   \raggedleft}
  {\par
   \vspace{\stretch{3}}
   \clearpage}
%
\providecommand{\attrib}[1]{%
	\nopagebreak{\raggedleft\footnotesize\slshape #1\par}
		}

\usepackage{epigraph}
\setcounter{chapter}{-1}
\setcounter{secnumdepth}{3}

\def\DefaultEpigraphWidth{.6\textwidth}

\usepackage{enumitem}
\setlist[itemize]{noitemsep, topsep=2pt}
\setlist[enumerate]{label=(\oldstylenums{\arabic*}), noitemsep, topsep=2pt}

\usepackage{fancyhdr}
\pagestyle{fancy}
\rhead[\thepage]{\smallcap{\nouppercase{\rightmark}}}
\lhead[\smallcap{\nouppercase{\rightmark}}]{\thepage}
\chead{}
\lfoot{}
\rfoot{}
\cfoot{}
%


% stile del comando \cite
\makeatletter
  \def\@cite#1#2{[\textsf{#1}\if@tempswa , #2\fi]}
  \def\@biblabel#1{[\textsf{#1}]}
\makeatother

\providecommand{\refbf}[1]{\textbf{\ref{#1}}}
\providecommand{\xto}[1]{\xrightarrow{#1}}
\providecommand{\xot}[1]{\xleftarrow{#1}}

%##############################################################

\providecommand{\po}{\ar@{}[dr]|(.7){\text{po}}}
\providecommand{\pb}{\ar@{}[dr]|(.3){\text{pb}}}
\providecommand{\pp}{\ar@{}[dr]|{\text{pp}}}


\newcommand{\pullback}[2]{\obj at=($(#1)!.2!(#2)$):{\lrcorner}}
\newcommand{\pushout}[2]{\obj  at=($(#1)!.8!(#2)$):{\ulcorner}}
\newcommand{\pullout}[2]{\pullback{#1}{#2}; \pushout{#1}{#2}}

\usepackage{hyperref}
 \hypersetup{%
   unicode=		   true,
   pdftoolbar=	 true,
   pdfmenubar=	 true,
   pdffitwindow= true,
   pdfstartview= {FitH},
   pdftitle=	   {t-structures},
   pdfauthor=	   {Fosco Loregian},
   pdfcreator =  {},
   pdfproducer = {},
   colorlinks=	 true,
   linkcolor=	   black,
   urlcolor=	   blue!70,
   citecolor=	   blue!70}

% \pdfvariable suppressoptionalinfo \numexpr 1+2+4+8+16+32+64+128+256+512 \relax


\providecommand{\cover}[1]{\mathfrak{#1}}
  \newcommand{\B}{\cover{B}}

\usepackage[bottom=6cm,
            left=5cm,
            right=4.5cm,
            top=4.5cm]{geometry}

\usepackage[greek, english]{babel}
\usepackage{bidi}

% This prevents the )(1.12 phenomenon:
\makeatletter
\def\maketag@@@#1{\hbox{\m@th\normalfont\LRE{#1}}}
\def\tagform@#1{\maketag@@@{(\ignorespaces#1\unskip)}}
\makeatother

\newcommand{\arXivPreprint}[1]{arXiv preprint \href{http://arxiv.org/abs/#1}{arXiv:#1}}

\usepackage{pifont}
\def\uno{\text{\ding{172}}}
\def\due{\text{\ding{173}}}
\def\tre{\text{\ding{174}}}
\def\qua{\text{\ding{175}}}
\def\cin{\text{\ding{176}}}
\def\sei{\text{\ding{177}}}
\def\set{\text{\ding{178}}}
\def\ott{\text{\ding{179}}}
\def\nov{\text{\ding{180}}}
\def\zer{\text{\ding{181}}}


\renewcommand{\|}{|\!|}


\newcommand{\HebrewEpigraph}[2]{%
\setlength{\epigraphwidth}{.75\textwidth}
\renewcommand{\textflush}{flushright}
\epigraph{\setRL{\hebrew{#1}}\unsetRL}
{#2}
\renewcommand{\textflush}{flushleft}
\setlength{\epigraphwidth}{\DefaultEpigraphWidth}
}

\def\[{\begin{equation}}
\def\]{\end{equation}}