\chapter*{A less serious introduction}
\thispagestyle{empty}
\subsection*{To the reader}
\epigraph{Le anime, al contrario delle lame, si affilano evitando ogni contatto.}{Elia Spallanzani}
Ciò che stai per leggere è il prodotto di un lavoro di indagine che esula enormemente dalla matematica; parlare di algebra omologica, di teoria delle categorie, di geometria, topologia o fondamenti è funzionale a uno scopo diverso dalla ``semplice'' matematica. Come conseguenza, questo lavoro contiene diverse cose in aggiunta ad essa: la mia visione della materia, che ho raffinato (o peggiorato, o irrigidito) negli anni; dosi molto elevate di un discutibile, troppo personale senso estetico; un ancor più discutibile gusto per il citazionismo e diverse idee che, cresciute in libertà nell'arco di anni, non sono state smussate, semmai affilate, proprio perché intoccate. Non ultimo, un certo rifiuto per le scene, la predilezione per la calma esatta che precede una tempesta al suono metallico che fa il denaro, o al caos di una conferenza affollata solo per dare ai suoi partecipanti un pretesto per una vacanza in montagna o al mare.

Se c'è del bello e del valido in queste pagine, il merito è essenzialmente tutto di Domenico: la conquista impagabile di questi anni è sapere di essere riuscito a guadagnarmi non già la parità professionale, ma anche l'amicizia di un  individuo di questa caratura, che merita un ringraziamento a parte non solo in qualità di relatore.

Domenico mi ha insegnato diverse cose; solo alcune, le meno importanti, riguardano come si fa il \emph{Matematico}. Se in queste pagine c'è del brutto, la colpa è mia, che ancora non ho capito bene come si mette la maiuscola a questa ambiziosa parola.

Questa tesi deve molto, nel bene e nel male, a I\@. Calvino, J.L\@. Borges, G\@. Perec, R\@. Queneau (e altri patafisici), D\@. Aury, E\@. Spallanzani e altri autori reali o immaginari. Essa è ispirata alle loro opere nello stesso modo, e nello stesso senso, in cui è ispirata ad altre, che parlano di matematica in un senso più palese e tecnico.

Ovviamente ne risulta solo una brutta copia; un prodotto strano, un poco più lungo e di forma sgraziata, che non le permette di essere a suo agio in nessuno dei due regni del sapere. Ciò che la salva probabilmente è lo spirito con cui è stata scritta: di questa sezione mi sono impossessato per esporlo al meglio delle mie capacità. Nessuna ambizione ad essere diverso o speciale, qui; solo la consapevolezza che far sembrare questo lavoro ``come tutti gli altri'' avrebbe tradito la parte di me che, fin da quando ha voce per esprimerla, insegue e propaganda l'assoluta, irremovibile unitarietà di tutti i saperi umani. L'idea folle di questi anni (assecondare la quale equivale a macchiarsi di una colpa più grave dell'idealismo) per cui vi è un solo soggetto; seguire questa idea implica che citare nello stesso luogo W\@. Blake, la Genesi, Watchmen, A\@. Crowley e i testi di scherma e mistica orientale è null'altro che un modo di alludere a questa unità, a questa ingombrante inseparabilità, a questa molteplice connessione. Suggerire questa idea è quel che mi interessava raggiungere quando ho iniziato; il resto, \emph{tutto} il resto, è funzionale a tale obiettivo.

Tra le cose buone di questi anni c'è di più, e di più prezioso di una ``tesi di dottorato'': ho avuto accesso al cuore di alcune persone, che probabilmente avrebbero fatto le stesse cose nello stesso modo, ma che con me le hanno fatte (spero) con maggiore gioia. Ho regalato amore, e se a volte ho chiesto qualcosa in cambio, è stato perché avevo messo sul piatto la moneta più preziosa di cui disponevo: quel che ho fatto, e le ragioni che mi hanno mosso nel farlo erano un atto d'amore verso la bellezza che vedevo. Per questo, titolo di dottore o meno, ho già vinto qualcosa: poco, e non mi basta ancora, ma qualcosa che altri non possono dire di avere, sono qualcosa che altri (quelli che ``non rivendicano originalità di pensiero'') non possono dire di essere. E voglio continuare, anche se questa incalcolabile ricchezza è talmente sottile da passare invisibile attraverso il setaccio delle valutazioni istituzionali (\emph{quanto} hai scritto, mai \emph{il perché}; quanto hai raccolto, mai quanto hai seminato) perché ho una responsabilità di mentore e compagno di viaggio, verso persone che probabilmente non conosco ancora, ma a cui non per questo tengo di meno.

Ciò che è più prezioso, allora, non sono i teoremi che sono, siamo riusciti a dimostrare, e le definizioni che sono riuscito a sbirciare nel Libro del Sommo Fascista mentre questo era distratto. La cosa più preziosa di questo testo non c'è scritta altrove perché l'avrei potuta comunicare solo attraverso una complicata parola di una lingua dell'emisfero boreale di Tl\"on, o con un complesso termine \emph{ithkuil}: sta nelle persone che mi hanno accompagnato in questi anni, in quelle che hanno deciso di smettere di farlo, e in quelle che ho mandato via io. 

Alcuni tra i teoremi che stai per leggere hanno il tepore estremamente specifico e gradevole che l'automobile lasciata al sole fa quando verso la fine di marzo smette di fare freddo, altri sono intrisi della puzza di fumo che ha la tua giacca quando passi la giornata sugli scadenti ma irrimediabilmente romantici mezzi pubblici di una città che ha solo due linee di metro ma ne sogna almeno quattordici; alcuni sanno di sangue o di lacrime, perché un pugno ti ha rotto le labbra, e per qualche settimana non riesci a parlare, e altri esistono solo perché ad un certo punto qualcuno mi ha aperto la porta di casa sua quando ero fragile e avevo il cuore spezzato; alcuni sono spariti, non li trovo più perché sono un grafomane molto disordinato, ma mi ricordo bene com'erano fatti. Devo solo riscriverli come si deve e sperare che non li trovi prima $Y$, ``un poco arrugginiti per la pioggia del mercoledì''. Altri sperabilmente verranno nel futuro, prossimo o lontano: in fin dei conti, io sono solo uno scriba al servizio di una donna bellissima, che parla poco anche a chi se lo merita più di me. Tutti, nessuno escluso, sono un talismano che porto con orgoglio, per lenire la mia anima dall'assurdo inconsolabile di cui sono intrise le cose mondane.

Come sempre, stilare un elenco esaustivo di \emph{tutte} le persone che bla\-bla\-bla è un obiettivo perso in partenza, e dunque bla\-bla\-bla. Dato che però di qualcuno mi ricordo, nomino almeno loro. L'ordine è solo quello che mi suggerisce la mia memoria ubriaca (non c'è altra condizione ammissible che un leggero stordimento, adatta a scrivere questa parte di tesi, e questo stordimento si può ottenere solo col sonno o con l'alcool).
\begin{enumerate}
\item {\bf Jow}, detto Wanny the dog, detto Giuditta, detto Fonso, detto Drugo, detto Sergente, detto \dots Se ho imparato qualcosa fuori della matematica, probabilmente il merito è suo. C'è una linea molto sottile che unisce ``la Settimana Algebristica'' a ``I smoke for my glaucoma'': Fonso è l'unico che riesce a percorrerla dritta, rimanendo in entrambi i contesti uno degli amici migliori che potrei sperare di avere. \emph{Fortissimo}.
\item {\bf Lamù}, a cui va tutto l'amore che riesco a conchiudere in un bacio, che sa allo stesso tempo di granita al limone e delle lacrime di uno che si commuove fin troppo facilmente. Spero col passare del tempo mangeremo gelati sempre più grossi, pranzando davanti a tramonti sempre nuovi, guardando però sempre lo stesso mare.
\item {\bf John von Neumann}, che capirà un giorno che tipo di ricchezza si porta dentro; quello stesso giorno capirà come metterla a frutto, e spero di essere nei paraggi per imparare qualcosa di meraviglioso. Il mio scopo, nel farmelo amico, è difendere la sua dedizione, il candore con cui ama, in un senso ancora perfettamente vergine, quella donna bellissima che sussurra nel buio.
\item {\bf Delizia e Delirio}, che probabilmente non ha idea della ragione per cui la sto ringraziando, ed è giusto che sia così.
\item {\bf Saezuri}, che si chiederà anche lei perché la sto ringraziando: la ragione è in un parco giochi alle pendici dei colli Euganei.
\item {\bf Morgana}, perché è la seconda persona che, se Grothendieck l'avesse conosciuta, le avrebbe detto ``ton c{\oe}ur est un cardinal inaccessible''.
\item {\bf Xena}, perché una delle cose migliori di questi anni è averla vista crescere mentre io cercavo il bambino (quello che, dentro di me, sa stare solo), e aver capito come far combaciare questi movimenti in direzioni opposte.
\item {\bf la cerchia di \textsc{f}\&\textsc{h}}, che mi piace pensare comunicherà mediante bestemmie e offese alle rispettive madri anche quando saremo adulti e dovremo invitarci, probabilmente, a fare una vacanza in montagna. Aver riunito queste persone nello stesso luogo è stato un contributo infinitesimo alla crescita di individui già grandi, che sono tuttavia fiero di avere dato.
\item {\bf il nipote di Quillen}, che forse si illudeva di aver trovato un mentore e che è stato, al contrario, più di una volta fonte di esempio. Io ho solo innaffiato una pianta già forte, alla cui ombra, un giorno, sarò felice di studiare sorseggiando un the.
\item {\bf Michele G.}, cui rivolgo i sentimenti migliori che un uomo possa provare: un rispetto fraterno, una enorme invidia per come sa essere solido nei propri principi, aperto all'altro e onesto come pochissimi altri individui che ho mai potuto conoscere.
\item {\bf 5nuff}, probabilmente l'individuo più sfaccettato che io abbia mai conosciuto, e probabilmente l'unico che ha potuto dirmi una certa frase in un certo modo in un certo posto con un certo effetto.
\item {\bf Jean Rostand}, con cui mi sono divertito a incazzarmi per finta, così come ad incazzarmi davvero, e a volte non si è nemmeno capita molto la differenza. È giusto così.
\item {\bf Sofia Bond}, perché è la terza persona che, se Grothendieck l'avesse conosciuta, le avrebbe detto ``ton c{\oe}ur est un cardinal inaccessible''.
\item {\bf Laura M.}, cui non potrò che far leggere questa tesi nell'aldilà, ma che mi fa da esempio anche da lì: se ho qualche speranza di essere diventato un dissidente colto, il merito è stato tutto suo. E soprattutto, è stata lei a farmi conoscere Borges (essendo, credo, solo parzialmente conscia delle conseguenze catastrofiche del gesto).
\item {\bf Salvo}, che sa dire la cosa giusta ad un'anima in pena quando questa ne ha bisogno; se questa tesi è scritta in questo modo, e non in un altro, è (de)merito anche suo. È stato un arbitro di eleganza, un esempio di stile, un suggeritore di mantra che aiutassero in momenti difficili, o un ottimo contrappunto per condividere i momenti felici. Semplicemente, mi è stato amico anche quando era difficile.
\item {\bf Sergio B.}, perché mi ha aiutato a realizzare che la tua probabilità di rimanere in quel gioco di scimmie ammaestrate che è la vita accademica è inversamente proporzionale a quanto sei strano, e mi ha quindi fatto capire che non avevo la minima speranza di carriera.
\item {\bf Rahil}, perché ne invidio la determinazione a rialzarsi, il gusto per l'arte sacra, e perché mi diverte l'assoluta incapacità di comprendere certi suoi messaggi $\heartsuit$.
\item {\bf la mia vicina di casa cisoceanica}, a cui non ho paura di rivelare una antica ma irrimediabile, incurabile cotta. Anche lontani un mare, spero che i tuoi momenti bui possano finire: sappi che ti voglio bene senza remora o dubbio.
\item {\bf ¿}, che non è ancora il momento di ringraziare, ma è bene portarsi avanti con il lavoro.
\item {\bf Palo Barsolny}, a cui troppo presto, o per meglio dire giusto in tempo, ho messo \cite{McL} in mano; sono contento di aver corso il rischio di distruggere Tutto con quel gesto. Sono contento di averlo perso e ritrovato. Giusto in tempo, o per meglio dire al tempo giusto. Non ultimo, è lui che ha creato {\tt koDi}.
\item {\bf Nash}, il cui candore fa perdonare qualsiasi stranezza. ``La più grande pena che quest'uomo deve vivere è la consapevolezza quotidiana che il mondo reale è un luogo molto peggiore di un film di Miyazaki.''
\item Alcuni luoghi mi hanno dato casa: mi risulta impossibile non ringraziarli.
\begin{itemize}
\item {\bf P}elagia, affinché riacquisti il senno che ha perduto.%adova
\item {\bf T}eodora, la prima ad accogliermi e nello stesso tempo a farmi sentire straniero. %rieste
\item {\bf C}osima, che ad un passo da Ko\hyp{}de\hyp{}mondo ha rapito il mio cuore. %avriago
\item {\bf B}erenice, che accoglie tutti quelli che trova, a condizione che non dia fastidio non essere i primi ad esser passati per quel letto. %ologna
\item {\bf P}atrizia, che è la donna troppo piena di sé per non accenderti di desiderio. %isa
\item {\bf R}ebecca, la donna altera che se vuoi stare con me va bene, ma decido io quando ci vediamo. %oma
\item {\bf T}eresa, nelle cui stanze tutte uguali è facile perdersi. %orino
\item e tante altre, mete di una notte oppure oggetto di fantasie su quanto costi trasferirvisi.
\end{itemize}
\end{enumerate}
E poi la Gattara, che non ho ancora capito dove sia finita; la Gnà, che è la prima persona a causa della quale questa tesi non parla di pannelli solari al silicio; Nancy, che nonostante qualsiasi calma interiore io continuerò a chiamare così; il musicologo che mi ha insegnato la grafologia mentre tornavo da Parigi; quattro ore di meditazione in un parcheggio su cosa fare della mia vita; l'odore di piscio in certe strade, cui finisci per affezionarti sapendo di non poterle cambiare; una certa vegana dal bel faccino; il capodanno nella casa di montagna del Drugo; il pensiero di realizzare \emph{Il Piano}${}^\smallcap{tm}$, che ancora mi tiene sveglio la notte a immaginare; tutti i libri che avrei voluto scrivere io; la metafora dei telescopi; il \oldstylenums{19} che non arriva mai e che non ho mai avuto il coraggio di prendere fino al capolinea; l'Aleph e lo Zahir (non so decidere di chi ho più paura); il centro culturale {\sf Multiplo}; gli sbirri e il reato che mi hanno perdonato (forse); il punto più basso che ha raggiunto il mio morale quando (per un attimo) ho pensato che avrei potuto chiuderla a chiave in casa.
Dipanare questo intricato gomitolo è stata la fatica di questi anni, che ancora dura e forse non finirà mai. Grazie di avermi dato così tanto da fare.

Riuscire a scrivere questi ringraziamenti, con queste esatte parole, è stato a volte l'unico motivo che mi ha fatto vincere l'orrore verso questo documento, o verso le persone, cose o situazioni che hanno tentato di distruggerlo. Perché effettivamente ci sono, come è naturale, anche una serie di persone cui auguro di bruciare all'inferno più in fretta e con maggior dolore di quanto già non li costringa a fare la loro quotidiana miseria. 

Ringrazio anche loro, e forse specialmente loro: hanno tracciato un margine molto nitido che mi ha fatto capire cosa non sono, cosa non riuscirei ad essere o a diventare nemmeno provando a portare una cravatta.
\vspace*{\fill}

% \setlength{\epigraphwidth}{\textwidth}
% \epigraph{Only you get to consciously decide what has meaning and what doesn't. \emph{That} is real freedom. \emph{That} is being educated and understanding how to think. The alternative is unconsciousness; the default setting; the rat-race; the constant, gnawing sense of having had, and lost, some infinite thing.}{David Foster Wallace}
% \setlength{\epigraphwidth}{\DefaultEpigraphWidth}
\begin{flushright}
Un punto imprecisato di $\mathbb{S}^2$,\\
\today
\end{flushright}