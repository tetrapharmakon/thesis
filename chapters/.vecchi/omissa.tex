\section{Omissa da \tstruct}

% \hrulefill

% \begin{remark}
% A general construction in homotopical algebra ensures that the Joyal model structure on $\cate{sSet}$ (see \cite{Hov} and \cite{Joy}) lifts to a model structure on $\cate{sSet}^\marked$, having as fibrant objects precisely the objects sent to quasicategories by the forgetful functor $U$ above. \todo[inline]{reference needed!}
% \end{remark}

% \hrulefill

% \begin{proposition}
% There exists an adjunction  \[\underline{{}^\boxslash(-)}\dashv \underline{(-)^\boxslash}\colon \cate{QCat}/X^{\Delta[1]}\leftrightarrows\left(\cate{QCat}
% /X^{\Delta[1]}\right)^\op 
% \]
% ``lifting'' the Galois connection ${}^\boxslash(-)\dashv (-)^\boxslash\colon \mrk(X)\leftrightarrows \mrk(X)$.\end{proposition}
% This can be seen as a quasicategorical version of \cite[Prop. \textbf{3.8}]{Gar}.
% \begin{proof}

% \end{proof}

% \hrulefill

% A \emph{weak} factorization system is defined to be a pair of markings $\mathcal E, \mathcal M\in\mrk(\CC)$ satisfying condition (1) before and the weakened condition ($2'$): $\EE ={}^{\boxslash}\MM$ and $\MM = \EE^{\boxslash}$. Weak factorizations arise naturally in contexts where the uniqueness of the solution to the lifting problem is too a strong requirement; a natural factory of such situations in ordinary category theory is of course homotopical algebra.

% \hrulefill


% A similar alternative definition is possible for weak factorization systems, allowing to replace condition ($2'$) with
% \begin{itemize}
% \item[($2'$a)] $\mathcal E\boxslash \mathcal M$;
% \item[($2'$b)] $\EE$ and $\MM$ are closed under isomorphisms in $\CC^{\Delta[1]}$.
% \end{itemize}

% \hrulefill

% \begin{remark}
% It is a rather trivial (but useful) remark that \emph{any} prefactorization $(\EE,\MM)\in \textsc{pf}(\CC)$ is the prefactorization right/left generated by a class $\mathcal K$; \todo[inline]{e per\`o non so se sia sempre possibile generarne uno con un insieme piccolo}.
% \end{remark} 
% \begin{example}\label{esem:modelquasi}
% A \emph{Quillen model structure} on a small-bicomplete quasicategory $\CC$ is defined by three markings $(\mathcal W,\mathcal F,\mathcal C)$ such that
% \begin{itemize}
% \item $\mathcal W$ is a 3-for-2 class containing all isomorphisms and closed under retracts;
% \item The markings $(\mathcal C,\mathcal W\cap \mathcal F)$ and $(\mathcal W\cap \mathcal C,\mathcal F)$ both form a weak factorization system on $\CC$.
% \end{itemize} 
% \end{example}
% \todo[inline]{Ma allora cosa vuol dire semanticamente che io posso definire una \emph{struttura modello} in una qcat?}

% \hrulefill

% \begin{proposition}\label{prop:equivcond}
% Let $(\EE, \MM) \in \textsc{wfs}(\CC)$, such that all cokernelpairs of morphisms in $\CC$ exist in $\EE$. Then the following conditions are equivalent:
% \begin{enumerate}
% \item[(i)]  $(\EE, \MM) \in  \textsc{fs}(\CC)$;
% \item[(ii)] $\EE$ satisfies either condition (i) or condition (ii) in Prop. \refbf{prop:clos};
% \item[(iii)] for every $e:A \to B$ in $\EE$ the canonical morphism $e'\colon B\coprod_A B \to B$ lies also in $\EE$ (where $B\coprod_A B$ is the codomain of the cokernelpair of $e$);
% \item[(iv)] if $gf =1$ with $f \in \EE$, then $g \in \EE$.
% \end{enumerate}
% \end{proposition}
% \begin{proof}

% \end{proof}
% \subsection{A second glance to \textsc{qfs}.} We add for the sake of completeness a different presentation of quasicategorical factorization systems, faithfully following \cite[pp. 178---]{Joy}.
% \begin{definition}[Orthogonality and Fillers]
% Let $\CC$ be a quasicategory, and $u\colon A\to B, f\colon X\to Y$ two edges of $\CC$. We define the space $\text{Sq}(u,f)$ of commutative squares associated to $(u,f)$ to be the space of simplicial maps $s\colon \Delta[1]\times\Delta[1]\to \CC$ such that $s|_{\Delta^0\times \Delta[1]}=u, s|_{\Delta[1]\times\Delta^0}=f$.

% A \emph{diagonal filler} for $s\in\text{Sq}(u,f)$ consists of an extension $\bar s\colon \Delta[1]\star\Delta[1]\to \CC$ (where $\star$ denotes the \emph{join} of simplicial sets, see \cite[\S\textbf{3.1} and \textbf{3.2}]{Joy}) of $s$ along the natural inclusion $\Delta[1]\times\Delta[1]\subset\Delta[1]\star \Delta[1]$.\end{definition}
% \begin{remark}
% Denote by $\text{Fill}(s)$ the top-left corner of the fiber sequence
% \[
% \xymatrix{
% \text{Fill}(s) \ar[r]\ar[d] & X^{\Delta[1]\star\Delta[1]} \ar[d]^q\\
% \Delta^0 \ar[r]_s & X^{\Delta[1]\times\Delta[1]}
% }
% \]
% The simplicial set $\text{Fill}(s)$ is a Kan complex, since $q$ is a Kan fibration (as a consequence of \cite[Prop. \textbf{2.18}]{Joy}).
% \end{remark}
% This leads us to the following
% \begin{definition}
% We say that the edge $u$ is \emph{left orthogonal} to the edge $f$ in the quasicategory $\CC$ (or $f$ is \emph{right orthogonal} to $u$) if $\text{Fill}(s)$ is a \emph{contractible} Kan complex for any $s\in \text{Sq}(u,f)$. We denote this relation between $u$ and $f$ as $u\perp f$.
% \end{definition}
% \begin{remark}
% Let $|\CC|_\cate{Cat}$ the fundamental category of $\CC$, and $h\colon \CC\to N|\CC|$ the canonical map obtained from the unit of the adjunction $|\cdot|_\cate{Cat}\dashv N$\footnote{The adjunction is described in \cite{Joy} as well as in \cite{HTT}; $|\cdot|_\cate{Cat}\colon \cate{sSet}\to\cate{Cat}$ is the Yoneda left (Kan) extension of the functor $i$ which regards the standard simplex $[n]=\{0<1<\dots<n\}$ as a category. The functor $|\cdot|_\cate{Cat}$ sends a simplicial set $X$ to the coend $\int^n X_n \cdot i[n]$.}. Then the relation $u\perp f$ implies that $h(u)\perp h(f)$ in the classical, 1-categorical sense, but the converse is in general not true.
% \end{remark}
% \begin{remark}
% If $(\EE, \MM)$ is only a \emph{weak} factorization system, then the subcategory $\F(\MM)$ is \emph{weakly} $\EE$-reflective in $\CC$: existence of non-unique liftings entail that we can only hope in a weak reflection, where the unit of the adjunction is only \emph{weakly} universal. 

% Nonetheless, the following discussion concentrates only on the theory of \emph{orthogonal} factorization systems, for which (see \cite[p. \textbf{179}]{Joy}) the (right) left class of a \textsc{fs} is always a (co)reflective sub-quasicategory.
% \end{remark}
% \begin{proposition}
% Factorization systems can be lifted along fibrations. More precisely, if $p\colon \CC\to \D$ is a left or right fibration between quasicategories, and $(\EE,\MM)$ is a factorization system on $\D$, then $(p^\leftarrow\EE, p^\leftarrow\MM)$ is a factorization system on $\CC$.
% \end{proposition}
% \begin{corollary}
% As a consequence, since the simplicial maps $\CC_{/X}\to \CC$ and $\CC_{Y/}\to\CC$ are left/right fibrations, every factorization system on $\CC$ \emph{lifts} to a factorization system on the slice/coslice quasicategory. This sheds a quasicategorical light on \cite{Joy2}.
% \end{corollary}
% \begin{example}[Monics and surjections]\label{esem:monic}
% We say that an arrow $X\to Y$ in a quasicategory is \emph{monic} if the square besides is cartesian (as a limit in the quasicategory, obviously). The class of monic arrows in $\CC$ is collected in a (often large) marking $\textsc{Mono}(\CC)=\textsc{Mono}$.
% \marginpar{omissis}%$\xymatrix{X \ar@{=}[r]\ar@{=}[d]& X \ar[d]^f \\ X \ar[r]_f & Y}$}

% The class of \emph{surjective} arrows is defined to be the class ${}^\perp\textsc{Mono}(\CC)$; we say that the quasicategory $\CC$ \emph{has a surjection-mono factorization} if $({}^\perp\textsc{Mono}(\CC), \textsc{Mono}(\CC))\in \textsc{fs}(\CC)$.
% \end{example}
% \begin{definition}[Regular quasicategory]
% A finitely complete quasicategory is said to be \emph{regular} if it admits a \emph{pullback-stable} surjection-mono factorization system; the coherent nerve of the category of Kan complexes, as a full sub-quasicategory of the nerve of the whole $\cate{sSet}$, is regular. Notice that this is the quasicategorical counterpart of \emph{Barr-regular categories}.
% \end{definition}
% Hence \emph{regularity} is a property which can be captured in the language of mere factorization systems; can other classes of (quasi)categories be defined as categories endowed with suitable \textsc{fs}?
% \begin{example}[Postnikov towers via factorization]
% We now describe an intrinsic procedure to obtain \emph{Postnikov decompositions} of objects in a quasicategory; we follow\dots 
% \begin{definition}[0-factorization]
% A simplicial set is called a \emph{0-object} if (regarding the latter as a constant simplicial set) the canonical map $X\to \pi_0X$ is a weak homotopy equivalence. An object $A$ of a quasicategory $\CC$ is called a 0-object if the hom-functor $\CC(-,A)$ sends any $X\in\CC$ to a 0-object $\CC(X,A)\in\cate{sSet}$. An object $A$ is a 0-object if and only if the diagonal $A\to A\times A$ is monic in the sense of Definition \refbf{monic}, if and only if the map $A^{S^1}\to A$ is invertible in the fundamental category of $\CC$. 

% An arrow $A\to B$ is called a 0-\emph{cover} if it is a 0-object in the quasicategory $\CC_{/B}$; we collect all the 0-covers of $\CC$ in a (often large) marking 0-\textsc{cov}, and we define 0-\textsc{conn}$={}^\perp\text{0-}\textsc{cov}$, the class of 0-connected arrows. 
% \end{definition}
% A quasicategory $\CC$ \emph{admits 0-factorizations} if $(\text{0-}\textsc{conn}, \text{0-}\textsc{cov})\in \textsc{fs}(\CC)$.

% There is a notion of 0-connected/0-cover and of $n$-conn/$n$-cov; all these form nested factorization systems; the Post tower of an arrow $A\to B$ consists of a factorization
% \[
% omissis
% % \xymatrix@C=2mm@R=4mm{
% % A && \ar[ll]_{q_0}\ar@{.>}[dl] X_0 && \ar[ll]_{p_1} X_1 &&\ar[ll]_{p_2}  \dots && \ar[ll]_{p_n}  X_n && \ar[ll]_{q_n} B \\
% % &X_{-1}\ar@{.>}[ul]
% % }
% \]
% where $q_0$ is a 0-cover, each $p_i$ is an $(i-1)$-connected $i$-cover and $q_n$ is $n$-connected.
% \end{example}


\hrulefill


\[
omissis
% \xymatrix@C=1.5cm@R=1mm{
% \mathcal A \ar@{|->}[r]^{\text{enrichment}/\text{Ch}^+(\cate{Ab})}& \mathcal A^\sim \ar@{|->}[r]^{\text{homwise Dold-Kan}}& \mathcal A^\sim_\Delta \ar@{|->}[r]^N & \cate{D}_\infty (\mathcal A) \stackrel{\text{Ho}}{\longmapsto} \cate{D} (\mathcal A)\\
% \cate{AbCat} \ar@{.>}@/^1pc/[r]&\ar@{.>}@/^1pc/[l] \text{Ch}^+(\cate{Ab})\text{-}\cate{Cat} \ar@{.>}@/^1pc/[r]&\ar@{.>}@/^1pc/[l] \textbf{sGrp}\text{-}\cate{Cat} \ar@{.>}@/^1pc/[r]&\ar@{.>}@/^1pc/[l] \infty\text{-}\cate{Cat}_\text{st}\\
% & 
% }
\]

\hrulefill

We start considering the category $\Delta[1]\times\Delta[1] =  (\Lambda_2^2)^\lhd = (\Lambda_0^2)^\rhd$ (see the diagrams besides; \marginpar{omissis} 
% \footnotesize $\xymatrix@R=5mm@C=5mm{
% (0,0) \ar[r]\ar[d]\ar@{}[dr]|{\Lambda_0^2}& (1,0)\\
% (0,1) & \\
% \ar@{}[dr]|{\normalsize \Lambda_2^2}& (1,0)\ar[d]\\
% (0,1) \ar[r]& (1,1)
% }$} 
each of these descriptions will turn out to be useful) and denote it as $\square$ for short. In the same way, we denote pictorially the two horn-inclusions
\begin{gather*}
i_\ulcorner\colon \ulcorner\to \square \quad \big(= \Lambda_0^2\to (\Lambda_0^2)^\rhd\big)\\
i_\lrcorner\colon \lrcorner\to \square \quad \big(= \Lambda_2^2\to (\Lambda_2^2)^\lhd\big)
\end{gather*}
(see \cite[Notation \textbf{1.2.8.4}]{HTT}) and the induced maps
\begin{gather}
i_\ulcorner^*\colon \text{Map}(\square, \CC) \to \text{Map}(\ulcorner,\CC)\\
i_\lrcorner^*\colon \text{Map}(\square, \CC) \to \text{Map}(\lrcorner,\CC)
\end{gather}
from the category of commutative squares in $\CC$, ``restricting'' a given diagram to its top or bottom part, respectively. These functors are part of a string of adjoints
\begin{gather}
(i_\ulcorner)_! \dashv \boxed{i_\ulcorner^* \dashv (i_\ulcorner)_*}\colon \text{Map}(\square,\CC)\leftrightarrows \text{Map}(\ulcorner,\CC)\\
\boxed{(i_\lrcorner)_! \dashv i_\lrcorner^*} \dashv (i_\lrcorner)_*\colon \text{Map}(\square,\CC)\leftrightarrows \text{Map}(\lrcorner,\CC)
\end{gather}
where $(i_\ulcorner)_!$ and $(i_\lrcorner)_*$ are easily seen to be evaluations at the initial and terminal object of $\ulcorner$ and $\lrcorner$ respectively. 

It's rather easy to see that, given $F\in \text{Map}(\square, \CC)$ the canonical morphisms obtained from the boxed adjunctions,
\begin{gather*}
\eta_{\ulcorner, F}\colon F\to (i_\ulcorner)_*i_\ulcorner^* F\\
\epsilon_{\lrcorner, F}\colon (i_\lrcorner)_!  i_\lrcorner^*F\to F
\end{gather*}
give the canonical ``comparison'' arrow $F(1,1) \to \varinjlim i_\ulcorner^* F$ and $\varprojlim i_\lrcorner^*F\to F(0,0)$.

With these notations we can give the following definitions.

\hrulefill

\begin{proof}
Let $\varnothing\to 1$ be the canonical arrow between the initial and terminal object of $\CC$, which exist since $\CC$ has finite limits.

We can build the pushout and pullback squares besides 
\marginpar{omissis}
% $\xymatrix@R=6mm@C=6mm@u{
% \varnothing \ar[r]\ar[d] & 1\ar[d]  & F\ar[r]\ar[d] & \varnothing \ar[d]\\
% 1 \ar[r] & C & \varnothing \ar[r] & 1
% }$}
and glue them together to get 
\[omissis
% \xymatrix@R=4mm@C=4mm{
% F \ar[r]\ar[d] & \varnothing \ar[r]\ar[d]& 1\ar[d] \\
% \varnothing \ar[r] & 1\ar[r]  & C
% }
\]
It's easy to see that $F\cong \varnothing$ and $C\cong 1\amalg 1$ (and this is true independently from the pullout axiom, simply by definition); so now square \ding{172} besides must be a pullout, and $F\cong\varnothing\to\varnothing$ can only be the identity arrow, so that the arrow $1\to C$ is an isomorphism too (since the class of isomorphisms is closed under cobase change). This entails that there is an isomorphism $1\to\varnothing\cong 1\amalg 1$, which allows to conclude: the square
\[
omissis
% \xymatrix@R=4mm@C=4mm{
% \varnothing \ar[r]\ar[d] & 1\ar@{=}[d]\\
% 1 \ar[r]& 1
% }
\]
is a pullout and isomorphisms are closed under base change too.

\marginpar{omissis}
% $\xymatrix@R=6mm@C=6mm{
% F\ar[r]\ar[d]\ar@{}[dr]|{\text{\ding{172}}} & \varnothing \ar[d]\\
% 1 \ar[r]& C
% }$}
We now use this fact to show that products and coproducts coincide everywhere: build the outer square in the following diagram, out of the smaller square (which are precisely the pullbacks/pushouts needed to define products and coproducts):
\[
\xymatrix{%@R=4mm@C=4mm{
Y\ar[r]\ar[d] & X\times Y\ar[r]\ar[d] & Y \ar[d] && X\ar[r]\ar[d] & X\times Y\ar[r]\ar[d] & X \ar[d]\\
0\ar[r]\ar[d] & X\ar[r]\ar[d] & 0\ar[d] && 0\ar[r]\ar[d] & Y\ar[r]\ar[d] & 0\ar[d]\\
Y\ar[r] & X\amalg Y\ar[r] & Y && X\ar[r] & X\amalg Y\ar[r] & X
}
\]
These two diagrams imply the presence of biproducts, in the form of a result shown by Freyd (see \cite{freyd1964abelian}): there exists an object $S$ (the \emph{biproduct} of $X,Y$, hence unique up to unique isomorphism) such that
\begin{itemize}
\item There are arrows $Y\leftrightarrows S\leftrightarrows X$;
\item The arrow $Y\to S\to Y$ compose to the identity of $Y$, and the arrow $X\to S\to X$ compose to the identity of $X$;
\item There are ``exact sequences'' (in the sense of a pointed, finitely bicomplete category) $0\to Y\to S\to X\to 0$ and $0\to X\to S\to Y\to 0$.
\end{itemize}
It is evident that the diagrams above contain all these informations.
\end{proof}
A pleasant consequence of Freyd characterization is that in any additive category the enrichment over the category of abelian groups is \emph{canonical}; in fact, exploiting the isomorphism $Y\times Y\cong Y\amalg Y$ one is able to define the \emph{sum} of $f,g\colon X\rightrightarrows Y$ as
\[
omissis
%f+g\colon X\xto{\bsmat[[] 1\\ 1\esmat[]]} X\times X \xto{(f,g)} Y\times Y\cong Y\amalg Y\xto{\bsmat[[] 1 & 1\esmat[]]} Y
\]
In fact, this result can be retrieved in the setting of stable quasicategories (see \cite[Lemma \textbf{1.1.2.9}]{LurieHA}); we do not want to reproduce the whole argument: instead we want to investigate the construction of the \emph{loop} and \emph{suspension} functors in a pointed category.

\hrulefill

\begin{definition}
Let $\CC$ be a finitely cocomplete, pointed quasicategory; denote by $\CC^\square_\text{cocart}$ the full subcategory of $\text{Map}(\square,\CC)$ spanned by the cocartesian squares of the form
\[
\xymatrix{%@R=5mm@C=5mm{
X \ar[r]\ar[d]& 0\ar[d] \\
0' \ar[r]& Y
}
\]
where $0,0'$ are (possibly different, but equivalent) zero objects of $\CC$; dually, we can define $\CC^\square_\text{cart}$ the full subcategory of $\text{Map}(\square,\CC)$ spanned by the cartesian squares of the form
\[
\xymatrix{%@R=5mm@C=5mm{
X \ar[r]\ar[d]& 0\ar[d] \\
0' \ar[r]& Y
}
\]
where $0,0'$ are (possibly different, but equivalent) zero objects of $\CC$. 
\end{definition}
\begin{proposition}[\protect{\cite[Prop. \textbf{5.3}]{Gro}}]
The canonical evaluations
\[
\text{e}_{(0,0)}\colon \CC^\square_\text{cocart}\to \CC\qquad 
\text{e}_{(1,1)}\colon \CC^\square_\text{cart}\to \CC
\]
(where $(i,j)$ denotes the vertex $(i,j)\in \square_0$) are acyclic Kan fibrations.
\end{proposition}
From this it follows that we can choose sections (unique up to homotopy) $s_\Sigma$ and $s_\Omega$ for $\text{e}_{(0,0)}$ and $\text{e}_{(1,1)}$ respectively. This leads to the following
\begin{definition}[Loop and suspension functors]
Let $\CC$ be a finitely bicomplete, pointed quasicategory; we define the \emph{suspension} of an object $X\in\CC$ as the composition
\[
\Sigma\colon \CC\xto{s_\Sigma}\CC^\square_\text{cocart}\xto{\text{e}_{(1,1)}} \CC
\]
and (dually) the \emph{looping} of $X$ as the composition
\[
\Omega\colon \CC\xto{s_\Omega}\CC^\square_\text{cart}\xto{\text{e}_{(0,0)}}\CC
\]
More explicitly, $\Sigma X$ and $\Omega X$ are uniquely determined from the homotopy cocartesian and homotopy cartesian squares
\end{definition}
\hrulefill

\begin{definition}[Triangulated category]
An additive category $\CC$ is called a \emph{category with suspension} if it is endowed with an exact autoequivalence $\Sigma\colon \CC\to\CC$; a category with suspension $(\CC,\Sigma)$ is said to be \emph{triangulated} if the following axioms are satisfied:
\begin{itemize}
\item[\textsc{pt0})] There exists a class of diagrams in $\CC$, called \emph{distinguished triangles} of the form $X\to Y\to Z\to \Sigma X$ which is closed under isomorphism and contains every sequence of the form $X\xto{\id_X}X\to 0\to \Sigma X$;
\item[\textsc{pt1})] Any arrow $f\colon\Delta[1]\to\CC$ fits into a distinguished triangle $X\xto{f}Y\to Z\to \Sigma X$;
\item[\textsc{pt2})] (rotation) The diagram $X\xto{u}Y\xto{v} {Z}\xto{w} \Sigma X$ is distinguished if and only if the ``rotated diagram'' $Y\xto{-v}Z\xto{-w}\Sigma X\xto{-\Sigma u}\Sigma Y$ is distinguished;
\item[\textsc{pt3})] (completion) In any diagram of the form
\[
\xymatrix{
X \ar[r]\ar[d]_f & Y \ar[r]\ar[d]_g & Z\ar[r] & \Sigma X\ar[d]^{\Sigma f} \\
X'\ar[r] & Y'\ar[r] & Z'\ar[r] & \Sigma X' 
}
\]
where the rows are distinguished triangles, there exists a morphism $h\colon Z\to Z'$ making the whole diagram a morphism of triangles (which, once regarded triangles as suitable functors $J\to \CC$ are simply natural transformations between two such functors).
\item[\textsc{tr})] Given \emph{three} distinguished triangles
\begin{gather*}
X\xto{f}Y\to Y/X\to X[1] \\
Y\xto{g}Z\to Z/Y\to Y[1]\\
X\xto{gf}Z\to Z/X\to X[1]
\end{gather*}
arranged in a \emph{braid} diagram
\[
omissis
% \xymatrix{%@!=3mm{
% X\ar@/^1.5pc/[rr]^{gf}\ar[dr]_f &&Z\ar[dr]_p\ar@/^1.5pc/[rr] && Z/Y\ar[dr] \ar@/^1.5pc/[rr] &\circlearrowleft & Y/X[1] \\
% &Y\ar[ur]_g\ar[dr]&&Z/X\ar@{.>}[ur]^t\ar[dr]_q&&Y[1]\ar[ur] \\
% &&Y/X\ar@{.>}[ur]^s\ar@/_2pc/[rr] &&X[1]\ar[ur] \\
% }
\]
then there is a (non-unique) way to complete it with the arrows $s,t$ indicated.
\end{itemize}
\end{definition}
With the exception of Axiom \textsc{tr}, which is somehow characteristic, and by no means the less natural among triangulated category axioms, one can easily see that all of them are easy consequences of universal properties of (homotopy) limits: distinguished triangles are precisely those diagrams of the form
\[
\xymatrix{
X \ar[r]\ar[d]\pp & Y \ar[r]\ar[d]\pp & 0 \ar[d] \\
0 \ar[r]& Z\ar[r] & \Sigma X
}
\]
where both squares are pullout. Every arrow $f\colon\Delta[1]\to \CC$ fits into a distinguished triangle, since every arrow admits a cofiber $Y\to C$ given by the homotopy colimit of the diagram $0\leftarrow X\to{f}Y$; this can be completed on its own right taking the cofiber of the arrow $Y\to C$; the 2-for-3 property applied to the diagram
\[
\xymatrix{
X}
\]
implies now that $W\simeq \Sigma X$; the completion axiom is an immediate consequence of the universal property of homotopy pushouts.

Hence we concentrate on a detailed proof of Axiom \textsc{tr}: in the classical, 1-categorical theory, it can be justified in at least two ways:
\begin{itemize}
\item The \emph{freshman algebraist's theorem} holds in triangulated categories: $\frac{Z/X}{Y/X}\cong Z/Y$
\item Given the braid diagram above, not only the triangle ending with $Z/Y\to (Y/X)[1]$ can be completed, but the choice can be made in a \emph{coeherent} manner.
\end{itemize}
In fact once translated the braid diagram in a diagram in a stable quasicategory $\CC$ we are in the following situation:
\[
omissis
% \xymatrix{%R=6mm{
% X\ar@[red][r]^f \ar[d]\ar@[green]@/^1.5pc/[rr]^{gf} & Y \ar@[blue][r]^g \ar@[red][d] & Z\ar[r]\ar@[green][d]\ar@/_1pc/@[blue][dd] & 0\ar[d] \\
% 0 \ar[r] & Y/X  \ar[d]\ar@[red]@/_1pc/[rr] & Z/X \ar@[green][r] & X[1] \ar[r]\ar@[red][d] & 0\ar[d] \\
% & 0 \ar[r] & Z/Y \ar@[blue][r] & Y[1] \ar@[red][r] & (Y/X)[1]
% }
\]
where different colours denote different fiber sequences (\ie, triangles in the homotopy category). Axiom \textsc{tr} says that we can find arrows $Y/X\to Z/X\to Z/Y$  such that the triangle $Y/X\to Z/X\to Z/Y\to (Y/X)[1]$ is distinguished. 

The completion axiom (which we already know to hold) now implies that the diagram
\[
\xymatrix{%R=6mm{
X \ar[r]^f\ar[d]_f& Y\ar[r]\ar[d]^g & Y/X \ar[r] & X[1] \ar[d]^{f[1]}\\
Y \ar[r]_g & Z\ar[r] & Z/X\ar[r] & Y[1]
}
\]
can be completed with an arrow $Y/X \xto{\phi} Z/X$ completing the square
\[
\xymatrix{%R=6mm{
Y \ar[r]\ar[d]& Z\ar[d] \\
Y/X \ar[r]_\phi & Z/X.
}
\]
Now consider the ojects $V={\rm hocolim}\Big( Y/X \xot{} Y \xto{g} Z \Big)$ and $W={\rm hocolim}\Big( 0\xot{} Y/X \xto{\phi}Z/X \Big)$; 
\[
\xymatrix{%C=4mm@R=4mm{
Y \pp \ar[rr]\ar[dd]&& Z\ar[dl]\ar[dd] \\
&V\ar[dr]& \\
Y/X\ar[dd] \ar[rr]_\phi\ar[ur]&& Z/X\ar[dd]\\
& \\
0 \ar[rr]&& W
}
\]
2-out-of-3 now implies that the outer rectangle is a pushout, hence $W\cong Z/Y$. It remains to prove that $V\cong Z/X$; this follows from the 2-out-of-3 property applied to  the two joined pullout diagrams
\[
\xymatrix{
X\pp \ar[r]\ar[d] & Y\pp \ar[r]\ar[d] & Z\ar[d] \\
0 \ar[r] & Y/X \ar[r] & V.
}
\]
\section{Omissa da {\tt recol}} 
\begin{theorem}\label{ttfarerecol}
There exists a bijective correspondence between \textsc{ttf} triples and recollements on a stable $\D$, given by the following two correspondences:
\begin{itemize}
\item If $(\cate X, \cate Y, \cate Z)$ is a stable \textsc{ttf} triple, and we adopt the notation of (\refbf{ttf}), then the diagram
\makeatother \[\label{rcl}
\xymatrix{%C=1.4cm{
  \cate Y	& \D	& \cate X
  \ar "1,1";"1,2" |{i_{\cate Y}}
  \ar@<8pt> "1,2";"1,1" ^{\tau_{\ge}^{\cate Y}}
  \ar@<-8pt> "1,2";"1,1" _{\tau_{<}^{\cate Y}}
  \ar "1,2";"1,3" |{\tau_{\ge}^{\cate X}}
  \ar@<8pt> "1,3";"1,2" ^{i_{\cate X}}
  \ar@<-8pt> "1,3";"1,2" _{i_{\cate Z} \circ \tau_{<}^{\cate Z} \circ  i_{\cate X}}
}
\]
\makeatletter 
is a recollement on $\D$.
\item If $(i,q)\colon \D^0 \lrlarrows  \D \lrlarrows  \D^1$ is a recollement on $\D$, then the triple $(q_L(\D^1), i(\D^0), q_R(\D^1))$ is a \textsc{ttf}-triple on $\D$.
\end{itemize}
\end{theorem}
Following and slightly generalizing \cite[Thm. \textbf{2.5}]{parshall1988derived} we give the following
\begin{definition}[The ($\infty$-)category $\cate{Recol}$]
A morphism between two recollements $(i,q) $ and $(i', q')$ consists of a triple of functors $(F_0, F_{01}, F_1)$ such that the following square commutes in every part (\ie for every choice of horizontal arrows):
\makeatother
\[
\xymatrix{
  \D^0	& \D	& \D^1 \\
  '{\D^0}	& '{\D}	& '{\D^1} \\
  \ar|i "1,1";"1,2" 
  \ar "1,1";"2,1" _{F_0}
  \ar "1,2";"2,2" _{F_{01}}
  \ar|q "1,2";"1,3" 
  \ar@<-6pt> "1,2";"1,1" 
  \ar@<6pt> "1,2";"1,1" 
  \ar "1,3";"2,3" ^{F_1}
  \ar@<-6pt> "1,3";"1,2" 
  \ar@<6pt> "1,3";"1,2" 
  \ar "2,1";"2,2" |{i'}
  \ar "2,2";"2,3" |{q'}
  \ar@<-6pt> "2,2";"2,1" 
  \ar@<6pt> "2,2";"2,1" 
  \ar@<-6pt> "2,3";"2,2" 
  \ar@<6pt> "2,3";"2,2" 
}
\]
\makeatletter
This definition turns the collection of all recollement data into a $\infty$-category denoted $\cate{Recol}$ and called the ($\infty$-)category of recollements.
\end{definition}
\begin{remark}
The natural definition of equivalence between two recollement data (all three functors $(F_0, F_{01}, F_1)$ are equivalences) has an equivalent reformulation (see \cite[\abbrv{Thm} \textbf{2.5}]{parshall1988derived}) asking that only two out of three functors \todo[inline]{qualcosa}; nevertheless this must not be interpreted as a full 3-for-2 condition, as (\emph{loc. cit.}) warns the reader that\todo[inline]{qualcos'altro}.

We can also find a genuinely weaker notion of equivalence between recollement data, asking only (see \cite[\S \textbf{1.7}]{hugel2011recollements}) that the essential images of the fully faithful functors in the first recollement $(i,q)$ are equivalent via functors $F_i\colon i(\D_0)\cong i'('{\D^0}), F_R\colon q_R(\D^1) \cong q_R'('{\D^1}), F_L \colon q_L(\D^1) \cong q_L'('{\D^1})$.
\end{remark}
\begin{proposition}
The correspondence of Thm. \refbf{ttfarerecol} is the object part of an isomorphism of categories between the category $\cate{Recol}$ of recollements and the category $\cate{TTF}$ of \textsc{ttf} triples.
\end{proposition}
\begin{proof}
.
\todo[inline]{Da scrivere}
\end{proof}
\hrulefill

\todo[inline]{Va corretto il pointer a questa osservazione nella introduzione}
\begin{remark}[A ``Hypotheses non fingo'']
The \ror's lemma is a fundamental step towards our characterization of $\fF_0\glue \fF_1$: coming to the gist of the present section, we have to spend a word about this result.

Despite the simplicity of its proof, a clear explanation of the \emph{reasons} behind the validity of the \ror lemma seems rather elusive; even the genesis of the result in question is far from being guided by some sort of precognition. After several attempts to prove the equality (\refbf{BBDleftclass}) both authors conjectured that a proof of the equality $\EE_0\glue \EE_1=\{f\in\hom(\D)\mid \{q, i_R\}(f)\in \EE\}$ would have been easier; clarified by the argument used in this latter proof, and once they understood how to modify the old argument, they attacked the initial conjecture with a renewed artillery, and were left with what is now called the ``Sneetches' theorem''.
\end{remark}

\makeatother
\[ 
\label{diag1bis}
\scalebox{.85}{
\xymatrix{%R=2mm@C=3mm{
  	& 0	& 	& i_L SY	& 	& i_L WY	& 	& i_L Y \\
  0	& 	& i_L SX	& 	& i_L WX	& 	& i_L X \\
  	& 	& 	& 	& 	& 	& 	& i_L KY \\
  0	& 	& S_0 i_L KX	& 	& i_L CX	& 	& i_L KX \\
  	& 	& 	& 	& 	& 	& 	& i_L RY \\
  	& 	& 0	& 	& R_0 i_L WX	& 	& i_L RX \\
  	& 	& 	& 	& 	& 	& 	& \# \\
  	& 	& 	& 	& 0	& 	& \#
  \ar "1,2";"1,4" 
  \ar "1,4";"1,6" 
  \ar "1,6";"1,8" 
  \ar_\wr "1,8";"3,8" 
  \ar "2,1";"2,3" 
  \ar@{=} "2,1";"4,1" 
  \ar "2,3";"2,5" 
  \ar_\wr "2,3";"4,3" 
  \ar "2,3";"1,4" 
  \ar "2,5";"2,7" 
  \ar_\wr "2,5";"4,5" 
  \ar "2,5";"1,6" 
  \ar@{} "2,7";"3,8" |{\text{\large \ding{192}}}
  \ar_\wr "2,7";"4,7" 
  \ar "2,7";"1,8" 
  \ar@{->>} "3,8";"5,8" 
  \ar "4,1";"4,3" 
  \ar "4,3";"4,5" 
  \ar "4,3";"6,3" 
  \ar "4,5";"4,7" 
  \ar@{->>} "4,5";"6,5" 
  \ar@{} "4,7";"5,8" |{\text{\large \ding{193}}}
  \ar@{->>} "4,7";"6,7" 
  \ar "4,7";"3,8" 
  \ar "5,8";"7,8" 
  \ar "6,3";"6,5" 
  \ar "6,5";"6,7" 
  \ar "6,5";"8,5" 
  \ar "6,7";"5,8" 
  \ar "6,7";"8,7" 
  \ar "8,5";"8,7" 
}}
\]
\makeatletter

% \begin{remark}
% For the ease of the reader we group the recollements present in $\dgrm{G}_n$ for low values of $n$:
% \begin{itemize}
% \item[$n=2)$]
% \makeatother  \begin{gather}
% 	\scalebox{.8}{
%     \xymatrix@!R=4mm@!C=4mm{
%       	& 	& \textcolor{semilightgray}{\D_{012}} \\
%       	& \D_{01}	& 	& \textcolor{semilightgray}{\D_{12}} \\
%       \D_0	& 	& \D_1	& 	& \textcolor{semilightgray}{\D_2} \\
      
%       \ar@{.>}@[semilightgray] "1,3";"2,4" 
%       \ar@{.>}@[semilightgray] "2,2";"1,3" 
%       \ar "2,2";"3,3" 
%       \ar@{.>}@[semilightgray] "2,4";"3,5" 
%       \ar "3,1";"2,2" 
%       \ar@{.>}@[semilightgray] "3,3";"2,4" 
%     }
% 	\quad
%     \xymatrix@!R=4mm@!C=4mm{
%       	& 	& \textcolor{semilightgray}{\D_{012}} \\
%       	& \textcolor{semilightgray}{\D_{01}}	& 	& \D_{12} \\
%       \textcolor{semilightgray}{\D_0}	& 	& \D_1	& 	& \D_2 \\
      
%       \ar@{.>}@[semilightgray] "1,3";"2,4" 
%       \ar@{.>}@[semilightgray] "2,2";"1,3" 
%       \ar@{.>}@[semilightgray] "2,2";"3,3" 
%       \ar "2,4";"3,5" 
%       \ar@{.>}@[semilightgray] "3,1";"2,2" 
%       \ar "3,3";"2,4" 
%     }}\notag\\
% 	%
% 	\scalebox{.8}{ 
% 	\xymatrix@!R=4mm@!C=4mm{
% 	  	& 	& \D_{012} \\
% 	  	& \D_{01}	& 	& \textcolor{semilightgray}{\D_{12}} \\
% 	  \textcolor{semilightgray}{\D_0}	& 	& \textcolor{semilightgray}{\D_1}	& 	& \D_2 \\
	  
% 	  \ar@{-} "1,3";"2,4" 
% 	  \ar "2,2";"1,3" 
% 	  \ar@{.>}@[semilightgray] "2,2";"3,3" 
% 	  \ar "2,4";"3,5" 
% 	  \ar@{.>}@[semilightgray] "3,1";"2,2" 
% 	  \ar@{.>}@[semilightgray] "3,3";"2,4" 
% 	}
% 	\quad 
% 	\xymatrix@!R=4mm@!C=4mm{
% 	  	& 	& \D_{012} \\
% 	  	& \textcolor{semilightgray}{\D_{01}}	& 	& \D_{12} \\
% 	  \D_0	& 	& \textcolor{semilightgray}{\D_1}	& 	& \textcolor{semilightgray}{\D_2} \\
	  
% 	  \ar "1,3";"2,4" 
% 	  \ar "2,2";"1,3" 
% 	  \ar@{.>}@[semilightgray] "2,2";"3,3" 
% 	  \ar@{.>}@[semilightgray] "2,4";"3,5" 
% 	  \ar@{-} "3,1";"2,2" 
% 	  \ar@{.>}@[semilightgray] "3,3";"2,4" 
% 	}}
% \end{gather}\makeatletter
% \item[$n=3)$] The pattern should be clear now, so we simply list the various recollement found in $\dgrm{G}_2$:
% \begin{itemize}
% \item Each $\D_i\lrlarrows \D_{\llbracket i,i+1\rrbracket}\lrlarrows \D_{i+1}$
% \item
% \end{itemize}
% \end{itemize}
% \end{remark}

% \hrulefill

\begin{theorem}
Let $J$ be a $\mathbb{Z}$-poset, and 
$$(i,q)\colon \D^0 \lrlarrows  \D \lrlarrows  \D^1$$
a recollement; let $\tee_i\colon J\to  \ts (\D_i)$ be $J$-ary semiorthogonal decompositions on $\D^i$. Then the $J$-family $\tee_0\glue\tee_1$ defined above is again a semiorthogonal decomposition on $\D$
\end{theorem}
\begin{proof}
According to \cite[\S\textbf{4.1}]{heart} semiorthogonal decompositions on a stable $\infty$-category $\CC$ correspond to $J$-equivariant maps $J\to  \ts (\CC)$ such that $J$ carries the trivial $\mathbb{Z}$-action, and \cite[\abbrv{Rem} \textbf{4.6}]{heart} (together with the whole discussion in \S\textbf{4}) entails that the corresponding \textsc{ntt} on the category $\CC$ has stable classes on its own right, namely both $(\EE,\MM)$ are closed under pullbacks and pushouts in $\CC$.

Hence we are left with the proof of the following statement: in the same hypotheses and notation as above,
\begin{quote}
if $\EE_0,\EE_1$ are left parts of two exact normal torsion theories $\fF_0,\fF_1$ on $\D_0, \D_1$, then the gluing $\EE_0\glue \EE_1$ is the left part of the exaxct normal torsion theory $\fF_0\glue\fF_1$ on $\D$.
\end{quote}
This, in fact, is rather easy to prove: suppose $f\in\hom(\D)$ is such that $\{q, i_R\}(f)\in \EE$. Then in any pullback square
\makeatother
\[
	\xymatrix{
	  	&  \\
	  	& 
	  \ar^p "1,1";"1,2" 
	  \ar "1,1";"2,1" _{f'}
	  \ar^f "1,2";"2,2" 
	  \ar_h "2,1";"2,2" 
	}
\]
\makeatletter
we can apply the functors $\{q, i_R\}$ and obtain again a pullback square (\abbrv{Prop} \refbf{recoexact}): this entails that, since $q(f)\in \EE_1$, $i_R(f)\in \EE_0$ and both these classes are closed under pullback, also $\{q, i_R\}(f')\in \EE$.
\end{proof}

\newpage

\begin{itemize}
\item Se $\CC$ è una $\infty$-cat qualsiasi, quali FS su $\CC$ diventano NTT su $\text{Sp}(\CC)$?
\item La questione delle torsion pairs: diverse richieste alla coppia di subcat si traducono in diverse proprietà del FS (deco semiortogonale, chiusa per shift, sciccherie varie). Capire meglio questo fatto.
\item Accumulare più esempi
\item 
\item 
\item 
\item 
\item 
\end{itemize}