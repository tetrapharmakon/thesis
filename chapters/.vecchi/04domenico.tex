
\index{Gluing!--- of $t$-structures}
\subsection{The classical gluing of $t$-structures.}\label{classrec}\index{t-structure!gluing of ---s|see {Gluing}}\index{.glue@$\glue$}
The main result in the classical theory of recollements is the so-called \emph{gluing theorem}, which tells us how to obtain a $t$-structure $\tee = \tee_0 \glue \tee_1$\footnote{The symbol $\glue$ (pron. \emph{glue}) reminds the alchemical token describing the process of \emph{amalgamation} between two or more elements (one of which is often mercury): albeit amalgamation is not recognized as a proper stage of the \emph{Magnum Opus}, several sources testify that it belongs to the alchemical tradition (see \cite[pp. \textbf{409-498}]{roth1976deutsches}).} on $\D$ starting from two $t$-structures $\tee_i$ on the categories $\D^i$ of a recollement $\rec$.
\begin{theorem}[Gluing Theorem]\label{gluing}
Consider a recollement $$\rec = (i,q)\colon \D^0 \lrlarrows  \D \lrlarrows  \D^1,$$ and let $\tee_i$ be $t$-structures on $\D^i$ for $i=0,1$; then there exists a $t$-structure on $\D$, called the \emph{gluing} of the $\tee_i$ (along the recollement $\rec$, but this specification is almost always omitted) and denoted $\tee_0\glue \tee_1$, whose classes $\big( (\D^0\glue \D^1)_{\ge 0}, (\D^0\glue \D^1)_{<0}\big)$ are given by
\begin{gather}
(\D^0\glue \D^1)_{\ge 0} = \Big\{ X\in\D\mid (q X\in \D_{\ge 0}^1)\land (i_L X\in \D_{\ge 0}^0) \Big\};\notag \\
(\D^0\glue \D^1)_{<0} = \Big\{ X\in\D\mid ( q X\in \D_{< 0}^1)\land (i_R X\in \D_{< 0}^0) \Big\}.\label{glued}
\end{gather}
\end{theorem}
\begin{remark}
Following Notation \refbf{veryshort} we have that $X\in \D_{\ge 0}$ iff $\{q, i_L\}(X)\in \D_{\ge 0}$  and $Y\in \D_{<0}$ iff $\{q, i_R\}(X)\in \D_{<0}$, which is a rather evocative statement: the left/right class of $\tee_0\glue \tee_1$ is determined by the left/right adjoint to $i$.%\colon \D^0\to \D^0\glue \D^1$.
\end{remark}

\begin{remark}
The ``wrong way'' classes
\begin{gather}
(\D^0\glue \D^1)^\bigstar_{\ge 0} = \Big\{ X\in\D\mid (\{q,i_R\} X\in \D_{\ge 0} \Big\};\notag \\
(\D^0\glue \D^1)^\bigstar_{<0} = \Big\{ X\in\D\mid (\{q,i_L\} X\in \D_{< 0}  \Big\}.
\end{gather}
do not define a $t$-structure in general. However they do in the case the recollement situation $\rec$ is the lower part of a 2-\emph{recollement}, i.e., there exists a diagram of the form
\[ 
\xymatrix{
  \mathbf{C}^0	& \mathbf{C}	& \mathbf{C}^1
  \ar@{<-}@<9pt> "1,1";"1,2" ^{i_1}
  \ar@<3pt> "1,1";"1,2" |{i_2}
  \ar@{<-}@<-3pt> "1,1";"1,2" |{i_3}
  \ar@<-9pt> "1,1";"1,2" _{i_4}
  \ar@{<-}@<9pt> "1,2";"1,3" ^{q_1}
  \ar@<3pt> "1,2";"1,3" |{q_2}
  \ar@{<-}@<-3pt> "1,2";"1,3" |{q_3}
  \ar@<-9pt> "1,2";"1,3" _{q_4}
}
\]
where both
\[
\rec_2=\xymatrix{
  \mathbf{C}^0	&\mathbf{C}	& \mathbf{C}^1
  \ar|{i_2} "1,1";"1,2" 
  \ar@<8pt>^{i_3} "1,2";"1,1" 
  \ar@<-8pt>_{i_1} "1,2";"1,1" 
  \ar|{q_2} "1,2";"1,3" 
  \ar@<8pt>^{q_3} "1,3";"1,2" 
  \ar@<-8pt>_{q_1} "1,3";"1,2" 
}
\]
and
\[
\rec_3=\xymatrix{
 \mathbf{C}^1	& \mathbf{C}	& \mathbf{C}^0
  \ar|{q_3} "1,1";"1,2" 
  \ar@<8pt>^{q_4} "1,2";"1,1" 
  \ar@<-8pt>_{q_2} "1,2";"1,1" 
  \ar|{i_3} "1,2";"1,3" 
  \ar@<8pt>^{i_4} "1,3";"1,2" 
  \ar@<-8pt>_{i_2} "1,3";"1,2" 
}
\]
are recollements, with $\rec=\rec_3$. Indeed, in this situation one has
\begin{align*}
(\D^0\glue \D^1)^\bigstar_{\ge 0} &= \Big\{ X\in\D\mid (\{q,i_R\} X\in \D_{\ge 0} \Big\}\\
&= \Big\{ X\in\mathbf{C}\mid (\{i_3,q_2\} X\in \mathbf{C}_{\ge 0} \Big\}\\
&=(\mathbf{C}^0\glue^{\rec_2} \mathbf{C}^1)_{\ge 0}.
\end{align*}
More generally, an $n$-recollement is defined as the datum of three stable $\infty$-categories $\mathbf{C}^0, \mathbf{C}, \mathbf{C}^1$ organized in a diagram
\[\label{diag:nreco}
	\xymatrix{
	  \D^0	& \D	& \D^1
	  \ar@<5pt> "1,1";"1,2" |{i_2}
	  \ar@{<-}@<11pt> "1,1";"1,2" ^{i_1}
	  \ar@{<-}@<-3pt> "1,1";"1,2" |{i_3}
	  \ar@<-10pt>@{}|\vdots "1,1";"1,2" 
	  \ar@{<-}@<-15pt> "1,1";"1,2" _{i_{n+2}}
	  \ar@<5pt> "1,2";"1,3" |{q_2}
	  \ar@{<-}@<11pt> "1,2";"1,3" ^{q_1}
	  \ar@{<-}@<-3pt> "1,2";"1,3" |{q_3}
	  \ar@<-10pt>@{}|\vdots "1,2";"1,3" 
	  \ar@{<-}@<-15pt> "1,2";"1,3" _{q_{n+2}}
	}
\]with $n+2$ functors on each edge, such that every consecutive three functors form recollements  $\rec_{2k} = (i_{2k}, q_{2k})$, $\rec_{2h+1} = (q_{2h+1}, i_{2h+1})$, for $k=1, \dots, n-1$, $h=1, \dots, n-2$, see \cite[Def. \textbf{2}]{nrecol1}.  Applications of this formalism to derived categories of algebras, investigating the relationships between the recollements of derived categories
and the Gorenstein properties of these algebras, can be found in \cite{nrecol1,nrecol2}.
\end{remark}


\begin{notat}
\label{notat:recdec}
It is worth to notice that $\D^0\glue \D^1$ \index{.D^0glueD^1@$\D^0\glue \D^1$}has no real meaning as a category; this is only an intuitive shorthand to denote the pair $(\D,\tee_0\glue\tee_1)$; \marginnote{\dbend} more explicitly, it is a shorthand to denote the following situation:
\begin{quote}
The stable $\infty$-category $\D$ fits into a recollement $(i,q)\colon \D^0\lrlarrows \D \lrlarrows \D^1$, $t$-structures on $\D^0$ and $\D^1$ have been chosen, and $\D$ is endowed with the glued $t$-structure $\tee_0\glue \tee_1$.
\end{quote}
%This notation will prove extremely useful all along \S\refbf{properties}.
\end{notat}
A proof of the gluing theorem in the classical setting of triangulated categories can be found in \cite[Thm. \textbf{7.2.2}]{Banagl} or in the standard reference \cite{BBDPervers}. For the sake of completeness, we briefly sketch the argument given in \cite{Banagl} rephrasing it in the language of stable $\infty$-categories.
%as we will need it in the torsio-centric reformulation of the gluing theorem. 
\begin{proof}[Proof of Thm. \protect{\ref{gluing}}]
We begin showing the way in which every $X\in\D$ fits into a fiber sequence $SX\to X\to RX$ where $SX\in (\D^0\glue \D^1)_{\ge 0}, RX\in (\D^0\glue \D^1)_{<0}$. \index{Gluing!co/reflection of a ---}
Let $\fF_i$ denote the normal torsion theory on $\D^i$, inducing the $t$-structure $\tee_i$ according to the notation of \refbf{}; let $\eta_1 \colon qX\to R_1 qX$ be the arrow in the fiber sequence
\[\label{fibseqofq}
S_1 qX \xto{\epsilon_1} qX\xto{\eta_1} R_1 qX
\]
obtained thanks to $\fF_1$; let $\hat\eta$ be its \emph{mate} $X\to q_R R_1 qX$ in $\D$ under the adjunction $q\dashv q_R$, and let $WX=\fib(\hat \eta)$. 

Now, consider $i_L WX$ in the fiber sequence 
\[
S_0 i_L WX \xto{\sigma_0} i_L WX \xto{\theta_0} R_0 i_LWX\]
induced by $\fF_0$ on $\D_0$, and its mate $\hat \theta\colon WX\to i R_0 i_LWX$; take its fiber $SX$, and the object $RX$ defined as the pushout of $i R_0 i_L WX \xot{\hat\theta}
 WX \to X$.

To prove that these two objects are the candidate co/trun\-ca\-tion we consider the diagram
\[\label{star}
\xymatrix{
  SX	& WX	& X \\
  0	& i R_0 i_L WX	& RX \\
  	& 0	& q_R R_1 qX
  \ar "1,1";"1,2" 
  \ar "1,1";"2,1" 
  \ar "1,2";"1,3" 
  \ar "1,2";"2,2" _{\hat\theta}
  \ar@{.>}@/^1.4pc/ "1,3";"3,3" ^{\hat\eta}
  \ar "1,3";"2,3" 
  \ar "2,1";"2,2" 
  \ar "2,2";"2,3" 
  \ar "2,2";"3,2" 
  \ar "2,3";"3,3" 
  \ar "3,2";"3,3" 
}\]
where all the mentioned objects fit, and where every square is a pullout. We have to prove that $SX\in (\D^0\glue \D^1)_{\ge 0}$ and $RX\in (\D^0\glue \D^1)_{<0}$. To do this, apply the functors $q, i_L, i_R$ to (\refbf{star}), obtaining the following diagram of pullout squares (recall the e\-xact\-ness properties of the recollement functors, stated in \aprop \refbf{recoexact}):
\[
\scalebox{.9}{%
\xymatrix@C=6mm{
  qSX	& qWX	& qX \\
  0	& 0	& q RX \\
  	& 0	& R_1 qX
  \ar^\sim "1,1";"1,2" 
  \ar "1,1";"2,1" 
  \ar "1,2";"1,3" 
  \ar "1,2";"2,2" 
  \ar "1,3";"2,3" 
  \ar@{=} "2,1";"2,2" 
  \ar "2,2";"2,3" 
  \ar@{=} "2,2";"3,2" 
  \ar "2,3";"3,3" 
  \ar "3,2";"3,3" 
}%\quad
\xymatrix@C=6mm{
  i_L SX	& i_L WX	& i_L X \\
  0	& R_0 i_L WX	& i_L RX \\
  	& 0	& i_L q_R R_1 qX
  \ar@{} "1,1";"2,2" |{\text{\large \uno}}
  \ar "1,1";"1,2" 
  \ar "1,1";"2,1" 
  \ar "1,2";"1,3" 
  \ar "1,2";"2,2" 
  \ar "1,3";"2,3" 
  \ar "2,1";"2,2" 
  \ar "2,2";"2,3" 
  \ar "2,2";"3,2" 
  \ar "2,3";"3,3" 
  \ar "3,2";"3,3" 
}
%\quad
\xymatrix@C=6mm{
  i_R SX	& i_R WX	& i_R X \\
  0	& R_0 i_L WX	& i_R RX \\
  	& 0	& 0
  \ar "1,1";"1,2" 
  \ar "1,1";"2,1" 
  \ar "1,2";"1,3" 
  \ar "1,2";"2,2" 
  \ar "1,3";"2,3" 
  \ar "2,1";"2,2" 
  \ar^\sim "2,2";"2,3" 
  \ar "2,2";"3,2" 
  \ar "2,3";"3,3" 
  \ar@{=} "3,2";"3,3" 
}}\notag
\]
where we took into account the relations $qi=0, i_Rq_R = 0 = i_L q_L$. We find that
\begin{itemize}
\item $qSX\cong qWX\cong S_1 qX\in \D_{\ge 0}^1$, since $0\to S_1 qX$ lies in $\MM_1$, and $qRX\cong R_1 qX\in\D_{<0}^1$;
\item $i_L SX\cong S_0 i_L WX \in \D_{\ge 0}^0$, by the pullout square \text{\uno};
\item $i_R RX\cong R_0 i_L W X\in \D_{<0}^0$.
\end{itemize}
It remains to show that the two classes $\D_{\ge 0},\D_{<0}$ are orthogonal; to see this, suppose that $X\in \D_{\ge 0}$ and $Y\in \D_{<0}$. We consider the fiber sequence $ii_R Y\to Y\to q_R q Y$ of axiom $(\oldstylenums{4})$ in \abbrv{Def} \refbf{def:recol}, to obtain (applying the homological functor $\D(X, -)$) 
\[
\xymatrix@R=2mm{
  \D(X, ii_R Y)	& \D(X, Y)	& \D(X,q_R q Y) \\
  \D(i_LX, i_R Y)	& 	& \D(qX, qY) \\
  0	& 	& 0
  \ar@{=} "1,1";"2,1" 
  \ar "1,1";"1,2" 
  \ar "1,2";"1,3" 
  \ar@{=} "1,3";"2,3" 
  \ar@{=} "2,1";"3,1" 
  \ar@{=} "2,3";"3,3" 
}
\]and we conclude, thanks to the exactness of this sequence.
\end{proof}
\begin{remark}
The definition of $\tee_0\glue \tee_1$ entails that all the recollement functors
 $(i,q)\colon $ $(\D^0, \tee_0) \lrlarrows (\D, \tee_0\glue \tee_1) \lrlarrows (\D^1, \tee_1)$ become $t$-exact in the sense of \cite[\abbrv{Def} \textbf{1.3.3.1}]{LurieHA}. 
\end{remark}
\fare
\begin{remark}
Strictly speaking, the domain of definition of the gluing operation $\glue$ is the set of triples $(\tee_0, \tee_1, \rec)$ where $(\tee_0,\tee_1)\in \ts (\D^0)\times \ts (\D^1)$ and $\rec=(i,q)$ is a recollement $\D^0\lrlarrows \D \lrlarrows \D^1$, but unless this (rather stodgy) distinction is strictly necessary we will adopt an obvious abuse of notation.
\end{remark}
\begin{remark}[A standard technique]\label{generalproced}
The procedure outlined above is in some sense paradigmatic, and it's worth to trace it out as an abstract way to deduce properties about objects and arrows fitting in a diagram like (\refbf{star}). This algorithm will be our primary technique to prove statements in the ``torsio-centric'' formulation of recollements:
\begin{itemize}
\item We start with a particular diagram, like for example (\refbf{star}) or (\refbf{twostar}) below; our aim is to prove that a property (being invertible, being the zero map, lying in a distinguished class of arrows, etc.\@\xspace) is true for an arrow $h$ in this diagram.
\item We apply (possibly only some of) the recollement functors to the diagram, and we deduce that $h$ has the above property from
\begin{itemize}
\item The recollement relations between the functors (\abbrv{Def} \refbf{def:recol});
\item The exactness of the recollement functors (\abbrv{Prop} \refbf{recoexact});
\item The joint reflectivity of the pairs $\{q, i_L\}$ and $\{q, i_R\}$ (Lemma \refbf{reflectsall});
\end{itemize}
\end{itemize}
\end{remark}


\section{Stable Recollements.}\label{stabrecoll}

\renewcommand{\textflush}{flushright}
\epigraph{MANCA!
  %בּוֹ: וְיֹרְדִים עֹלִים אֱלֹהִים מַלְאֲכֵי וְהִנֵּה הַשָּׁמָיְמָה8 מַגִּיעַ7 וְרֹאשׁוֹ6 אַרְצָה5 מֻצָּב4 סֻלָּם3 וְהִנֵּה2 וַיַּחֲלֹםINIZIO
}{\textsc{Genesis} \oldstylenums{28}:\oldstylenums{12}}
% \cjRL{	way*a.ha:loM w:hin*eh sul*AM mu.s*Ab 
% 		'ar:.sAh w:ro'+sO mag*i`a ha+s*AmAy:mAh 
% 		w:hin*eh mal:'a:key 'E:lohiym `oliym 
% 		w:yor:diym b*wo;}}
%	{\textsc{Genesis} \oldstylenums{28}:\oldstylenums{12}}
\renewcommand{\textflush}{flushleft}
\index{Gluing!``Jacob ladder''}
\subsection{The Jacob's ladder: building co/reflections.}
The above procedure to build the functors $R,S$ depends on several choices (we forget half of the fiber sequence $S_1 q X\to qX\to R_1 qX$) and it doesn't seem independent from these choices, at least at first sight. 

The scope of this first subsection is to show that this apparent asymmetry arises only because we are hiding half of the construction, taking into account only half of the fiber sequence (\refbf{fibseqofq}). Given an object $X\in \D$ a dual argument yields \emph{another} way to construct a fiber sequence
\[
	S' X\to X\to R' X
\]
out of the recollement data, which is naturally isomorphic to the former $SX\to X\to RX$. 

We briefly sketch how this dualization process goes: starting from the coreflection arrow $\epsilon_1 \colon S_1qX\to qX$, taking its mate $q_LS_1 qX\to X$ under the adjunction $q_L\dashv q$, and reasoning about its cofiber we can build a diagram which is dual to the former one, and where every square is a pullout:
\[\label{twostar}
\scalebox{.9}{\xymatrix{
                q_L S_1 q X	& S'X	& X \\
                0	& i S_0 i_R KX	& KX \\
                	& 0	& R'X
                \ar "1,1";"1,2" 
                \ar "1,1";"2,1" 
                \ar "1,2";"1,3" 
                \ar "1,2";"2,2" 
                \ar "1,3";"2,3" 
                \ar "2,1";"2,2" 
                \ar "2,2";"2,3" 
                \ar "2,2";"3,2" 
                \ar "2,3";"3,3" 
                \ar "3,2";"3,3" 
              }}
\]
\begin{proposition}[The Jacob's ladder]\label{thejacbo}
The two squares of the previous constructions fit into a ``ladder'' induced by canonical isomorphisms $SX\cong S'X, RX\cong R'X$; the construction is functorial in $X$. The ``Jacob's ladder'' is the following diagram:
\[\label{ladder:objs}
\scalebox{.9}{\xymatrix{
                q_L S_1 qX	& SX	& WX	& X \\
                0	& iS_0 i_R KX	& CX	& KX \\
                	& 0	& i R_0 i_L WX	& RX \\
                	& 	& 0	& q_R R_1 q X
                \ar "1,1";"1,2" 
                \ar "1,1";"2,1" 
                \ar "1,2";"1,3" 
                \ar "1,2";"2,2" 
                \ar "1,3";"1,4" 
                \ar "1,3";"2,3" 
                \ar "1,4";"2,4" 
                \ar "2,1";"2,2" 
                \ar "2,2";"2,3" 
                \ar "2,2";"3,2" 
                \ar "2,3";"2,4" 
                \ar "2,3";"3,3" 
                \ar "2,4";"3,4" 
                \ar "3,2";"3,3" 
                \ar "3,3";"3,4" 
                \ar "3,3";"4,3" 
                \ar "3,4";"4,4" 
                \ar "4,3";"4,4" 
              }}
\]
\end{proposition}
\begin{proof}
It suffices to prove that both $SX, S'X$ lie in $\D_{\ge 0}$ and both $RX, R'X$ lie in $\D_{\le 0}$; given this, we can appeal (a suitable stable $\infty$-categorical version of) \cite[\abbrv{Prop} \textbf{1.1.9}]{BBDPervers} which asserts the functoriality of the truncation functors, \ie that when the same object $X$ fits into \emph{two} fiber sequences arising from the same normal torsion theory, then there exist the desired isomorphisms\footnote{In a torsio-centric perspective, this follows from the uniqueness of the factorization of a morphism with respect to the normal torsion theory having reflection $R$ and coreflection $S$; see \refbf{rmk:uniqueness-of-fact}.}.

The procedure showing this is actually the same remarked in \refbf{generalproced}: we apply $q,i_L, i_R$ to the diagram (\refbf{twostar}) and we exploit exactness of the recollement functors to find pullout diagrams showing that $R'X\in \D_{<0}$ and $S'X\in \D_{\ge 0}$.

Once these isomorphisms have been found, it remains only to glue the two sub-diagrams
$$\scalebox{.8}{
\xymatrix@R=7mm@C=4mm{
 \gray{q_L S_1 qX}	& SX	& WX	& X \\
 \gray{0}	& \gray{iS_0 i_R KX}	& \gray{CX}	& \gray{KX} \\
 	& 0	& i R_0 i_L WX	& RX \\
 	& 	& 0	& q_R R_1 q X
 \ar@{.>}@[semilightgray] "1,1";"1,2" 
 \ar@{.>}@[semilightgray] "1,1";"2,1" 
 \ar "1,2";"1,3" 
 \ar@{-} "1,2";"2,2" 
 \ar "1,3";"1,4" 
 \ar@{-} "1,3";"2,3" 
 \ar@{-} "1,4";"2,4" 
 \ar@{.>}@[semilightgray] "2,1";"2,2" 
 \ar@{.>}@[semilightgray] "2,2";"2,3" 
 \ar "2,2";"3,2" 
 \ar@{.>}@[semilightgray] "2,3";"2,4" 
 \ar "2,3";"3,3" 
 \ar "2,4";"3,4" 
 \ar "3,2";"3,3" 
 \ar "3,3";"3,4" 
 \ar "3,3";"4,3" 
 \ar "3,4";"4,4" 
 \ar "4,3";"4,4" 
}
%\qquad
	\xymatrix@R=7mm@C=4mm{
	  q_L S_1 qX	& S'X	& \gray{WX}	& X \\
	  0	& iS_0 i_R KX	& \gray{CX}	& KX \\
	  	& 0	& \gray{i R_0 i_L WX}	& R'X \\
	  	& 	& \gray{0}	& \gray{q_R R_1 q X}
	  \ar "1,1";"1,2" 
	  \ar "1,1";"2,1" 
	  \ar@{-} "1,2";"1,3" 
	  \ar "1,2";"2,2" 
	  \ar "1,3";"1,4" 
	  \ar@{.>}@[semilightgray] "1,3";"2,3" 
	  \ar "1,4";"2,4" 
	  \ar "2,1";"2,2" 
	  \ar@{-} "2,2";"2,3" 
	  \ar "2,2";"3,2" 
	  \ar "2,3";"2,4" 
	  \ar@{.>}@[semilightgray] "2,3";"3,3" 
	  \ar "2,4";"3,4" 
	  \ar@{-} "3,2";"3,3" 
	  \ar "3,3";"3,4" 
	  \ar@{.>}@[semilightgray] "3,3";"4,3" 
	  \ar@{.>}@[semilightgray] "3,4";"4,4" 
	  \ar@{.>}@[semilightgray] "4,3";"4,4" 
	}
}$$
to obtain the ladder. Now, this construction is obtained by taking into account the fiber sequence $S_1 q X\to q X\to R_1 q X$ as a whole, and since this latter object is uniquely determined up to isomorphism, we obtain a diagram of endofunctors
\[\label{ladder:arrows}
	\scalebox{.8}{\xymatrix{
	                q_L S_1 q	& S	& W	& 1 \\
	                0	& iS_0 i_R K	& C	& K \\
	                	& 0	& i R_0 i_L W	& R \\
	                	& 	& 0	& q_R R_1 q
	                \ar "1,1";"1,2" 
	                \ar "1,1";"2,1" 
	                \ar "1,2";"1,3" 
	                \ar "1,2";"2,2" 
	                \ar "1,3";"1,4" 
	                \ar "1,3";"2,3" 
	                \ar "1,4";"2,4" 
	                \ar "2,1";"2,2" 
	                \ar "2,2";"2,3" 
	                \ar "2,2";"3,2" 
	                \ar "2,3";"2,4" 
	                \ar "2,3";"3,3" 
	                \ar "2,4";"3,4" 
	                \ar "3,2";"3,3" 
	                \ar "3,3";"3,4" 
	                \ar "3,3";"4,3" 
	                \ar "3,4";"4,4" 
	                \ar "4,3";"4,4" 
	              }}
\]where every square is a pullout (again giving to a category of functors the obvious stable structure \cite[\abbrv{Prop} \textbf{1.1.3.1}]{LurieHA}), and where the functorial nature of $W$, $K$ and $C$ is a consequence of their construction. Notice also that this latter diagram of functors uses homogeneously all the recollement functors, and that it is ``symmetric'' with respect to the antidiagonal (it switches left and right adjoints, as well as reflections and coreflections).
\end{proof}
The functors $S,R$ are the co/truncations for the recoll\'ee $t$-structure, and %in such a way that 
the normality of the torsion theory is witnessed by the pullout subdiagram
\[\label{normalite}
	\scalebox{.8}{\xymatrix@!R=7mm{
	                SX	& \gray{WX}	& X \\
	                \gray{iS_0 i_RKX}	& \gray{CX}	& \gray{KX} \\
	                0	& \gray{i R_0 i_L WX}	& RX.
	                \ar@{}|(.3)\lrcorner "1,1";"2,2" 
	                \ar@{-} "1,1";"1,2" 
	                \ar@{-} "1,1";"2,1" 
	                \ar "1,2";"1,3" 
	                \ar@{.>}@[semilightgray] "1,2";"2,2" 
	                \ar@{-} "1,3";"2,3" 
	                \ar@{.>}@[semilightgray] "2,1";"2,2" 
	                \ar "2,1";"3,1" 
	                \ar@{.>}@[semilightgray] "2,2";"2,3" 
	                \ar@{.>}@[semilightgray] "2,2";"3,2" 
	                \ar "2,3";"3,3" 
	                \ar@{-} "3,1";"3,2" 
	                \ar "3,2";"3,3" 
	                \ar@{}|(.3)\ulcorner "3,3";"2,2" 
	              }}
\]\begin{notat}
From now on, we will always refer to the diagram above as ``the Jacob ladder'' of an object $X\in\D$, and/or to the diagram induced by a morphism $f\colon X\to Y$ between the ladder of the domain and the codomain, \ie to three-dimensional diagrams like
\[\label{functorial-ladder}
\scalebox{.8}{ 
\xymatrix@R=4.5mm@C=2mm{
  	& q_L S_1 q Y	& 	& \textcolor{black}{SY}	& 	& WY	& 	& \textcolor{black}{Y} \\
  q_L S_1 qX	& 	& \textcolor{black}{SX}	& 	& WX	& 	& \textcolor{black}{X} \\
  	& 	& 	& 	& 	& 	& 	& KY \\
  0	& 	& i S_0 i_R KX	& 	& CX	& 	& KX \\
  	& 	& 	& 	& 	& 	& 	& \textcolor{black}{RY} \\
  	& 	& 0	& 	& i R_0 i_L WX	& 	& \textcolor{black}{RX} \\
  	& 	& 	& 	& 	& 	& 	& q_R R_1 qY \\
  	& 	& 	& 	& 0	& 	& q_R R_1 qX
  \ar "1,2";"1,4" 
  \ar "1,4";"1,6" 
  \ar "1,6";"1,8" 
  \ar "1,8";"3,8" 
  \ar "2,1";"1,2" 
  \ar "2,1";"2,3" 
  \ar "2,1";"4,1" 
  \ar "2,3";"1,4" ^{Sf}
  \ar "2,3";"2,5" 
  \ar "2,3";"4,3" 
  \ar "2,5";"1,6" ^{Wf}
  \ar "2,5";"2,7" 
  \ar "2,5";"4,5" 
  \ar "2,7";"4,7" 
  \ar^f "2,7";"1,8" 
  \ar "3,8";"5,8" 
  \ar "4,1";"4,3" 
  \ar "4,3";"6,3" 
  \ar "4,3";"4,5" 
  \ar "4,5";"6,5" 
  \ar "4,5";"4,7" 
  \ar "4,7";"6,7" 
  \ar "4,7";"3,8" ^{Kf}
  \ar "5,8";"7,8" 
  \ar "6,3";"6,5" 
  \ar "6,5";"6,7" 
  \ar "6,5";"8,5" 
  \ar "6,7";"8,7" 
  \ar "6,7";"5,8" ^{Rf}
  \ar "8,5";"8,7" 
  \ar "8,7";"7,8" 
}}
\]
\end{notat}
\subsection{The {\sc ntt} of a recollement.}
\index{Gluing!\smallcap{ntt} of a ---}
Throughout this subsection we outline the torsio-centric translation of the classical results recalled above. In particular we give an explicit definition of the $\glue$ operation when it has been ``transported'' to the set of normal torsion theories, independent from its characterization in terms of the pairs a\-isle-co\-a\-isle of the two $t$-structures. From now on we assume given a recollement
$$
\xymatrix{
  \D^0	& \D	& \D^1.
  \ar|i "1,1";"1,2" 
  \ar@<8pt> "1,2";"1,1" ^{i_L}
  \ar@<-8pt> "1,2";"1,1" _{i_R}
  \ar|q "1,2";"1,3" 
  \ar@<8pt> "1,3";"1,2" ^{q_L}
  \ar@<-8pt> "1,3";"1,2" _{q_R}
}$$
Given $t$-structures $\tee_i\in  \ts (\D^i)$, in view of our characterization theorem \refbf{thm:rosetta}, there exist normal torsion theories $\fF_i=(\EE_i, \MM_i)$ on $\D^i$ such that $(\D^i_{\ge 0}, \D^i_{<0})$ are the classes $(0/\EE_i,\MM_i/0)$ of torsion and torsion-free objects of $\D^i$, for $i=0,1$; an object $X$ lies in $(\D^0\glue \D^1)_{\ge 0}$ if and only if $qX\in \EE_1$ and $i_LX\in \EE_0$\footnote{Thanks to the Sator lemma we are allowed to use ``$X\in \K$'' as a shorthand to denote that either the initial arrow $\var{0}{X}$ or the terminal arrow $\var{X}{0}$ lie in a 3-for-2 class $\K \subseteq \hom(\CC)$. From now on we will adopt this notation.}, and similarly an object $Y$ lies in $\D_{\le 0}$ if and only if $qY\in \MM_1$ and $i_RY\in \MM_0$.
\begin{remark}\label{must-itself-come}
The $t$-structure $\tee = \tee_0\glue \tee_1$ on $\D$ must itself come from a normal torsion theory which we denote $\fF_0\glue \fF_1$ on $\D$, so that $\big((\D^0\glue \D^1)_{\ge 0},(\D^0\glue \D^1)_{<0}\big)=\big(0/(\EE_0\glue\EE_1), (\MM_0\glue \MM_1)/0\big)$; in other words the following three conditions are equivalent for an object $X\in\D$:
\begin{itemize}
\item $X$ lies in $(\D^0\glue \D^1)_{\ge 0}$;
\item $X$ lies in $\EE_0\glue\EE_1$, \ie $RX\cong 0$ in the notation of (\refbf{normalite});
\item $\{q, i_L\}(X)\in \EE$, following Notation \refbf{veryshort}.
\end{itemize}
\end{remark}
We now aim to a torsio-centric characterization of the classes $(\EE_0\glue\EE_1, \MM_0\glue \MM_1)$, relying on the factorization properties of $(\EE_i, \MM_i)$ alone: since we proved \abbrv{Thm}\refbf{gluing} above, there must be a normal torsion theory $\fF_0\glue \fF_1 = \big(\EE_0\glue\EE_1, \MM_0\glue\MM_1\big)$ inducing $\tee_0\glue\tee_1$ as $\big(0/(\EE_0\glue\EE_1), (\MM_0\glue\MM_1)/0\big)$: in other words,
\begin{quote}
$\fF_0\glue \fF_1$ is the (unique) normal torsion theory whose torsion/torsionfree classes are $\big((\D^0\glue \D^1)_{\ge 0},(\D^0\glue \D^1)_{<0}\big)$ of \abbrv{Thm}\refbf{gluing},
\end{quote}
Clearly this is only an application of our ``Rosetta stone'' theorem \refbf{thm:rosetta}, so in some sense this result is ``tautological''. But there are at least two reasons to concentrate in ``proving again'' \abbrv{Thm}\refbf{gluing} from a torsio-centric perspective:
\begin{itemize}
\item The construction offered by the Rosetta stone is rather indirect, and only appropriate to show formal statements about the factorization system $\fF(\tee)$ induced by a $t$-structure;
\item In a stable setting, the torsio-centric point of view, using factorization systems, is more primitive and more natural than the classical one using 1-categorical arguments (\ie, $t$-structures $\tee$ on the homotopy category of a stable $\D$ are induced by normal torsion theories in $\D$; in the quotient process one loses important informations about $\tee$).
\end{itemize}
Both these reasons lead us to adopt a ``constructive'' point of view, giving an explicit characterization of $\fF_0\glue \fF_1$ which relies on properties of the factorization systems $\fF_0$, $\fF_1$ alone, independent from triangulated categorical arguments. 

In the following section we will discuss the structure and properties of the factorization system $\fF_0\glue \fF_1$, concentrating on a self-contained % and categorically well motivated
 construction of the classes $\EE_0\glue\EE_1$ and $\MM_0\glue\MM_1$.% starting from an obvious \emph{ansatz} which follows Remark \refbf{must-itself-come}.%, as we will see in \abbrv{Thm}\refbf{thm:trueglued}.
The discussion above, and in particular the fact that an initial/terminal arrow $0\leftrightarrows X$ lies in $\EE_0\glue\EE_1$ if and only if $\{q, i_L\}(X)\in \EE$, suggests that we define
$\EE_0\glue\EE_1 = \big\{f\in\hom(\D)\mid 
\{q, i_L\}(f)\in \EE 
\big\}$ and $\MM_0\glue\MM_1 = \big\{g\in\hom(\D)\mid 
\{q, i_R\}(g) \in \MM
\big\}$. Actually it turns out that this guess is not far to be correct: the correct classes are indeed given by the following: 
%In fact this is true, and is (half of) our \abbrv{Thm}\refbf{thm:trueglued}. 
%\todo[inline]{Qui comincia il casino}
%The refined gluing theorem for normal torsion theories acquires the following form:
\begin{theorem}
\label{thm:trueglued}
Let $\D$ be a stable $\infty$-category, in a recollement 
$$
(i,q)\colon \D^0 \lrlarrows  \D  \lrlarrows  \D^1,
$$
and let $\tee_i$ be a $t$-structure on $\D^i$. Then the recoll\'ee $t$-structure $\tee_0\glue \tee_1$ is induced by the normal torsion theory $(\EE_0\glue\EE_1,\MM_0\glue\MM_1)$ with classes
\begin{gather}
\EE_0\glue\EE_1 = \big\{f\in\hom(\D)\mid 
\{q, i_LW\}(f)\in \EE 
\big\};\label{BBDleftclass}\\
\MM_0\glue\MM_1 = \big\{g\in\hom(\D)\mid 
\{q, i_RK\}(g) \in \MM
\big\}.
\end{gather}
\end{theorem}
\begin{proof}
We only need to prove the statement for $\EE_0\glue\EE_1$, since the statement for $\MM_0\glue\MM_1$ is completely specular. Thanks to the characterization (\refbf{charac}) of section \S\refbf{classical}, an arrow $f\in\hom(\D)$ lies in $\EE_0\glue \EE_1$ if and only if $Rf$ (as constructed in the Jacob ladder (\refbf{functorial-ladder})) is an isomorphism in $\D$, so we are left to prove that, given $f\in\hom(\D)$:
\[
\text{$Rf$ is an isomorphism in $\D$}\quad\Leftrightarrow\quad 
\text{$\{q, i_LW\}(f)\in \EE$}.
\]
Equivalently, we have to prove that
\[
\text{$Rf$ is an isomorphism}\quad\Leftrightarrow\quad 
\text{$\{R_1q, R_0i_LW\}(f)$ are isomorphisms}.
\]

%For the sake of clarity, we split the proof in different parts: we separately prove the implications 
%\begin{itemize}
%\item (\oldstylenums{1}) implies (\oldstylenums{2});
%\item (\oldstylenums{3}) implies (\oldstylenums{2});
%\item (\oldstylenums{2}) implies that $\{q, i_R, i_L\}(f)\in \EE$ (this part is, itself, divided again in three sub-lemmas, one for each functor $\{q, i_R, i_L\}$).
%\end{itemize}
%\subsection*{$\{ q, i_R\}(f)\in \EE$ implies that $R(f)$ is invertible.}\label{para:firstsneetch}
%Apply the functor $q$ to the Jacob ladder (\refbf{functorial-ladder}), to obtain
%
%\[\label{q-of-Jacob}
%\scalebox{.8}{ 
%\xymatrix@R=2mm@C=3mm{
%  	& S_1 q Y	& 	& q SY	& 	& q WY	& 	& q Y \\
%  S_1 qX	& 	& q SX	& 	& q WX	& 	& q X \\
%  	& 	& 	& 	& 	& 	& 	& q KY \\
%  0	& 	& 0	& 	& \cancel{q CX }	& 	& q KX \\
%  	& 	& 	& 	& 	& 	& 	& q RY \\
%  	& 	& 0	& 	& 0	& 	& qRX \\
%  	& 	& 	& 	& 	& 	& 	& R_1 qY \\
%  	& 	& 	& 	& 0	& 	& R_1 qX
%  \ar^\sim "1,2";"1,4" 
%  \ar^\sim "1,4";"1,6" 
%  \ar "1,6";"1,8" 
%  \ar "1,8";"3,8" 
%  \ar "2,1";"1,2" 
%  \ar^\sim "2,1";"2,3" 
%  \ar "2,1";"4,1" 
%  \ar "2,3";"1,4" 
%  \ar^\sim "2,3";"2,5" 
%  \ar "2,3";"4,3" 
%  \ar "2,5";"1,6" 
%  \ar "2,5";"2,7" 
%  \ar "2,5";"4,5" 
%  \ar "2,7";"4,7" 
%  \ar "2,7";"1,8" 
%  \ar^\wr "3,8";"5,8" 
%  \ar@{=} "4,1";"4,3" 
%  \ar@{=} "4,3";"6,3" 
%  \ar^\sim "4,3";"4,5" 
%  \ar^\wr "4,5";"6,5" 
%  \ar "4,5";"4,7" 
%  \ar^\wr "4,7";"6,7" 
%  \ar "4,7";"3,8" 
%  \ar^\wr "5,8";"7,8" 
%  \ar@{=} "6,3";"6,5" 
%  \ar "6,5";"6,7" 
%  \ar@{=} "6,5";"8,5" 
%  \ar^\wr "6,7";"8,7" 
%  \ar "6,7";"5,8" 
%  \ar "8,5";"8,7" 
%  \ar "8,7";"7,8" 
%}}
%\] The fact that $qf\in \EE_1$ entails $R_1q f\colon R_1 q X\to R_1 q Y$ is an isomorphism, hence $qRf$ is an isomorphism as well, since it fits into the square
%\[
%	\xymatrix{
%	  q R X	& q R Y \\
%	  R_1 q X	& R_1 q Y.
%	  \ar "1,1";"1,2" 
%	  \ar_\wr "1,1";"2,1" 
%	  \ar^\wr "1,2";"2,2" 
%	  \ar_\sim "2,1";"2,2" 
%	}
%\] Our next aim is to prove that $i_R(Rf)$ is invertible too: we proceed in two steps showing that it lies both in $\EE_0$ and $\MM_0$; when this is done the joint conservativity of the pair $\{q, i_R\}$ (Lemma \refbf{reflectsall}) concludes.
%
%The arrow $i_RRf$ lies in $\MM_0$, as it is immediate to notice from the commutative triangle
%\[
%	\xymatrix{
%	  i_RRX	& i_RRY \\
%	  0	& 0
%	  \ar@{>->} "1,1";"2,1" 
%	  \ar "1,1";"1,2" 
%	  \ar@{>->} "1,2";"2,2" 
%	  \ar@{=} "2,1";"2,2" 
%	}
%\] as a consequence of the 3-for-2 closure property for $\MM_0$, in the diagram below\footnote{Several arrows in diagram (\refbf{diag2}) are decorated with a ``two-headed'' arrow $\twoheadrightarrow$ to denote they belong to $\EE_0$: the crucial fact from which this is deduced is that $i_RSX \in \EE_0$ (the curved, dotted arrow) by the \ror lemma, as it will be repeated in a while. Nothing interesting happens in the corner filled with \#'s.}: 
%
%\[
%\scalebox{.8}{
%\label{diag2}
% 	\xymatrix@R=2mm@C=3mm{
% 	  	& \#	& 	& i_R SY	& 	& i_R WY	& 	& i_R Y \\
% 	  \#	& 	& i_R SX	& 	& i_R WX	& 	& i_R X \\
% 	  	& 	& 	& 	& 	& 	& 	& i_R KY \\
% 	  0	& 	& S_0 i_R KX	& 	& i_R CX	& 	& i_R KX \\
% 	  	& 	& 	& 	& 	& 	& 	& i_R RY \\
% 	  	& 	& 0	& 	& R_0 i_L WX	& 	& i_R RX \\
% 	  	& 	& 	& 	& 	& 	& 	& 0 \\
% 	  	& 	& 	& 	& 0	& 	& 0
% 	  \ar "1,2";"1,4" 
% 	  \ar "1,4";"1,6" 
% 	  \ar^\sim "1,6";"1,8" 
% 	  \ar@{->>} "1,8";"3,8" 
% 	  \ar "2,1";"2,3" 
% 	  \ar "2,1";"4,1" 
% 	  \ar "2,3";"2,5" 
% 	  \ar@{.>>}@/_3pc/|\hole "2,3";"6,3" 
% 	  \ar@{->>} "2,3";"4,3" 
% 	  \ar "2,3";"1,4" 
% 	  \ar^\sim "2,5";"2,7" 
% 	  \ar@{->>} "2,5";"4,5" 
% 	  \ar "2,5";"1,6" 
% 	  \ar@{->>} "2,7";"4,7" 
% 	  \ar@{->>} "2,7";"1,8" 
% 	  \ar@{->>} "3,8";"5,8" 
% 	  \ar "4,1";"4,3" 
% 	  \ar "4,3";"4,5" 
% 	  \ar@{->>} "4,3";"6,3" 
% 	  \ar^\sim "4,5";"4,7" 
% 	  \ar@{->>} "4,5";"6,5" 
% 	  \ar@{->>} "4,7";"6,7" 
% 	  \ar "4,7";"3,8" 
% 	  \ar "5,8";"7,8" 
% 	  \ar "6,3";"6,5" 
% 	  \ar^\sim "6,5";"6,7" 
% 	  \ar "6,5";"8,5" 
% 	  \ar "6,7";"5,8" 
% 	  \ar "6,7";"8,7" 
% 	  \ar@{=} "8,5";"8,7" 
% 	  \ar@{=} "8,7";"7,8" 
% 	}}
% \] 
% Showing that $i_RRf$ lies in $\EE_0$ is slightly more involved: from the diagram above, we deduce that $i_RRf\in \EE_0$ is equivalent to several other conditions, and in particular it is implied from $\var{i_R X}{i_R KX}\in \EE_0$: given this the commutativity of the square
%\[
%	\xymatrix{
%	  i_R X	& i_R Y \\
%	  i_R KX	& i_R KY \\
%	  i_R R X	& i_R R Y
%	  \ar@{->>} "1,1";"1,2" 
%	  \ar@{->>} "1,1";"2,1" 
%	  \ar@{->>} "1,2";"2,2" 
%	  \ar "2,1";"2,2" 
%	  \ar@{->>} "2,1";"3,1" 
%	  \ar@{->>} "2,2";"3,2" 
%	  \ar "3,1";"3,2" 
%	}
%\] gives the result, thanks to the 3-for-2 closure property for $\EE_0$. To prove that $\var{i_R X}{i_R KX}\in \EE_0$, return to diagram (\refbf{diag2}): by the \ror lemma $i_R SX\in \EE_0$ too; now consider the commutative triangle
%\[
%	\xymatrix{
%	  i_R SX	& S_0 i_R KX \\
%	  0	& 0
%	  \ar@{->>} "1,1";"2,1" 
%	  \ar "1,1";"1,2" 
%	  \ar@{->>} "1,2";"2,2" 
%	  \ar@{=} "2,1";"2,2" 
%	}
%\] from which we deduce that $\var{i_R SX}{S_0 i_R KX}\in \EE_0$. Finally, consider the pullout square
%\[
%	\xymatrix{
%	  i_R SX	& i_R X \\
%	  S_0 i_R KX	& i_R KX
%	  \ar "1,1";"1,2" 
%	  \ar@{->>} "1,1";"2,1" 
%	  \ar "1,2";"2,2" 
%	  \ar "2,1";"2,2" 
%	}
%\] This entails that $\var{i_R X }{i_R KX}\in \EE_0$, since $\EE_0$ is closed under pushouts in $\D^0$.
%\subsection*{$\{ q, i_LW\}(f)\in \EE$ implies that $R(f)$ is invertible.}
We begin by showing that if $\{ R_1q, R_0i_LW\}(f)$ are isomorphisms, then also $Rf$ is an isomorphism.
By the joint conservativity of the recollement data (Lemma \ref{reflectsall}) we need to prove that  if $\{ R_1q, R_0i_LW\}(f)$ are isomorphisms, then both $qRf$ and $i_LRf$ are isomorphisms.
Apply the functor $q$ to the Jacob ladder (\refbf{functorial-ladder}), to obtain

\[\label{q-of-Jacob}
\scalebox{.8}{ 
\xymatrix@R=2mm@C=3mm{
  	& S_1 q Y	& 	& q SY	& 	& q WY	& 	& q Y \\
  S_1 qX	& 	& q SX	& 	& q WX	& 	& q X \\
  	& 	& 	& 	& 	& 	& 	& q KY \\
  0	& 	& 0	& 	& q CX 	& 	& q KX \\
  	& 	& 	& 	& 	& 	& 	& q RY \\
  	& 	& 0	& 	& 0	& 	& qRX \\
  	& 	& 	& 	& 	& 	& 	& R_1 qY \\
  	& 	& 	& 	& 0	& 	& R_1 qX
  \ar^\sim "1,2";"1,4" 
  \ar^\sim "1,4";"1,6" 
  \ar "1,6";"1,8" 
  \ar "1,8";"3,8" 
  \ar "2,1";"1,2" 
  \ar^\sim "2,1";"2,3" 
  \ar "2,1";"4,1" 
  \ar "2,3";"1,4" 
  \ar^\sim "2,3";"2,5" 
  \ar "2,3";"4,3" 
  \ar "2,5";"1,6" 
  \ar "2,5";"2,7" 
  \ar "2,5";"4,5" 
  \ar "2,7";"4,7" 
  \ar "2,7";"1,8" 
  \ar^\wr "3,8";"5,8" 
  \ar@{=} "4,1";"4,3" 
  \ar@{=} "4,3";"6,3" 
  \ar^\sim "4,3";"4,5" 
  \ar^\wr "4,5";"6,5" 
  \ar "4,5";"4,7" 
  \ar^\wr "4,7";"6,7" 
  \ar "4,7";"3,8" 
  \ar^\wr "5,8";"7,8" 
  \ar@{=} "6,3";"6,5" 
  \ar "6,5";"6,7" 
  \ar@{=} "6,5";"8,5" 
  \ar^\wr "6,7";"8,7" 
  \ar "6,7";"5,8" 
  \ar "8,5";"8,7" 
  \ar "8,7";"7,8" 
}}
\] 
%The fact that $qf\in \EE_1$ entails $R_1q f\colon R_1 q X\to R_1 q Y$ is an isomorphism, 
Hence $qRf$ is an isomorphism, since it fits into the square
\[
	\xymatrix{
	  q R X	& q R Y \\
	  R_1 q X	& R_1 q Y.
	  \ar "1,1";"1,2" 
	  \ar_\wr "1,1";"2,1" 
	  \ar^\wr "1,2";"2,2" 
	  \ar_\sim "2,1";"2,2" 
	}
\] 
%Applying $q$ to the Jacob ladder and reasoning as before we obtain that $q(Rf)$ is an isomorphism. It remains to show that $i_LRf$ is an isomorphism too, to conclude thanks to the joint reflectivity of the pair $\{q, i_L\}$. 
%
%To see this last statement, 
Now apply the functor $i_L$ to the Jacob ladder, obtaining 
\[ 
\label{diag1bis}
\scalebox{.8}{
\xymatrix@R=2mm@C=3mm{
  	& 0	& 	& i_L SY	& 	& i_L WY	& 	& i_L Y \\
  0	& 	& i_L SX	& 	& i_L WX	& 	& i_L X \\
  	& 	& 	& 	& 	& 	& 	& i_L KY \\
  0	& 	& S_0 i_L KX	& 	& i_L CX	& 	& i_L KX \\
  	& 	& 	& 	& 	& 	& 	& i_L RY \\
  	& 	& 0	& 	& R_0 i_L WX	& 	& i_L RX \\
  	& 	& 	& 	& 	& 	& 	&i_Lq_R R_1 qY \\
  	& 	& 	& 	& 0	& 	& i_Lq_R R_1 qX
  \ar "1,2";"1,4" 
  \ar "1,4";"1,6" 
  \ar "1,6";"1,8" 
  \ar_\wr "1,8";"3,8" 
  \ar "2,1";"2,3"
   \ar "2,1";"1,2" 
  \ar@{=} "2,1";"4,1" 
  \ar "2,3";"2,5" 
  \ar_\wr "2,3";"4,3" 
  \ar "2,3";"1,4" 
  \ar "2,5";"2,7" 
  \ar_\wr "2,5";"4,5" 
  \ar "2,5";"1,6" 
  \ar@{} "2,7";"3,8" |{\uno}%{\text{\large \uno}}
  \ar_\wr "2,7";"4,7" 
  \ar "2,7";"1,8" 
  \ar@{->>} "3,8";"5,8" 
  \ar "4,1";"4,3" 
  \ar "4,3";"4,5" 
  \ar "4,3";"6,3" 
  \ar "4,5";"4,7" 
  \ar@{->>} "4,5";"6,5" 
  \ar@{} "4,7";"5,8" |{\due}%{\text{\large \ding{193}}}
  \ar@{->>} "4,7";"6,7" 
  \ar "4,7";"3,8" 
  \ar "5,8";"7,8" 
  \ar "6,3";"6,5" 
  \ar "6,5";"6,7" 
  \ar "6,5";"8,5" 
  \ar "6,7";"5,8" 
  \ar "6,7";"8,7" 
  \ar "8,5";"8,7" 
  \ar "8,7";"7,8"
}}
\]
%The arrow $i_LRf\colon i_L RX\to i_LRY$ lies in $\EE_0$ by the 3-for-2 property. Since $i_LWf\colon i_LWX\to i_LWY$ is in $\EE_0$ by hypothesis,  $R_0i_LWf\colon R_0i_LWX\to R_0i_LWY$ is an isomorphism. 
As noticed above, $R_1q f$ is an isomorphism, so also 
$i_Lq_RR_1q f$ is an isomorphism. Then $i_LRf$ is an isomorphism by the five-lemma applied to the morphism of fiber sequences
\begin{equation}\label{morphism-of-fiber-sequences}
\scalebox{.8}{
\xymatrix@R=2mm@C=3mm{
  	 	 		 	&  	& R_0 i_L WY	& i_L RY \\
  	 			 R_0 i_L WX	& 	& i_L RX \\
  	 	 		 	& 	& 	&i_Lq_R R_1 qY \\
  	 	 	 	 0	& 	& i_Lq_R R_1 qX
  \ar "2,3";"4,3" 
   \ar "2,1";"2,3" 
  \ar "2,1";"4,1" 
  \ar "4,1";"4,3" 
    \ar "2,3";"1,4" 
   \ar "4,3";"3,4" 
  \ar "1,4";"3,4" 
  \ar "2,1";"1,3"
   \ar "1,3";"1,4" 
 }}
\end{equation}

Vice versa: assuming $Rf$ is an isomorphism in $\D$, we want to prove that $\{R_1 q , R_0i_LW \}(f)$ are isomorphisms.  Diagram (\refbf{q-of-Jacob}) gives directly that $R_1 qf$ is an isomorphism, since the square
\[
	\xymatrix{
	  q RX	& q RY \\
	  R_1 q X	& R_1 q Y
	  \ar^\sim "1,1";"1,2" 
	  \ar_\wr "1,1";"2,1" 
	  \ar^\wr "1,2";"2,2" 
	  \ar "2,1";"2,2" 
	}
\] is commutative.
%Applying $R_0$ to $i_R(Jacob)$ (\ie to diagram (\refbf{diag2})) we obtain a diagram like the following, depicted as a farsighted person would do:
%\[
% \scalebox{.75}{\xymatrix@!C=.1mm@R=4mm{
%                  	& 	& 	& 	& 	& 	& 	&  \\
%                  	& 	& 	& 	& 	& 	&  \\
%                  	& 	& 	& 	& 	& 	& 	&  \\
%                  	& 	& 	& 	& 	& 	&  \\
%                  	& 	& 	& 	& 	& 	& 	&  \\
%                  	& 	& 	& 	& 	& 	&  \\
%                  	& 	& 	& 	& 	& 	& 	&  \\
%                  	& 	& 	& 	& 	& 	& 
%                  \ar "1,2";"1,4" 
%                  \ar "1,4";"1,6" 
%                  \ar|\bullet "1,6";"1,8" 
%                  \ar|\bullet "1,8";"3,8" 
%                  \ar "2,1";"4,1" 
%                  \ar "2,1";"2,3" 
%                  \ar|\bullet "2,3";"4,3" 
%                  \ar "2,3";"1,4" 
%                  \ar "2,3";"2,5" 
%                  \ar|\bullet "2,5";"4,5" 
%                  \ar "2,5";"1,6" 
%                  \ar|\bullet "2,5";"2,7" 
%                  \ar@{} "2,7";"3,8" |{\cin}%{\text{\large \ding{196}}}
%                  \ar|\bullet "2,7";"4,7" 
%                  \ar|\circ "2,7";"1,8" 
%                  \ar|\bullet "3,8";"5,8" 
%                  \ar "4,1";"4,3" 
%                  \ar|\bullet "4,3";"6,3" 
%                  \ar "4,3";"4,5" 
%                  \ar|\bullet "4,5";"4,7" 
%                  \ar|\bullet "4,5";"6,5" 
%                  \ar@{} "4,7";"5,8" |{\sei}%{\text{\large \ding{197}}}
%                  \ar "4,7";"3,8" 
%                  \ar|\bullet "4,7";"6,7" 
%                  \ar "5,8";"7,8" 
%                  \ar "6,3";"6,5" 
%                  \ar|\bullet "6,5";"6,7" 
%                  \ar "6,5";"8,5" 
%                  \ar "6,7";"8,7" 
%                  \ar|\bullet "6,7";"5,8" 
%                  \ar "8,5";"8,7" 
%                  \ar "8,7";"7,8" 
%                }}
% \] (the bulleted arrows are isomorphisms): now consider the commutative square \cin + \sei:
%\[
%	\xymatrix@C=1.5cm{
%	  R_0 i_R X	& R_0 i_R Y \\
%	  R_0 i_R R X	& R_0 i_R R Y
%	  \ar "1,1";"1,2" 
%	  \ar_\wr "1,1";"2,1" 
%	  \ar^\wr "1,2";"2,2" 
%	  \ar_{R_0 i_R R f}^\sim "2,1";"2,2" 
%	}
%\] which entails that $R_0 i_R f$ is an isomorphism, \ie that $i_R f\in \EE_0$. 
%
Then, from diagram (\ref{morphism-of-fiber-sequences}) we see that, since both $i_Lq_RR_1qf$ and $i_LRf$ are isomorphisms, so is also $R_0i_LWf$. 
\end{proof}
\begin{remark}
From the sub-diagram
\[ 
\xymatrix{
  i_L X	& i_L Y \\
  i_L KX	& i_L KY \\
  i_L RX	& i_L RY
  \ar@{} "1,1";"2,2" |{\uno}%{\text{\large \uno}}
  \ar "1,1";"1,2" ^{i_L f}
  \ar_\wr "1,1";"2,1" 
  \ar^\wr "1,2";"2,2" 
  \ar@{} "2,1";"3,2" |{\due}%{\text{\large \ding{193}}}
  \ar@{->>} "2,1";"2,2" 
  \ar@{->>} "2,1";"3,1" 
  \ar@{->>}@{->>} "2,2";"3,2" 
  \ar_\sim "3,1";"3,2" 
}
\] of diagram (\refbf{diag1bis}) one deduces that if $Rf$ is an isomorphism, then $i_L f\in \EE_0$, by the 3-for-2 closure property of $\EE_0$. By Propostion \ref{??} this mean that
$\{ q, i_LW\}(f)\in \EE$ implies that $\{ q, i_L\}(f)\in \EE$. The converse implication has no reason to be true in general. However it is true for terminal (or initial) morphisms. Namely, from the Rosetta stone one has that $X\in \EE_0\glue\EE_1$ if and only if $X\in (\D^0\glue \D^1)_{\ge 0}$, and so if and only if $\{q, i_L\}(X)\in \EE$. On the other hand, by Proposition \ref{???}, $X\in \EE_0\glue\EE_1$ if and only if $\{q, i_LW\}(X)\in \EE$. The fact that the condition $\{q, i_L\}(X)\in \D_{\geq 0}$ is equivalent to the condition $\{q, i_LW\}(X)\in \D_{\geq 0}$ can actually be easily checked directly. Namely, if $qX\in \D^1_{\geq 0}$, then $q_RR_1qX=0$ and so $X=WX$ in this case. Specular considerations apply to the right class $\MM_0\glue\MM_1$.
\end{remark}


{\color{red}Sposterei questa parte a subito dopo il Remark \ref{veryshort}}


