\chapter{Factorization Systems}\label{chap:effe.esse}
\thispagestyle{empty}
\section{Overview of factorization systems.}
\setlength{\epigraphwidth}{.75\textwidth}
\epigraph{It isn't that they can't see the solution.\\ It is that they can't see the problem.}{G.K. Chesterton}
This first chapter deals with the general definition of a \emph{factorization system} in the setting of $\infty$\hyp{}categories, as given by \cite{HTT} and \cite[\abbrv{p} \textbf{178}]{Joy}.

We do not claim originality here, aiming only at a balance between the creation of a flexible and natural formalism, to be used along the subsequent chapters, and the necessity of rigor and generality.

\index{Factorization system!history of}
The current literature seems to be too poor and too rich at the same time, when dealing with factorization systems; several authors often decide to rebuild the basic theory from scratch when they prove a new result, as there are several slightly different flavours in which one wants to interpret the basic idea behind the definition (that is, \emph{factor every arrow of a category into two distinguished pieces}). 

Since simple and extremely pervasive structures are often discovered independently, an accurate overview of the topic is somewhat impossible; we can however try to date back to the pioneering \cite{maclane1948groups}, published in 1948 (!) a forerunner to the modern notion of factorization system (see in particular axioms \textsc{bc}1--5).

With the passing of time, it became clear that together with unique factorization, the \emph{orthogonality} relation between the arrows of two classes $\EE,\MM\subseteq \hom(\CC)$ played an essential r\^ole in the definition of a factorization system.\footnote{However, the r\^ole of uniqueness is much more essential (see Remark \refbf{all.what.matters}): even if factorization of arrows with respect to a prefactorization is unique, a strictly unique factorization with respect to two classes \emph{implies} orthogonality between the classes; from time to time we will need to exploit this useful remark, first observed by Joyal and used in \cite{Joyalcatlab} as definition of a factorization system. In any case, assuming the orthogonality relation and the factorization property as primitive and unrelated properties is a common practice.}
Another forerunner of the modern theory is Isbell's \cite{Isb} (there, the author doesn't mention orthogonality, but clearly refers to what in the subsequent \cite{FK} will be called in this way); his work was first popularized in the lucid and methodical presentation of the latter paper, which together with \cite{CHK} has been a fundamental starting point, and a source of suggestions for the main result exploited throughout this work: \emph{reflective subcategories of a category $\CC$ originate from the calculus of factorization systems}. This will be the main theme developed in the following chapters, and will culminate in \achap \refbf{chap:tstruct} with the proof of our ``Rosetta stone'' theorem.
\section{Markings and prefactorizations.}
\epigraph{\greek{καὶ στήσει τὰ μὲν πρόβατα ἐκ δεξιῶν αὐτοῦ τὰ δὲ ἐρίφια ἐξ εὐωνύμων.}}{Matthew 25:33}
\begin{definition}[Marked simplicial set]\index{Simplicial set!marked ---}\index{Marking}
Recall that a \emph{marked simplicial set} $\underline X$ consists of a pair $(X, \mS)$, where $X$ is a simplicial set, and $\mS \subseteq X_1$ is a class of distinguished 1\hyp{}simplices on $X$, which contains every degenerate 1\hyp{}simplex.
\end{definition}
\begin{remark}
The class of all marked simplicial sets forms a category $\sSet^\marked$, where a morphism is a simplicial map $f\colon (X,\mS_X)\to (Y, \mS_Y)$ which \emph{respects} the markings, in the sense that $f (\mS_X)\subseteq \mS_Y$; the obvious forgetful functor
\[
U\colon \sSet^\marked \to \sSet
\]
admits both a right adjoint $X\mapsto X^\sharp =(X, X_1)$ and a left adjoint $X\mapsto X^\flat = (X, s_0(X_0))$, given by choosing the maximal and minimal markings\index{Marking!maximal and minimal}, respectively (mnemonic trick: \textbf{r}ight adjoint is sha\textbf{r}p, \textbf{l}eft adjoint is f\textbf{l}at).
\end{remark}
\begin{notat}\index{Quasicategory!marked ---}
A \emph{marked $\infty$\hyp{}category} simply consists of a marked simplicial set which, in addition, is an $\infty$\hyp{}category. From now on, we will consider only marked $\infty$\hyp{}categories, and occasionally confuse $X$ and $X^\flat$.
\end{notat}
\begin{definition}[Orthogonality]\label{def:orthog}\index{Orthogonality!--- of arrows}\index{Lifting problem}
Let $f,g$ be two edges in an $\infty$\hyp{}category $\CC$. We will say that $f$ is \emph{left orthogonal} to $g$ (or equivalently that $g$ is \emph{right orthogonal} to $f$) if in any commutative square $\Delta[1]\times\Delta[1]\to \CC$ like the following,
\[\label{a.problem}
\begin{kD}
\lattice[mesh]{
	\obj (a): A; & \obj (x): X; \\
	\obj (b): B; & \obj (y): Y; \\
};
\mor a -> x g:-> y <- b f:<- a;
\mor b \alpha:-> x;
\end{kD}
\]
the space of liftings $a$ rendering the two triangles (homotopy) commutative is contractible.\footnote{By requiring that the space of liftings $\alpha$ is only \emph{nonempty} one obtains the notion of weak orthogonality. In the following discussion we will only cope with the stronger request, but we rapidly address the issue in \adef \refbf{weak.ortho} and in the subsequent points.}
\end{definition}
\begin{remark}
This is \adef \cite[\textbf{5.2.8.1}]{HTT}; compare also the older \cite[\adef \textbf{3.1}]{JanelidzeMarkl}.
\end{remark}
\begin{remark}
\index{.perp@$\perp$}
``Being orthogonal'' defines a binary relation on the set of edges in a marked $\infty$\hyp{}category $\CC$, denoted $f\perp g$.
\end{remark}
\begin{notat}\index{Galois connection!--- of $\perp$}
We denote $\prescript{\perp}{}{(-)}\dashv (-)^\perp$ the (antitone) Galois connection induced by the relation $\perp$ on subsets $\mS\subseteq \CC_1$;\footnote{Recall that if $R\subseteq A\times B$ is a relation, it induces a Galois connection $${}^{R}(-) \colon \mcal{P}\,(A)\rightleftarrows \mcal{P}\,(B)\colon (-)^{R};$$ ``negative thinking'' tells us that this is simply the nerve\hyp{}realization adjunction generated by $R$ regarded as a $\Omega$\hyp{}profunctor ($\Omega$ can, but must not, be the Boolean algebra $\{0,1\}$). This remark has, however, little importance in the ongoing discussion, and only serves the purpose of using the word ``profunctor'' in the present thesis.} more explicitly, we denote
\begin{gather*}
\mS^\perp = \{f\colon \Delta[1]\to \CC\mid s \perp f,  \forall s\in \mS\} \\
\prescript{\perp}{}{\mS} = \{f\colon \Delta[1]\to \CC\mid f \perp s,  \forall s\in \mS\},
\end{gather*}
and we consider the adjunction $\prescript{\perp}{}{(-)}\colon \mcal{P}\,(\hom(\CC)\leftrightarrows \mcal{P}\,(\hom(\CC))\colon (-)^\perp$.
\end{notat}
\begin{definition}[Category of markings]\label{def:mrkc}\index{Marking!category of ---s}
If $\CC$ is a small $\infty$\hyp{}category we can define a poset $\mrk(\CC)$ whose objects are markings of $\CC$ and whose arrows are given by inclusions as subsets of $\CC_1$. 
\end{definition}
\begin{remark}\index{Marking!maximal and minimal ---}
The maximal and the minimal markings are, respectively, the terminal and initial object of $\mrk(\CC)$; this category can also be characterized as the fiber over $\CC$ of the forgetful functor $U\colon \sSet^\marked \to \sSet$. Moreover, the Galois connection $\prescript{\perp}{}{(-)}\dashv (-)^\perp$ defined above induces an analogous adjunction on $\mrk(\CC)$, via the obvious identification. 
\end{remark}
This remark leads to a second
\begin{remark}
The correspondence $\prescript{\perp}{}(-)\dashv (-)^\perp$ forms a Galois connection in the category of markings of $X$; the maximal marking, and the marking $\iso$ made by all isomorphisms in $\CC$ exchange each other under these correspondences. More precisely,
\end{remark}
\begin{proposition}\label{prop:iso.are.ortho.to.all}
The following conditions are equivalent, for $f\colon \Delta[1]\to \CC$:
\begin{enumerate}
\item $f$ is an isomorphism;
\item $f\in \CC_1^\perp$;
\item $f\in \prescript{\perp}{}{\CC_1}$;
\item $f\perp f$.
\end{enumerate}
\begin{remark}
The technique applied here (devise suitable lifting problems which, solved, prove the claim) is a standard trick in the calculus of factorization systems: we will often use arguments like the following. 
\end{remark}
\begin{proof}
This case is extremely simple and paradigmatic. It is evident that the implications $1 \Rightarrow 2 \Rightarrow 4$ and $1 \Rightarrow 3 \Rightarrow 4$ (the inverse of $f$, composed with the upper horizontal arrow of a lifting problem, does the trick); to close the circle of implications, it is enough to show that $4 \Rightarrow 1$: this is evident, since the solution to the lifting problem
\[
\begin{kD}
\lattice[mesh]{
	\obj (a): ; & \obj (x): ; \\
	\obj (b): ; & \obj (y): ; \\
};
\mor a - x f:-> y - b f:<- a;
\mor b \alpha:-> x;
\end{kD}
\]
(where horizontal arrows are identity maps) must be the inverse of $f$ (in $\infty$\hyp{}categories, there is a contractible space of such inverses, agreeing with \adef \refbf{def:orthog}).
\end{proof}
\end{proposition}
\begin{proposition}
There exists an adjunction  \[\underline{{}^\boxslash(-)}\dashv \underline{(-)^\boxslash}\colon \cate{QCat}/X^{\Delta[1]}\leftrightarrows\left(\cate{QCat}
/X^{\Delta[1]}\right)^\op 
\]
``lifting'' the Galois connection ${}^\boxslash(-)\dashv (-)^\boxslash\colon \mrk(X)\leftrightarrows \mrk(X)$.\end{proposition}
This can be seen as an $\infty$\hyp{}categorical version of \cite[Prop. \textbf{3.8}]{Garner2009a}.
\begin{remark}[Orthogonality and locality]\label{object.ortho}\index{Orthogonality!--- of objects}
There is another notion of orthogonality of an object $X$ with respect to a morphism $f\in\hom(\CC)$; given these data, we say that $X$ is \emph{right\hyp{}orthogonal} to $f$ (or that $X$ is an \emph{$f$\hyp{}local} object) if the hom functor $\hom(-,X)$ inverts $f$. 

If $\CC$ has a terminal object $*$, this notion is related to \adef \refbf{def:orthog}, in the sense that $X$ is right\hyp{}orthogonal to $f$ if and only if the terminal arrow $X\to *$ is right orthogonal to $f$. For this reason, we always refer to \emph{object\hyp{}orthogonality} as orthogonality with respect to terminal arrows. (Obviously, there is a dual notion of left\hyp{}object\hyp{}orthogonality between $f$ and $B\in\CC$, which in the presence of an initial object reduces to left orthogonality with respect to $\varnothing\to B$).
\end{remark}
\begin{notat}\label{orth.between.classes}\index{Orthogonality!--- of classes}
Extending this notation a little bit more, we can speak about orthogonality between two objects, without introducing new definitions: in a category with both a terminal and initial object (which, since our blanket assumption in all the remaining chapters is to work in a stable $\infty$\hyp{}category, will always be the case) we can say that 
\begin{itemize}
\item Two objects $B$ and $X$ are orthogonal if $\hom(B,X)$ is contractible; we denote this (non\hyp{}symmetric) relation as $B\perp X$.
\item Two classes of object $\mcal{H}$ and $\mcal{K}$ in $\CC$ are orthogonal if each object $H\in\mcal{H}$ is orthogonal to each object $K\in\mcal{K}$; we denote this situation by $\mcal{H} \perp \mcal{K}$.
\end{itemize}
This notation will greatly help us in \achap \refbf{chap:tstruct} and \refbf{chap:hearts}
\end{notat}
The following nomenclature is modeled on the analogous categorical notion of a \emph{prefactorization system} introduced in \cite{FK}.
\begin{definition}\index{Factorization!prefactorization system}
A pair of markings $(\EE,\MM)$ in an $\infty$\hyp{}category $\CC$ is said to be a \emph{($\infty$\hyp{}categorical) prefactorization} when $\EE = \prescript{\perp}{}{\MM}$ and  $\MM = \EE^\perp$. In the following we will denote a prefactorization on $\CC$ as $\fF=(\EE,\MM)$. 
\end{definition}
\begin{remark}\label{def:prefacts}
The collection of all prefactorizations on a given $\infty$\hyp{}category $\CC$ forms a poset, which we will call $\pf(\CC)$, with respect to the order $\fF=(\EE,\MM)\preceq \fF'= (\EE',\MM')$ iff $\MM\subset\MM'$ (or equivalently, $\EE'\subset \EE$).
\end{remark}
\begin{remark}
It is evident (as an easy consequence of adjunction identities) that any marking $\mS\in\text{Mrk}(\CC)$ induces two \emph{canonical} prefactorizations on $\CC$, obtained by sending $\mS$ to $(\prescript{\perp}{}{\mS}, (\prescript{\perp}{}{\mS})^\perp)$ and $(\prescript{\perp}{}{(\mS^\perp)},\mS^\perp)$. These two prefactorizations are denoted $\mathbb S_\perp$ e ${}_\perp\mathbb S$, respectively, and termed the \emph{right} and \emph{left prefactorization} associated to $\mS$.
\end{remark}
\begin{definition}\label{df:rlgener}\index{Factorization!left\fshyp{}right generated}
If a prefactorization $\fF$ on $\CC$ is such that there exists a marking $\mS\in\mrk(\CC)$ such that $\fF=\mathbb S_\perp$ (resp., $\fF={}_\perp\mathbb S$) then $\fF$ is said to be \emph{right} (resp., \emph{left}) \emph{generated} by $\mS$.
\begin{remark}
Since in a prefactorization system $\fF = (\EE,\MM)$ each class uniquely determines the other, the prefactorization is characterized by any of the two markings $\EE,\MM$ and the poset $\pf(\CC)$ defined in \refbf{def:prefacts} can be confused with a sub\hyp{}poset of $\mrk(\CC)$ defined in \refbf{def:mrkc}; accordingly with \refbf{def:prefacts} the class of all prefactorizations $\fF=(\EE,\MM)$ on an $\infty$\hyp{}category $\CC$ is a poset whose greatest and smallest elements are respectively 
\[
{}_\perp(\CC^\sharp)  = (\iso, \CC_1)\text{ and } (\CC^\sharp)_\perp = (\CC_1, \iso).\]
\end{remark}
\begin{definition}\label{space.of.facts}\index{Factorization!space of ---s}
If $f\colon\Delta[1]\to \CC$ is an arrow in $\CC$, a \emph{factorization} of $f$ is an element of the simplicial set $\text{Fact}(f)$ defined to be the upper\hyp{}left corner on the following pullback diagram
\[
\begin{kD}
\lattice[mesh={4em}{8em}]{
	\obj (fact):\text{Fact}(f); & \obj (C1xC1):\CC^{\Delta[1]}\times \CC^{\Delta[1]}; \\
	\obj (Delta0):\Delta[0]; & \obj (C1):\CC^{\Delta[1]}; \\
};
\mor fact -> C1xC1 -> C1;
\mor * -> Delta0 -> *;
\pullback{fact}{C1};
\end{kD}
\]
A factorization $\sigma\in \text{Fact}(f)$ will usually be denoted as $(u,v)$ and the sentence ``$(v,u)$ is a factorization of $f\colon A\to B$'' will be shortened into ``$f = u\circ v\colon A\to F\to B$'' for an object $F$ determined from time to time, and called the \emph{factor} of $F$.
\end{definition}
\begin{remark}
Notice that since $\CC$ is an $\infty$\hyp{}category, so is $\text{Fact}(f)$ for $f\colon A\to B$; its morphisms can be depicted as commutative squares
\[
\begin{kD}
\lattice[comb={4em}]{
	\obj (bul):F; & \obj B; \\
	\obj A; & \obj (bul'):F'; \\
};
\mor A -> bul -> B;
\mor * -> bul' -> *;
\mor bul \varphi:-> bul';
\end{kD}
\]
where $\varphi\colon F\to F'$ is a morphism between the factors, such that the two triangles are commutative.
Moreover (see the introduction), we will only consider factorizations which are \emph{functorial}, in the obvious sense of being given as functors out of the arrow category $\CC^{\Delta[1]}$.

This isn't too restrictive an assumption, since the factorization systems relevant to the present work are all functorial (see, in particular, the proof of our \refbf{thm:rosetta}).
\end{remark}
\begin{remark}\label{ortho.is.pull}
A useful characterization of orthogonality available in 1\hyp{}categorical world is the following: given $f\colon A\to B, g\colon X\to Y$, we have $f\perp g$ if and only if the following square
\[\label{ortho.via.pull}
\begin{kD}
\lattice[mesh={4em}{8em}]{
	\obj (BX):\hom(B,X); & \obj (BY):\hom(B,Y); \\
	\obj (AX):\hom(A,X); & \obj (AY):\hom(A,Y); \\
};
\mor BX -> BY -> AY <- AX <- BX;
\end{kD}
\]
is a pullback (the proof of this fact is immediate). This characterization can be used to define \emph{enriched} factorization systems (see \cite{Day1974,Lucyshyn-Wright} and our discussion in \S\refbf{model.dep}).

This characterization exports, \emph{mutatis mutandis}, to the $\infty$\hyp{}categorical setting: see \cite[\textbf{A.4.(41)}]{Aaron}
\end{remark}
\section{Factorization systems.}
A basic transition step from prefactorizations to factorizations is the following result, which is the analogue of the classical results about uniqueness of a factorization with respect to a prefactorization system:
\begin{remark}\label{rmk:uniqueness.of.fact}
If $f\colon \Delta[1]\to \CC$ is a morphism, and $\fF = (\EE,\MM)\in \pf(\CC)$ is a prefactorization system on the $\infty$\hyp{}category $\CC$, then the subspace $\text{Fact}_{\fF}(f)\subseteq \text{Fact}(f)$ of factorizations $(e,m)$ such that $e\in \EE, m\in \MM$, is a contractible simplicial set as soon as it is nonempty.
\end{remark}
\begin{proof}
It all boils down to solving the right lifting problem: if $f\colon A\to B$ can be factored in two ways $(e,m), (e', m') \in \text{Fact}_{\fF}(f)$, the first lifting problem gives ``comparison'' arrows $X\leftrightarrows X'$, and the other two (together with essential uniqueness of the factorization) entail that $u,v$ are mutually inverse.
\[
\begin{kD}
\lattice[mesh]{
\obj A; & \obj X'; & 			\obj (A2):A; & \obj X; & 		\obj (A3):A; & \obj (X'2):X'; \\
\obj (X2):X; & \obj B; &		\obj (X3):X; & \obj (B2):B; & 	\obj (X'3):X'; & \obj (B3):B; \\
};
\mor A e':-> X' v:dashed,shift=-1mm,-> X2 swap:m:-> B;
\mor A swap:e:-> X2;
\mor X2 u:dashed,shift=-1mm,-> X' m':-> B;
\mor A2 e:-> X m:-> B2 m:<- X3 e:<- A2; \mor X3 vu:dashed,-> X;
\mor A3 e':-> X'2 m':-> B3 m:<- X'3 e':<- A3; \mor X'3 uv:dashed,-> X'2;
\end{kD}
\]
\end{proof}
\begin{definition}[$\fF$\hyp{}crumbled morphisms]\label{def:crumble}\index{Morphism!crumbled}
Given a prefactorization $\fF\in\pf(\CC)$ we say that an arrow $f\colon X\to Y$ is \emph{$\fF$\hyp{}crumbled}, (or \emph{$(\EE,\MM)$\hyp{}crumbled} for $\fF=(\EE,\MM)$) when there exists a(n essentially unique, in view of Remark \refbf{rmk:uniqueness.of.fact}) factorization for $f$ as a composition $m\circ e$, with $e\in\EE$, $m\in\MM$; let $\sigma_\fF$ be the class of all $\fF$\hyp{}crumbled morphisms, and define
\[
\pf_{\mS}(\CC) = \{\fF\mid \sigma_\fF\supset\mS\}\subset \pf(\CC).
\] 
\end{definition}
\begin{definition}\label{def:effe.esse}\index{Factorization!factorization system}
A prefactorization system $\fF=(\EE,\MM)$ in $\pf(\CC)$ is said to be a \emph{($\infty$\hyp{}categorical) factorization system} on $\CC$ if $\sigma_\fF=\hom(\CC)$; factorization systems, identified with $\pf_{\hom(\CC)}(\CC)$, form a sub\hyp{}poset $\smallcap{fs}(\CC)\leq \pf(\CC)$.
\end{definition}
This last definition (factorizations ``crumble everything'', \ie split every arrow in two) justifies the form of a more intuitive presentation for a factorization system on $\CC$, modeled on the classical, 1\hyp{}categorical definition:
\begin{definition}[$\infty$\hyp{}categorical Factorization System]\label{def:effe.esse2}
Let $\CC$ be an $\infty$\hyp{}category; a \emph{factorization system} (\smallcap{fs} for short) $\fF$ on $\CC$ consists of a pair of markings $\mcal{E}, \mcal{M}\in\mrk(\CC)$ such that
\begin{enumerate}
\item For every morphism $h\colon X\to Z$ in $\CC$ we can find a factorization $X\xto{e} Y\xto{m} Z$, where $e\in\mcal{E}$ and $m\in\mcal{M}$; an evocative notation for this condition, which we sometimes adopt, is $\CC = \MM\circ \EE$;
\item $\EE ={}^{\perp}\MM$ and $\MM = \EE^{\perp}$.
\end{enumerate}
\end{definition}
It is useful to introduce the following alternative formalism to express the class of $\fF$\hyp{}crumbled morphisms:
\begin{definition}[The ``$\semicol$'' symbol]\label{fat.notation}
Let $\CC$ be an $\infty$\hyp{}category and $\mcal{A},\mcal{B}\subseteq \hom(\CC)$; we denote $\mcal{A}\semicol \mcal{B}$ the class of all $f\in\hom(\CC)$ such that there exists a factorization $f = b\circ a$ with $a\in\mcal{A}, b\in\mcal{B}$. 
\end{definition}
It is obvious that for each prefactorization $\fF = (\EE,\MM)$, $\sigma_{\fF} = \EE\semicol \MM$ and that a prefactorization $\fF = (\EE,\MM)$ is a factorization if and only if $\EE\semicol\MM = \hom(\CC)$.

The proof of the following lemma is an immediate consequence of \adef \refbf{fat.notation}, \refbf{def:orthog}:
\begin{lemma}
Let $\CC$ be an $\infty$\hyp{}category, and let $\mcal{A}, \mcal{B}, \mcal{C},\mcal{D} \in\mrk(\CC)$.
\begin{itemize}
\item If $\mcal{A}\perp \mcal{C}$ and $\mcal{A}\perp \mcal{D}$, then $\mcal{A}\perp (\mcal{C}\semicol \mcal{D})$;
\item If $\mcal{A}\perp \mcal{C}$ and $\mcal{B}\perp \mcal{C}$, then $(\mcal{A}\semicol \mcal{B})\perp \mcal{C}$.
\end{itemize}
\end{lemma}
\begin{remark}\index{.fs@$\smallcap{fs}(\CC)$}
The collection of all factorization systems on an $\infty$\hyp{}category $\CC$ form a poset $\smallcap{fs}(\CC)$ with respect to the partial order induced by $\pf(\CC)$.
\end{remark}
\begin{remark}
In the presence of condition (1) of Definition \refbf{def:effe.esse}, the second condition may be replaced by
\begin{enumerate}
\item [(2a)] $\EE \perp \MM$ (namely $\EE\subset \prescript{\perp}{}{\MM}$ and $\MM\subset \EE^\perp$);
    \item [(2b)]  $\EE$ and $\MM$ are closed under
    isomorphisms in $\CC^{\Delta[1]}$.
\end{enumerate}
Notice that this is precisely \cite[Def. \textbf{5.2.8.8}]{HTT}.
\end{remark}
\begin{remark}\label{uniquely.determines.class}
Condition (2) of the previous Definition (or the equivalent pair of conditions (2a), (2b)) entails that each of the two classes $(\EE,\MM)$ in a factorization system on $\CC$ uniquely determines the other (compare the analogous statement about prefactorizations): this is \cite[Remark \textbf{5.2.8.12}]{HTT}.
\end{remark}
\end{definition}
It is often of great interest to determine whether a given class right\hyp{}generates or left\hyp{}generates (\adef \refbf{df:rlgener}) a factorization system and not only a prefactorization: a general procedure to solve this problem is to invoke the \emph{small object argument}.
\begin{theorem}[Small Object Argument]
Let $\CC$ be an $\infty$\hyp{}category, and $\mcal{J}\in\mrk(\CC)$. If for each $f\colon I\to J$ in $\mcal{J}$ the functor $\hom(I,-)$ commutes with filtered colimits, then $\mcal{J}_\perp$ is a factorization system on $\CC$.
\end{theorem}
\begin{remark}\label{ortho.are.ortho}
Let $\CC$ be an $\infty$\hyp{}category with initial object $\varnothing$. If a class $\mcal{K}$ is generated via the small object argument by a small set then $0/\mcal{K}$ is generated as the object\hyp{}orthogonal of a small set.
\end{remark}
\subsection{Weak factorization systems.}
We will not make use of the content of this subsection, as our main results unavoidably need uniqueness for solutions of lifting problems and factorizations; we only record this definition for the sake of completeness.

There is a more general notion of \emph{weak} factorization system on an $\infty$\hyp{}category, again modeled on the 1\hyp{}categorical notion. This ``weakness'' shows up in two respects:
\begin{itemize}
\item Orthogonality is no longer strong: in $\infty$\hyp{}categories, this means that the space of solution is no longer contractible, but only \emph{nonempty}.
\item Factorization is no longer unique up to a unique equivalence: this means that there are possibly many different connected components on the space of factorizations.
\end{itemize}
A 1\hyp{}dimensional example of such a structure is the pair of classes $(\smallcap{Mono}, \smallcap{Epi})$ in the category $\Set$ of sets and functions; there seems to be no mention of this condition in the world of $\infty$\hyp{}categories, in \cite{HTT} or \cite{Joy} (this could possibly be related to the fact that the concept of ``monomorphism'' does not naturally belong to the world of $\infty$\hyp{}categories, see also \refbf{no.easy.examples} below). 

However, a tentative definition of a \emph{model $\infty$\hyp{}category} has been given in \cite{Aaron}, with applications to Goerss\hyp{}Hopkins obstruction theory. \cite[\S\textbf{2}]{Aaron} contains plenty of examples of weak factorization systems in $\infty$\hyp{}categories.
\begin{definition}[Weak orthogonality]\label{weak.ortho}\index{Orthogonality!weak}
Let $f,g$ be two edges in an $\infty$\hyp{}category $\CC$. We will say that $f$ is \emph{weakly left orthogonal} to $g$ (or equivalently that $g$ is \emph{weakly right orthogonal} to $f$) if in any lifting problem like (\refbf{a.problem}) 
\index{.pitch@$\pitchfork$}
the space of solutions $a$ is nonempty. We denote this binary relation as $f\pitchfork g$, and the resulting Galois connection ${}^\pitchfork(-)\dashv (-)^\pitchfork$.
\end{definition}
Now, a \emph{weak prefactorization system} is a pair of classes $\fF = (\EE,\MM)\subseteq \hom(\CC)\times \hom(\CC)$ such that $\EE = {}^\pitchfork\MM$, $\MM = \EE^\pitchfork$. \index{Galois connection! of $\pitchfork$}\index{Factorization!weak factorization system}A \emph{weak factorization system} is a weak prefactorization system such that every arrow is $\fF$\hyp{}crumbled (\adef \refbf{def:crumble}).

Weak factorization systems are organized in a poset $\smallcap{wfs}(\CC)$ which contains as a sub\hyp{}poset $\fs(\CC)$ of \adef \refbf{def:mrkc}.

It is possible to relate weak orthogonality to strong orthogonality, and give conditions ensuring that a weak factorization system is indeed strong: see \cite[\S\textbf{1}]{RT} for further details, and consult \cite{Aaron} for further information about model $\infty$\hyp{}categories.
\begin{example}\label{df:infty.model.cat}\index{model-infty@Model $\infty$\hyp{}category}
A \emph{Quillen model structure} on a small\hyp{}bicomplete $\infty$\hyp{}category $\CC$ is defined by three markings $(\mcal{Wk},\mcal{Fib},\mcal{Cof})$ such that
\begin{itemize}
\item $\mcal{Wk}$ is a 3\hyp{}for\hyp{}2 class (\adef \refbf{def:3for2}) containing all isomorphisms and closed under retracts;
\item The markings $(\mcal{Cof},\mcal{Wk}\cap \mcal{Fib})$ and $(\mcal{Wk}\cap \mcal{Cof},\mcal{Fib})$ both form a weak factorization system on $\CC$.
\end{itemize} 
\end{example}
\section{Closure properties.}\label{closure.props}
\begin{definition}[Closure operators associated to markings]\label{def:satusatu}
Let $\CC$ be an $\infty$\hyp{}category. A marking $\mcal{J}\in\mrk(\CC)$ is called 
\begin{itemize}
\item[\textsc{w}.)] \index{Marking!wide ---} \textbf{wide} if it contains all the isomorphisms and it is closed under composition;
\end{itemize}
A wide marking $\mcal{J}$ (in an $\infty$\hyp{}category $\CC$ which admits in each case the $\infty$\hyp{}categorical co\fshyp{}limits needed to state the definition) is called
\begin{itemize}
\item[\textsc{p}.)] \index{Marking!presaturated ---}\textbf{presaturated} if is closed under \emph{co\hyp{}base change}, \ie whenever we are given arrows $j\in\mcal{J}$, and $h$ such that we can form the pushout
\[
\begin{kD}
\lattice[mesh]{
	\obj A; & \obj X; \\
	\obj B; & \obj Y; \\
};
\mor A h:-> X j':-> Y swap::<- B j:<- A;
\pushout{A}{Y};
\end{kD}
\]
then the arrow $j'$ is in $\mcal{J}$;
\item[\textsc{q}.)] \index{Marking!almost saturated ---}\textbf{almost saturated} if it is presaturated and closed under \emph{retracts} (in the category $\CC^{\Delta[1]}$), \ie whenever we are given a diagram like
\[
\begin{kD}
\lattice[mesh]{
	\obj A; & \obj A'; & \obj (A2):A; \\
	\obj B; & \obj B'; & \obj (B2):B; \\
};
\mor A i:-> A' r:-> A2;
\mor B i':-> B' -> B2;
\mor A u:-> B;
\mor A' v:-> B';
\mor A2 u:-> B2;
\end{kD}
\]
where $ri=\id_A$ and $r'i'=\id_B$, if $v$ lies in $\mcal{J}$, then the same is true for $u$;
\item[\textsc{c}.)] \index{Marking!cellular ---}\textbf{cellular} if it is presaturated and closed under \emph{transfinite composition}, namely whenever we have a cocontinuous functor $F\colon \alpha\to \mcal{J}$\footnote{This notation is a shorthand to denote the fact that each edge $F(i \le j)\colon F(i)\to F(j)$ is an element of $\mcal{J}$; alternatively, we can regard this notation as consistent, via the obvious identification between markings on $\CC$ and full subcategories of $\CC$.} defined from a limit ordinal $\alpha$ admits a composite in $\mcal{J}$, \ie the canonical arrow
\[
\xymatrix@C=1mm{
F(0) \ar[r] & **[r] F(\alpha) = \varinjlim_{i<\alpha}F(i)
}
\]
lies in $\mcal{J}$;
\item[\textsc{s}.)] \index{Marking!saturated ---}\textbf{saturated} if it is almost saturated and cellular.
\end{itemize} 
All these properties induce suitable closure operators, encoded as suitable (idempotent) monads on $\mrk(\CC)$, defined for any property $\textsf{p}$ among $\{ \textsc{w}, \textsc{p}, \textsc{q}, \textsc{c}, \textsc{s} \}$ as 
\[
\textsf{p}(-) \colon 
\mS \mapsto \textsf{p}(\mS) = \bigcap_{\mcal{U}\supseteq \mS}\Big\{ \mcal{U}\in \mrk(\CC)\mid \mcal{U} \text{ $P$ property has } \Big\}\]
In classical category theory, the \emph{cellularization} $\textsf{c}(-)$ and the \emph{saturation} $\textsf{s}(-)$ of a marking $\mcal{J}$ on $\CC$ are of particular interest (especially in homotopical algebra), in view of what we state in \aprop \refbf{satu}.
\begin{remark}
A little more generality is gained by supposing that the cardinality of the coproducts or the transfinite compositions in $\CC$ is bounded by some (regular) cardinal $\alpha$. In this case we speak of $\alpha$\hyp{}saturated or $\alpha$\hyp{}cellular classes, and define the closure operators of $\alpha$\hyp{}\emph{cellularization} and $\alpha$\hyp{}\emph{saturation}, etc.
\end{remark}
\end{definition}
The following Proposition is a standard result in the theory of factorization systems, which we will often need throughout the discussion; a proof of the 1\hyp{}categorical version of the statement can be found in any of the references about factorization systems provided in the bibliography.
\begin{proposition}\label{satu}
Let $(\CC,\mS)$ be a marking of a cocomplete $\infty$\hyp{}category $\CC$; then the marking $\prescript{\perp}{}{\mS}$ of $\CC$ is a saturated class. In particular, the left class of a weak factorization system in a cocomplete $\infty$\hyp{}category is saturated.
\end{proposition}
Completely dual definitions give rise to co\hyp{}$\textsf{p}$\hyp{}classes.\footnote{Obviously, wideness and closure under retracts are auto\hyp{}dual properties. The definition of \emph{transfinite op\hyp{}composition} needed to define co\hyp{}cellularity may be difficult to guess; see \cite{Joy} for reference.} again, suitable monads acting as co\hyp{}$\textsf{p}$\hyp{}closure operators are defined on $\mrk(\CC)$, giving the dual of Proposition \refbf{satu}:
\begin{proposition}
Let $(\CC,\mS)$ be a marking of a cocomplete $\infty$\hyp{}category $\CC$; then the marking $\mS^\perp$ of $\CC$ is a co\hyp{}saturated class. In particular, the right class of a weak factorization system in a complete category is co\hyp{}saturated.
\end{proposition}
\begin{proposition}\label{thereiso}
Let $\CC$ be an $\infty$\hyp{}category and $\fF =(\EE,\MM)\in \fs(\CC)$; then $\EE\cap \MM$ equals the class of all equivalences in $\CC$.
\end{proposition}
\begin{proof}
Again, the proof in a 1\hyp{}category case can be found in any reference about factorization systems. The idea is extremely simple: if $g\in\EE\cap \MM$ then it is orthogonal to itself, and we can invoke \refbf{prop:iso.are.ortho.to.all}.
\end{proof}
\begin{definition}\label{def:3for2}\index{.l32@$\smallcap{l32}$, $\smallcap{r32}$}\index{.r32@$\smallcap{r32}$, $\smallcap{l32}$}
Let $\mS\in\mrk(\CC)$; then for each 2\hyp{}simplex in $\CC$ representing a composable pair of arrows, whose edges are labeled $f,g$, and  $f g$  we say that
\begin{itemize}
\item $\mS$ is \smallcap{l32} if $f, fg\in \mS$ imply $g\in\mS$;
\item $\mS$ is \smallcap{r32} if $fg, g\in \mS$ imply $f\in\mS$.
\end{itemize}
A marking $\mS$ which is closed under composition and both \smallcap{l32} and \smallcap{r32} is said to \emph{satisfy the 3\hyp{}for\hyp{}2 property}, or a \emph{3\hyp{}for\hyp{}2 class}.
\end{definition}
\begin{proposition}
\label{prop:clos}
Given a \smallcap{fs} $(\mcal{E},\mcal{M})$ in the $\infty$\hyp{}category $\CC$, then
\begin{enumerate}
\item \index{Colimit!closure of $\EE/\MM$ under ---}If the $\infty$\hyp{}category $\CC$ has $K$\hyp{}colimits, for $K$ a given simplicial set, then the full subcategory of $\CC^{\Delta[1]}$ spanned by $\mcal{E}$ has $K$\hyp{}colimits; dually, if  the $\infty$\hyp{}category $\CC$ has $K$\hyp{}limits, then the full subcategory of $\CC^{\Delta[1]}$ spanned by $\mcal{M}$ has $K$\hyp{}limits;
\item The class $\mcal{E}$ is \smallcap{r32}, and the class $\mcal{M}$ is \smallcap{l32} (see Def. \refbf{def:3for2}).
\end{enumerate} 
\end{proposition}
\begin{proof}
Point (1) is \cite[Prop. \textbf{5.2.8.6}]{HTT}; point (2) is easy to prove for 1\hyp{}categories, and then the translation to the $\infty$\hyp{}categorical setting is straightforward.\footnote{This translation process being often straightforward, here and everywhere a bibliographic support is needed, we choose to rely on classical sources to prove most of the result involving $\infty$\hyp{}categorical factorization systems. This should cause no harm to the reader.}
\end{proof}
It is a remarkable, and rather useful result, that each of the properties (1) and (2) of the above Proposition characterizes factorizations among weak factorizations: see \cite[Prop. \textbf{2.3}]{RT} for more details.

We close this section with an useful observation, showing that in \adef \refbf{def:effe.esse2} ``factorization is all what matters'': asking two classes $(\EE,\MM)$ to uniquely crumble every morphism $f\in\hom(\CC)$ entails that $(\EE,\MM)$ are mutually orthogonal classes.
\begin{remark}\label{all.what.matters}
Let $\fF=(\EE,\MM)$ be a pair of wide markings such that every $f\in\hom(\CC)$ has a unique factorization $f=m\circ e$ with $m\in\MM$, $e\in\EE$; then, $\fF$ is a prefactorization, \ie $\EE\perp\MM$.
\end{remark}
\begin{proof}
Given a lifting problem 
\[
\begin{kD}
\lattice[mesh]{
	\obj A; & \obj X; \\
	\obj B; & \obj Y; \\
};
\mor A u:-> X m'':-> Y;
\mor[swap] A e'':-> B v:-> Y;
\end{kD}
\]
we can factor $u$ as $m\circ e$ and $v$ as $m' \circ e'$, so that the square becomes
\[
\begin{kD}
\lattice[mesh]{
	\obj A; & \obj U; & \obj X; \\
	\obj B; & \obj V; & \obj Y; \\
};
\mor A e:-> U m:-> X m'':-> Y;
\mor[swap] A e':-> B e':-> V m':-> Y;
\end{kD}
\]
Now, the factorizations $(m''\circ m, e)$ and $(m',e'\circ e'')$ must be isomorphic by the uniqueness assumption, so that there exists an isomorphism $U\cong V$
 which composed with $e', m$ gives a solution to the lifting problem.
\end{proof}
\section{A second glance at factorization.}\label{sec:2ndglance} 
\epigraph{\japanese{一条の矢は折るべく十条の矢は折るべからず}}{Japanese proverb}
We add here a different presentation of $\infty$\hyp{}categorical factorization systems, faithfully following \cite[pp. 178---]{Joy}.
\begin{definition}[Orthogonality and Fillers]\label{def:joyortho}\index{Filler|see {Orthogonality}}
Let $\CC$ be an $\infty$\hyp{}category, and $u\colon A\to B, f\colon X\to Y$ two edges of $\CC$. We define the space $\text{Sq}(u,f)$ of commutative squares associated to $(u,f)$ to be the space of simplicial maps $s\colon \Delta[1]\times\Delta[1]\to \CC$ such that $s|_{\Delta^0\times \Delta[1]}=u, s|_{\Delta[1]\times\Delta^0}=f$.

A \emph{diagonal filler} for $s\in\text{Sq}(u,f)$ consists of an extension $\bar s\colon \Delta[1]\star\Delta[1]\to \CC$ (where $\star$ denotes the \emph{join} of simplicial sets, see \cite[\S\textbf{3.1} and \textbf{3.2}]{Joy}) of $s$ along the natural inclusion $\Delta[1]\times\Delta[1]\subset\Delta[1]\star \Delta[1]$.\end{definition}
\begin{remark}
Denote by $\text{Fill}(s)$ the top\hyp{}left corner of the fiber sequence
\[
\begin{kD}
\lattice[mesh={4em}{7em}]{
	\obj (fill):\text{Fill}(s); & \obj (X^1s1):X^{\Delta[1]\star\Delta[1]}; \\
	\obj (d0):\Delta[0]; & \obj (X^1x1):X^{\Delta[1]\times\Delta[1]}.;\\
};
\mor fill -> X^1s1 q:-> X^1x1;
\mor * -> d0 swap:s:-> *;
\pullback{fill}{X^1x1};
\end{kD}
\]
The simplicial set $\text{Fill}(s)$ is a Kan complex, since $q$ is a Kan fibration (as a consequence of \cite[Prop. \textbf{2.18}]{Joy}).
\end{remark}
This leads us to the following
\begin{definition}\index{Orthogonality!fillers}
\index{.boxslash@$\boxslash$}
We say that the edge $u$ is \emph{left orthogonal} to the edge $f$ in the $\infty$\hyp{}category $\CC$ (or $f$ is \emph{right orthogonal} to $u$) if $\text{Fill}(s)$ is a \emph{contractible} Kan complex for any $s\in \text{Sq}(u,f)$. We denote this relation between $u$ and $f$ as $u\boxslash f$.
\end{definition}
A first and natural task is to prove that the two relations $\perp$ and $\boxslash$ defined on the set of edges $\CC_1$ of an $\infty$\hyp{}category coincide: this is immediate since $\Delta[1]\star\Delta[1] = \Delta[3]$ (\cite[p\@. 244]{Joy}) and since ``solved commutative squares'' can be identified with simplicial maps $\Delta[3]\to \CC$ (there is a unique edge outside the image of $\Delta[1]\times\Delta[1]\subset\Delta[1]\star \Delta[1]$; this is the solution to the lifting problem).

Given this, for the rest of the section we will stick to the notation $f\perp g$ to denote orthogonality in this sense (\cite{Joy} uses the same symbol and takes \refbf{def:joyortho} as a definition).
\begin{proposition}
Factorization systems can be lifted along left or right fibrations: if $p\colon \CC\to \D$ is such a simplicial map, and $(\EE,\MM) \in \fs(\D)$, then $(p^\leftarrow(\EE), p^\leftarrow(\MM))$ is a factorization system on $\CC$.
\end{proposition}
\begin{corollary}
As a consequence, since the simplicial maps $\CC_{/X}\to \CC$ and $\CC_{Y/}\to\CC$ are left\fshyp{}right fibrations, every factorization system on $\CC$ \emph{lifts} to a factorization system on the slice\fshyp{}coslice $\infty$\hyp{}category. This is the $\infty$\hyp{}categorical version of the classical statement saying that a factorization system on $\CC$ induces factorization systems on all co\fshyp{}slice categories $\CC_{/X}$ and $\CC_{Y/}$.
\end{corollary}
\subsection{A factor-y of examples.}
\cite[pp\@. 178---]{Joy} is an invaluable source of examples for factorization systems on $\infty$\hyp{}categories; a standard technique to provide such examples is to reduce suitable ``niceness'' properties for categories (like regularity or exactness, or the possibility to find ``Postnikov towers'' for morphisms) to the presence of suitable factorization systems on it.

This general tenet remains valid in an $\infty$\hyp{}categories.
\begin{remark}\label{no.easy.examples}
We must observe, here, that it is rather difficult (\ie rather more difficult than in 1\hyp{}categories) to produce intuitive examples of factorization systems in $\infty$\hyp{}category, as many of the 1\hyp{}dimensional examples rely on the intuition that $(\smallcap{Epi}, \smallcap{Mono})$ is a well\hyp{}behaved and paradigmatic example of such a structure in many categories (such as sets, toposes, abelian categories\dots: the factorization systems such that every $\EE$ is a epimorphism, and every $\MM$ is a monomorphism are called \emph{proper} in \cite{FK,kelly1980unified} to suggest how this notion is common and familiar).

This cannot be achieved in $\infty$\hyp{}category theory, as \cite[p. 562]{HTT} conveys the intuition that the notion of monomorphism is not as meaningful in $\infty$\hyp{}category theory as it is in 1\hyp{}category theory (compare, however, the statement that every topos has an $(\smallcap{Epi}, \smallcap{Mono})$\hyp{}factorization system with the existence of a ``Postnikov'' factorization system on each $\infty$\hyp{}topos, \cite{HTT}).
\end{remark}
Several construction can be performed inside the category of categories with factorization system: these are classical definitions that can be recovered in every text about factorization systems (especially those with an interest towards model categories). 
\begin{example}[Co\fshyp{}products, co\fshyp{}slices]\label{prod.of.fact}
Let $\CC$ be an $\infty$\hyp{}category; then every co\fshyp{}slice of $\CC$ inherits a factorization system from $\fF = (\EE,\MM)\in\fs(\CC)$ obtained by putting
\begin{gather}
\EE_{/X} = \{(Y,f) \xto{\varphi}(Z,g)\mid \varphi \in \EE \}\notag\\
\MM_{/X} = \{(Y,f)\xto{\psi}(Z,g)\mid \psi \in \MM \}
\end{gather}
(the definition for coslices $\CC_{X/}$ is analogous).

Let $\{\CC_i\}$ be any small family of $\infty$\hyp{}categories; the product $\prod \CC_i$ of all the elements of the family inherits a factorization system from a family $\fF_i = (\EE_i,\MM_i)\in\fs(\CC_i)$, defined by putting
\begin{gather}
\prod \EE_i = \{ (f_i)_{i\in I}\mid f_i\in\EE_i\; \forall i\in I \}\notag\\
\prod \MM_i = \{ (g_i)_{i\in I}\mid g_i\in\MM_i\; \forall i\in I \}.
\end{gather}
A similar definition works for coproducts: the coproduct $\coprod \CC_i$ inherits a factorization system defined in a dual fashion (an arrow $f\in\coprod \CC_i$ lies in one and only one $\CC_{i^*}$; $f\in\coprod \EE$ if and only if $f\in \EE_{i^*}$).
\end{example}
\begin{example}[Surjection\hyp{}mono factorizations]\label{esem:monic}\index{Factorization system!surj,mono@$(\smallcap{Surj}, \smallcap{Mono})$}
We say that an arrow $f\colon X\to Y$ in an $\infty$\hyp{}category is \emph{monic} if the square
\[
\xymatrix{X \ar@{=}[r]\ar@{=}[d]& X \ar[d]^f \\ X \ar[r]_f & Y}
\] 
is cartesian. The class of monic arrows in $\CC$ is collected in a marking $\textsc{Mono}(\CC)=\textsc{Mono}$.

The class of \emph{surjective} arrows is defined to be the class $\prescript{\perp}{}{\textsc{Mono}(\CC)}$; we say that the $\infty$\hyp{}category $\CC$ \emph{has a surjection\hyp{}mono factorization} if the prefactorization $(\prescript{\perp}{}{\textsc{Mono}(\CC)}, \textsc{Mono}(\CC))$ is also a factorization system.
\end{example}
\begin{definition}[Regular $\infty$\hyp{}category]\index{Quasicategory!regular ---|see {$(\smallcap{Surj}, \smallcap{Mono})$}}
A finitely complete $\infty$\hyp{}category is said to be \emph{regular} if it admits a \emph{pullback\hyp{}stable} surjection\hyp{}mono factorization system; the coherent nerve of the category of Kan complexes, as a full sub\hyp{}$\infty$\hyp{}category of the nerve of the whole $\cate{sSet}$, is regular. Notice that this is the $\infty$\hyp{}categorical counterpart of \emph{Barr\hyp{}regular categories}.
\end{definition}
\subsection{Chains of factorization systems.}
\begin{definition}[$k$\hyp{}ary factorization system]\label{mult.fs}\index{Factorization system!multiple}
Let $k \ge 2$ be a natural number. A \emph{$k$\hyp{}fold factorization system} on a category $\CC$ consists of a monotone map $\phi\colon \Delta[k-2]\to \textsc{fs}(\CC)$, where the codomain has the partial order of \adef \refbf{def:prefacts}; denoting $\phi(i)=\fF_i$, a $k$\hyp{}fold factorization system on $\CC$ consists of a chain
\[
\fF_1 \preceq \dots \preceq \fF_{k-1},
\]
This means that if we denote $\fF_i = (\EE_i, \MM_i)$ we have two chains --any of which determines the other-- in $\hom(\CC)$:
\begin{gather*}
\EE_1\supset \EE_2 \supset\dots\supset \EE_{k-1},\\
\MM_1\subset \MM_2 \subset\dots\subset \MM_{k-1}.
\end{gather*}
\end{definition}
The definition of a $k$\hyp{}ary factorization system is motivated by the fact that a chain in $\textsc{fs}(\CC)$ results in a way to factor each arrow ``coherently'' as the composition of $k$ pieces, coherently belonging to the various classes of arrows. This is explained by the following simple result:
\begin{lemma}\label{multiple.fact}
Every arrow $f\colon A\to B$ in a category endowed with a $k$\hyp{}ary factorization system $\fF_1\preceq\dots\preceq \fF_{k-1}$ can be uniquely factored into a composition
\[
A \xto{\EE_1} X_1 \xto{\EE_2\cap \MM_1} X_2\to\dots\to X_{k-2} \xto{\EE_{k-1}\cap \MM_{k-2}} X_{k-1} \xto{\MM_{k-1}} B,
\]
where each arrow is decorated with the class it belongs to.
\end{lemma}
\begin{proof}
For $k=1$ this is the definition of factorization system: given $f\colon X\to Y$, we have its $\mathbb{F}_{i_1}$\hyp{}factorization
\[
X \xto{\EE_{i_1}} Z_{i_1}  \xto{\MM_{i_{1}}} Y.
\]
Then we work inductively on $k$. Given an arrow $f\colon X\to Y$ we first consider its $\mathbb{F}_{i_k}$\hyp{}factorization
\[
X \xto{\EE_{i_k}} Z_{i_k}  \xto{\MM_{i_{k}}} Y,
\]
and then observe that the chain $i_1\leq\cdots\leq i_{k-1}$ induces a $(k-1)$\hyp{}ary factorization system on $\CC$, which we can use to decompose $Z_{i_k}\to Y$ as
\[
Z_{i_k} \xto{\EE_{i_{k-1}}} Z_{i_{k-1}} \xto{\EE_{i_{k-2}}\cap \MM_{i_{k-1}}} Z_{i_{k-2}}\to\dots\to Z_{i_{2}}\xto{\EE_{i_{1}}\cap \MM_{i_{2}}} Z_{i_{1}} \xto{\MM_{i_{1}}} Y,
\]
and we are only left to prove that $Z_{i_{k}} \to Z_{i_{k-1}}$ is actually in $\EE_{i_{k-1}}\cap \MM_{i_{k}}$. This is an immediate consequence of the left cancellation property for the class $\MM_{i_{1}}$. Namely, since $\MM_{i_1}\subseteq \MM_{i_2} \subseteq\dots\subseteq \MM_{i_k}$, and $ \MM_{i_k}$ is closed for composition, the morphism $Z_{i_{k-1}}\to Y$ is in $\MM_{i_k}$. Then the \smallcap{l32} property applied to 
\[
Z_{i_{k}}\to Z_{i_k-1}\xto{\MM_{i_k}} Y
\]
concludes the proof.
\end{proof}
\subsubsection{The transfinite case.}\label{transfinite.case}\index{Factorization system!transfinite|see {multiple}} We are now interested to refine the previous theory in order to deal with possibly infinite chains of factorization systems. From a conceptual point of view, it seems natural how to extend the former definition to an infinite ordinal $\alpha$; it must consists on a ``suitable'' functor $F\colon \alpha\to \textsc{fs}(\CC)$. 

The problem is that suitable necessary co\fshyp{}continuity assumptions for such a $F$ might be covered by the fact that its domain is finite (and in particular admits an initial and a terminal object): in principle, dealing with infinite quantities could force such $F$ to fulfill some other properties in order to preserve the basic intuition behind factorization.

We start, now, by recalling a number of properties motivating \adef \refbf{mult.fs} below.
\begin{notat}
A factorization system on $\CC$ naturally defines a pair of pointed\fshyp{}co\hyp{}pointed endofunctors on $\CC^{\Delta[1]}$, starting from the factorization
\[
\begin{kD}
\lattice[goldcomb]{
	\obj X; & \obj Y; \\
	& \obj F(f).; \\
};
\mor X -> Y;
\mor[swap] * {\overleftarrow{F}(f)}:-> {Ff} {\overrightarrow{F}(f)}:-> Y;
\end{kD}
\]
(This has also been noticed in \cite{HTT}). 
\index{Factorization system!--- as monads}
A refinement of this  notion (see \cite{grandis2006natural, Gar, riehl2011algebraic}) regards this pair of functors as monad\fshyp{}comonad on $\CC^{\Delta[1]}$: in this case $F\colon \CC^{\Delta[1]}\to\CC$ is a functor and $f\mapsto \overleftarrow{F}(f)$ has the structure of a (idempotent) comonad, whose comultiplication is
\[
\xymatrix@C=1.4cm{
Ff \ar[r]^{\overleftarrow{F}(\overrightarrow{F}(f))}\ar[d]_{\overrightarrow{F}(f)} & FFf \ar[d]^{\overrightarrow{F}(\overrightarrow{F}(f))}\\
Y \ar@{=}[r] &Y
}
\]
and $f\mapsto \overrightarrow{F}(f)$ has the structure of a (idempotent) monad, whose multiplication is
\[
\xymatrix@C=1.4cm{
X \ar@{=}[r]\ar[d]_{\overleftarrow{F}(\overleftarrow{F}(f))} & Ff\ar[d]^{\overleftarrow{F}(f)}\\
FFf \ar[r]_{{\overrightarrow{F}(\overleftarrow{F}(f))}}& Ff.
}
\]
\end{notat}
\begin{remark}[On functors to posetal categories]\index{Category!posetal ---}
(Small) posets form the category $\cate{PCat}$ of (small) posetal categories (categories where every hom\hyp{}set is either empty or has one element).

This category is reflective in $\cate{Cat}$, since we have an adjunction
\[
\cate{PCat}(p\CC, P)\cong \cate{Cat}(\CC,P).
\]
(The poset $p(\cate J)$ results as the partially ordered set $\text{Ob}(\cate J)$ where $A\le B$ iff there is an arrow from $A$ to $B$.) Hence functors $\cate{J}\to P$ are uniquely determined by a monotone function $p(\cate J)=J\to P$, with respect to this order on $J$.
\end{remark}
\begin{remark}[On (co)limits in slice categories]\index{Colimit!in slices}
Slice and coslice categories $\CC_{X/}$ and $\CC_{X/}$ are complete and cocomplete whenever $\CC$ is: colimits in $\CC_{X/}$ and limits $\CC_{X/}$ are simply reflected by the natural forgetful functor $U\colon \CC_{X/},\CC_{X/}\to \CC$, so that the limit of a diagram $j\mapsto \var{X}{A_j}$ is simply the arrow $\var{X}{\varprojlim_j^\CC A_j}$ (and dually for colimits in $\CC_{X/}$); limits in a slice category, and colimits in a coslice category are, generally, more difficult to compute.

The general recipe for (say) colimits of a functor $F\colon \cate{J}\to \CC_{X/}$ exploits the isomorphism
\[
\text{Fun}(\cate{J}, \CC_{X/})\cong \text{Fun}_F(\cate{J}^\lhd, \CC)
\]
where $\cate{J}^\lhd$ is the category $[0]\star \cate{J}$ obtained freely adding an initial object, and $\text{Fun}_F(\cate{J}^\lhd, \CC)$ is the category of functors $\cate{J}^\lhd\to\CC$ which coincide with $F$ when restricted to $\cate{J}\subset [0]\star \cate{J}$.

Now fortunately, whenever the indexing category is \emph{connected}, limits in slice categories, and colimits in coslice categories are again reflected along the natural forgetful $U$: a particular application of this result, when $J$ is an ordinal regarded as a category, serves to state the following definition.
\end{remark}
\begin{definition}\label{mult.fs.trans}
Let $\alpha$ be an ordinal. A \emph{$\alpha$\hyp{}ary factorization system}, or \emph{factorization system in $\alpha$\hyp{}stages}, on $\CC$ consists of a monotone function $\alpha\to \textsc{fs}(\CC)\colon i\mapsto \fF_i$ such that, if we denote by
\[
\xymatrix{
X \ar[rr]\ar[dr]_{\overleftarrow{F}_i(f)}&& Y \\
& F_i(f)\ar[ur]_{\overrightarrow{F}_i(f)}
}
\]
the $\fF_i$\hyp{}factorization of $f\colon X\to Y$, we have the following two ``tame convergence'' conditions:
\begin{gather*}
\varprojlim_{i\in\alpha} \overrightarrow{F}_i(f) =
\varprojlim_{i\in\alpha} \var{F_i(f)}{Y} = \var{X}{Y};
\qquad \varinjlim_{i\in\alpha} \overleftarrow{F}_i(f) =
\varinjlim_{i\in\alpha} \var{X}{F_i(f)} = \var{X}{Y}\\
\varinjlim_{i\in\alpha} \overrightarrow{F}_i(f) =
\varinjlim_{i\in\alpha} \var{F_i(f)}{Y} = 1_Y;
\qquad \varprojlim_{i\in\alpha} \overleftarrow{F}_i(f) =
\varprojlim_{i\in\alpha} \var{X}{F_i(f)} = 1_X
\end{gather*}
(all the diagrams have to be considered defined in suitable slice and coslice categories) which can be summarized in the presence of ``extremal'' factorizations
\[
\xymatrix{
X \ar[rr]^f\ar@{=}[dr]_{\varprojlim_i \overleftarrow{F}_if}&& Y & X\ar[rr]^f\ar[dr]_{\varinjlim_i \overleftarrow{F}_if} && Y\\
& X\ar[ur]_{\varprojlim_i \overrightarrow{F}_if} &&& Y\ar@{=}[ur]_{\varinjlim_i \overrightarrow{F}_if}
}
\]
\end{definition}
\begin{theorem}[The multiple small object argument]
Let $\mcal{J}_1\subseteq \cdots \subseteq \mcal{J}_n$ be a chain of markings on $\CC$; if each class $\mcal{J}_\alpha$ has small domains then applying $n$ times the small object argument, the extensivity of the $\prescript{}{\perp}{((-)^\perp)}$ and $(\prescript{\perp}{}{(-)})^\perp$ closure operators entails that there exists a chain of factorization systems
\[
\big({}^\perp(J_n^\perp),J_n^\perp \big) \preceq \cdots \preceq \big({}^\perp(J_1^\perp),J_1^\perp \big)
\]
(the order relation is that of \adef \refbf{def:prefacts}).
\end{theorem}
\setlength{\epigraphwidth}{\DefaultEpigraphWidth}