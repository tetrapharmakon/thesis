\chapter{Introduction}
\thispagestyle{empty}
The present work re\hyp{}enacts the classical theory of \emph{$t$\hyp{}structures} reducing the classical definition given in \cite{BBDPervers,Kashiwara} to a rather primitive categorical gadget: suitable \emph{reflective factorization systems} (\adef \refbf{tortorfree}, \refbf{normal}), which we call \emph{normal torsion theories} following \cite{CHK,RT}. A  relation between these two objects has previously been noticed by other authors \cite{RT,HPS,Beligiannisreiten} on the level of homotopy categories. The main achievement of the present thesis is to observe and prove that this relation exists genuinely when the definition is lifted to the higher\hyp{}dimensional world where the notion of triangulated category comes from, \ie \emph{stable $(\infty,1)$\hyp{}categories}. 

Stable $(\infty,1)$\hyp{}categories provide a far more natural setting to interpret the language of homological algebra: the main conceptual aim of the present work is to give explicit examples of this meta\hyp{}principle.

To achieve this result, it seemed unavoidable to adopt a preferential model for $(\infty,1)$\hyp{}category theory: instead of working in a `model\hyp{}free' setting, we choose the ubiquitous dialect of Lurie's \emph{stable quasicategories}; discussing to which extent (if any) the results we prove are affected by this choice, and establishing a meaningful dictionary between the validity of the general statement \refbf{thm:rosetta} in various different flavours on $\infty$\hyp{}category theory occupies sections \refbf{model.stable} and \refbf{model.dep}; despite the fact that this is one of the most important issues from a categorical point of view, a rapid convergence of the present thesis into its final form has to be ensured; hence, we will defer a torough examination of the topic of model (in)dependence to subsequent works.

The first part of the thesis (\achap \refbf{chap:effe.esse}--\refbf{chap:tstruct}) builds (or rather, `reinterprets') the calculus of factorization in the setting of $\infty$\hyp{}categories. The desire to link this calculus with homological algebra and higher algebra deserves further explanation. 

The language of factorization systems proved to be ubiquitous inside and outside category theory (among various different applications now established in the mathematical practice, the `modern view' in algebraic topology revolves around the notion of orthogonality and lifting\fshyp{}extension problem, as it is said in the first pages of \cite{whitehead61elements}. The modern `synthetic' approach to homotopy theory inescapably relies on the notion of a (weak) factorization system (\cite{Qui,Dwyer1995,riehl2011algebraic}).

In light of this, finding `concrete' means of application for the calculus of factorization should be a natural step towards a popularization of this pervasive and deep language. And among all the various fields of application, homological algebra, a notable kind of `abelian' homotopy theory, should be the most natural test bench to measure the validity of this effort. Despite the intrinsic simplicity, almost a triviality, of \athm \refbf{thm:rosetta}, and despite the fact that the author feels he had failed at such an ambitious task, the pages you're about to read should be interpreted in this spirit.
\section*{Structure of the thesis}
The thesis is the results of a re\hyp{}organization and methodical arrangement of the papers \cite{FL0,heart,recol,infty-stab} (all written having my advisor as co\hyp{}author) that have appeared on the {\tt arXiv} since August 2014; the content is essentially unchanged; some sections and subsections (like \eg \refbf{sec:2ndglance}, \refbf{transfinite.case}, \refbf{sec:examples}, a renewed proof of \refbf{to.be.repeated.verbatim}, and \achap \refbf{chap:operat}) do not appear anywhere at the moment of writing\footnote{\today}, but contain little new material and serve as linking sections making the discussion more complete and streamlined, developing certain natural derivations of the basic theory which would have easily exceeded the average length of a research paper. 

Figure (\refbf{dependen}) below depicts the dependencies among the various chapters: a dashed line indicates a feeble logical dependence, whereas a thick line indicates a stronger one, unavoidable at first reading.

The first three chapters outline the main result of the present work, summarized as follows: 
\begin{quote}
For each stable $\infty$\hyp{}category $\CC$ there is a bijective correspondence between $t$\hyp{}structures on the triangulated homotopy category $\ho(\CC)$ and suitable orthogonal factorization systems on $\CC$ called \emph{normal torsion theories}.
\end{quote}
This constitutes the backbone and the basic environment in which every subsequent application (the theory of \emph{recollements} in stable $\infty$\hyp{}categories in \achap \refbf{chap:recol}, and Bridgeland's theory of \emph{stability conditions} in \achap \refbf{chap:stabilities}) takes place.

The main original contribution given in the present work is the `Rosetta stone' theorem proving the quoted remark above; this is the main result of \cite{FL0}, the only preprint that, at the moment of writing, has also been published by a peer\hyp{}reviewed journal.

There are several minor results following from the `Rosetta stone', like the fact that constructions one can perform on normal torsion theories are (at least to the categorically\hyp{}minded) more natural and canonical than the corresponding construction in homological algebra, done on bare $t$\hyp{}structures.
\subsection*{A word on model dependency}\label{sec:model.dep}
Ideally speaking, if there is an equivalence between two models for $\infty$\hyp{}categories (say, \emph{red} and \emph{blue} $\infty$\hyp{}categories), these two models both possess a notion of factorization systems and a calculus\footnote{By a `calculus' of factorization systems we na\"ively mean an analogue of the major results expressed in \achap \refbf{chap:effe.esse}, translated from the red to the blue model.} thereof; moreover, these two notions of factorization system correspond to each other under the equivalence of models. Turning this principle of equivalence and correspondence into a genuine theorem is often a subtle matter (apart from being inherently difficult and a delicate issue, this is perhaps due to the fact that the author is ignorant of how to retrieve such a result in the existing literature): it is however possible to recognize at least three different settings having each its own `calculus of factorization':
\begin{itemize}
\item stable model categories, where one can speak about \emph{homotopy factorization systems} following \cite{bousfield1977constructions,Joy}; this leads to the definition of a \emph{homotopy $t$\hyp{}structures} on stable model categories as suitable analogues of normal torsion theories in the set $\smallcap{hfs}(\cate{M})$ of homotopy factorization systems on a model category $\cate{M}$.
\item \dg\hyp{}categories, where we speak about enriched (over $\textsf{Ch}(\mathbf{k})$) factorization systems (see \cite{Day1974}); this leads to the definition of \emph{\smallcap{dg}\hyp{}$t$\hyp{}structures} as enriched analogues of normal torsion theories in the set of $\smallcap{dg-fs}(\D)$ of enriched factorization systems on a \dg\hyp{}category $\D$.
\item derivators, where we can define \emph{$t$\hyp{}derivators} via a (genuinely new) notion of factorization system on a derivator, and recognize the analogue of normal torsion theory in this setting. 
\end{itemize}
At the moment of writing, all these points are being studied, and will hopefully appear as separate results in the near future.
\subsection*{A word on the state of the art}
Drawing equally from homological algebra, algebraic geometry, topology and category theory, the present work has not a single, well\hyp{}defined flavour. Several sources of inspirations came from classical literature in algebraic topology \cite{HPS,MR0246294,shc}; several others belong to the classical and less classical literature on algebraic geometry \cite{Verdier1996,Brid,Bridge2,Bondal1995}; others belong to pure category theory \cite{RT, CHK,JanelidzeMarkl,Korostenski199357,Lucyshyn-Wright,zangurashvili2004several}, and others (see below) do not even belong to what is canonically recognized as mathematical literature.

The approach to the theory of $\infty$\hyp{}categories taken here will certainly appear rather unorthodox to some readers: \cite{HTT,LurieHA} have taught the author more about 1\hyp{}categories than he did about $\infty$\hyp{}categories. This, again, must be attributed to the ignorance of the author, which is more comfortable with the language of categories rather than with homotopy theory.
\subsection*{Notation and Conventions}
Categories (in the broad sense of `categories and $\infty$\hyp{}categories') are denoted as boldface letters $\CC,\D$ and suchlike, opposed to generic, variable simplicial sets which are denoted by capital Latin letters (this creates an extremely rare, harmless conflict with the same notation adopted for objects in a category: the context always allows us to avoid confusion); functors between categories are always denoted as capital Latin letters in a sufficiently large neighbourhood\footnote{The set $A$ of letters of the English alphabet admits an obvious monotone bijection $A\xto{\varphi}\Delta[26]$; define a distance on $A$ by putting $d(-,=) \defequal |\varphi(-) - \varphi(=)|$.} of $F,G,H,K$ and suchlike; the category of functors $\CC\to \D$ is denoted as $\text{Fun}(\CC,\D)$, $\D^{\CC}$, $[\CC,\D]$ (or, at the risk of being pedantic, as $\cate{(Q)Cat}(\CC, \D)$); morphisms in $\text{Fun}(\CC,\D)$ (i.e. natural transformations between functors) are often written in the Greek alphabet; the simplex category $\bDelta$ is the \emph{topologist's delta}, having objects \emph{nonempty} finite ordinals $\Delta[n]:=\{0<1\dots<n\}$ regarded as categories in the obvious way; we adopt \cite{HTT} as a main reference for $\infty$\hyp{}category theory, even if we can't help but confess that we profited from every single opportunity to deviate from the aesthetic of that book; in particular, we accept the (alas!) settled abuse to treat `quasicategory' and `$\infty$\hyp{}category' as synonyms; any other unexplained choice of notation belongs to folklore, or leans on common sense.

%When we want to refer to a particularly difficult, tricky, misleading or simply a key passage, we follow Bourbaki's famous notation \cite{Bourbaki} for a `tournant dangereux', denoted as a \verb|\textdbend| \marginpar{\textdbend} symbol like the one besides, usually put on a margin note.
A general working principle of stable $\infty$\hyp{}category theory is that homological algebra becomes easier and better motivated when looked at from a higher perspective\footnote{This rather operative and meta\hyp{}linguistic principle is sketched in our Appendix \refbf{chap:stable.cats}, where a complete proof of how triangulated category axioms follow from the `pullout axiom' \refbf{pullout.axiom} is worked out in full detail.}. To refer to this more natural environment we will often call \emph{the stable setting} any theory of stable $\infty$\hyp{}categories.

A not completely standard choice of notation is the following: each time a concept {\sf notion} appears together with its dual, we write {\sf co}\fshyp{}{\sf notion} to denote that we refer to {\sf notion} and {\sf conotion} at the same time. So, if we write `$\CC$ is a co\fshyp{}complete category' we mean that $\CC$ is \emph{both} complete and cocomplete, and if we write `co\fshyp{}limit' we are speaking about limits \emph{and} colimits at the same time.
\subsection*{About the \emph{kamon} on the titlepage}
The titlepage contains the \emph{kamon} of the Tachibana branch of Yoneda (!) family, traditionally drawn \cite{Samurai} as a tea\hyp{}berry (a $t$\hyp{}berry!) inside a circle (`{\japanese 丸茶の実}', \emph{mar\-{u} Cha no Mi}):
\begin{center}
\includegraphics[scale=.05]{marunitachibana}
\end{center}
\subsection*{A word on the way I drew diagrams}
Basically every existing package to draw commutative diagrams sucks. Starting from this undeniable truth, I spurred P. B\@. (see the acknowledgements) to write a {\tt tikzlibrary} capable of producing beautiful and readable diagrams on both the coders' and the readers' side. The result is repo\hyp{}ed \href{https://github.com/paolobrasolin/koDi}{here} under the name {\tt koDi}.

{\tt koDi} acts via three different kinds of command: a \verb|\lattice| environment, describing where to put the objects of the commutative diagram: each object of a \verb|\lattice| is included in a \verb|\obj #;| environment, and a command \verb|\mor|, which produces a chain of morphisms of variable length, all linked by arrows \verb|->| having different styles (basically those of \verb|TiKZ|).

Each \verb|\obj #;| environment allows the user to label the node with a tag which can be internally referred to inside a \verb|\mor| environment: so, for example, an intelligent way to rename the node $\gamma(\widehat{X}^\textsf{s},\lambda_0)$ is \verb|\obj (gX-l0):{\gamma(\widehat{X}^\textsf{s},\lambda_0)};| whereas an arrow $\gamma(\widehat{X}^\textsf{s},\lambda_0) \to Y$ can be written \verb|\mor gX-l0 -> Y;|.

Since an example is worth a thousand words, here is the code producing diagram (\refbf{good.factor}).
\begin{center}
\begin{tikzpicture}
\lattice[mesh]{
	\obj  (Xge):X_{\ge 0}; & \obj X; & \obj  (Xle):X_{<0}; 
			& \obj  (Xge+):X_{\ge 0}[1]; \\
	\obj  (Yge):Y_{\ge 0}; & \obj C; & \obj (Xle'):X_{<0}; 
			& \obj  (Yge+):Y_{\ge 0}[1]; \\
	\obj (Yge'):Y_{\ge 0}; & \obj Y; & \obj  (Yle):Y_{<0}; 
			& \obj (Yge+'):X_{\ge 0}[1]; \\
};
\mor Xge -> X -> Xle -> Xge+ 
		{\tau_{\ge 0}(f)[1]}:-> Yge+ 2- Yge+';
\mor Xge swap:{\tau_{\ge 0}(f)}:-> Yge 
		dashed,-> C {crossing over},-> Xle' 
		{\tau_{<0}(f)}:-> Yle -> Yge+';
\mor Yge 2- Yge' -> Y -> Yle;
\mor Xle 2- Xle' -> Yge+;
\mor[swap] X e_f:dashed,-> C m_f:dashed,-> Y;
\mor[dashed,near start] X f:r> Y;
\end{tikzpicture}
\end{center}
\begin{lstlisting}
\begin{kD}
\lattice[mesh]{
	\obj  (Xge):X_{\ge 0}; & \obj X; & \obj  (Xle):X_{<0}; 
			& \obj  (Xge+):X_{\ge 0}[1]; \\
	\obj  (Yge):Y_{\ge 0}; & \obj C; & \obj (Xle'):X_{<0}; 
			& \obj  (Yge+):Y_{\ge 0}[1]; \\
	\obj (Yge'):Y_{\ge 0}; & \obj Y; & \obj  (Yle):Y_{<0}; 
			& \obj (Yge+'):X_{\ge 0}[1]; \\};
\mor Xge -> X -> Xle -> Xge+ 
		{\tau_{\ge 0}(f)[1]}:-> Yge+ 2- Yge+';
\mor Xge swap:{\tau_{\ge 0}(f)}:-> Yge 
		dashed,-> C {crossing over},-> Xle' 
		{\tau_{<0}(f)}:-> Yle -> Yge+';
\mor Yge 2- Yge' -> Y -> Yle; \mor Xle 2- Xle' -> Yge+;
\mor[swap] X e_f:dashed,-> C m_f:dashed,-> Y;
\mor[dashed,near start] X f:r> Y;
\end{kD}
\end{lstlisting}